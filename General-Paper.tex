\documentclass[12pt]{article}

\usepackage{setspace}

\usepackage{graphicx, color, fancyhdr, tikz-cd, enumitem, framed, adjustbox, bbm, upgreek, xcolor, manfnt}
\usepackage[framed,thmmarks]{ntheorem}
\usepackage[framemethod=tikz]{mdframed}
\usepackage{hyperref}

\hypersetup{
	colorlinks = true,
	linkcolor = [rgb]{0,0,0.5},
	citecolor = [rgb]{0.6,0,0},
	urlcolor = [rgb]{0,0,0.5}
}
\usepackage[style=alphabetic, bibencoding=utf8]{biblatex}
%Set the bibliography file
\bibliography{sources}

%lots of font stuff
\usepackage[T1]{fontenc}
\usepackage[urw-garamond]{mathdesign}
\usepackage{garamondx}
\let\mathcal\undefined
\newcommand{\mathcal}[1]{\text{\usefont{OMS}{cmsy}{m}{n}#1}}

%Document-Specific includes
\usepackage{ytableau}
\usepackage{mathtools}
\usepackage{scalerel}

%Replacement for the old geometry package
\usepackage{fullpage}
\usepackage{amsmath}

%Input my definitions
\input{./mydefs.tex}

%Shade definitions
\theoremindent0cm
\theoremheaderfont{\normalfont\bfseries} 
\def\theoremframecommand{\colorbox[rgb]{0.9,1,.8}}
\newshadedtheorem{defn}[thm]{Definition}

%Set apart my theorems and lemmas and such
\surroundwithmdframed[outerlinewidth=0.4pt,
  innerlinewidth=0pt,
  middlelinewidth=1pt,
  middlelinecolor=white,
  topline=false,bottomline=false,rightline=false,leftmargin=2em]{thm}
\surroundwithmdframed[outerlinewidth=0.4pt,
  innerlinewidth=0pt,
  middlelinewidth=1pt,
  middlelinecolor=white,
  topline=false,bottomline=false,rightline=false,leftmargin=2em]{lem}
\surroundwithmdframed[outerlinewidth=0.4pt,
  innerlinewidth=0pt,
  middlelinewidth=1pt,
  middlelinecolor=white,
  topline=false,bottomline=false,rightline=false,leftmargin=2em]{cor}
  \surroundwithmdframed[outerlinewidth=0.4pt,
  innerlinewidth=0pt,
  middlelinewidth=1pt,
  middlelinecolor=white,
  topline=false,bottomline=false,rightline=false,leftmargin=2em]{prop}
  \surroundwithmdframed[outerlinewidth=0.4pt,
  innerlinewidth=0,
  middlelinewidth=1pt,
  middlelinecolor=white,
  topline=false,bottomline=false,rightline=false,leftmargin=2em]{rmk}

%%%%%%%%%%%%%%%%%%%%%%%%%%%%%%%%%%%%%%%%%%%%%%%%%%%%%%%%%%%%%%%%%%%%%%
%%%%%%%%%%%%%%%%%%%%%%% Customize Below %%%%%%%%%%%%%%%%%%%%%%%%%%%%%%
%%%%%%%%%%%%%%%%%%%%%%%%%%%%%%%%%%%%%%%%%%%%%%%%%%%%%%%%%%%%%%%%%%%%%%

%header stuff
\setlength{\headsep}{24pt}  % space between header and text
\pagestyle{fancy}     % set pagestyle for document
\lhead{General Exam Paper} % put text in header (left side)
\rhead{Nico Courts} % put text in header (right side)
\cfoot{\itshape p. \thepage}
\setlength{\headheight}{15pt}
%\allowdisplaybreaks

% Document-Specific Macros
\newcommand*{\ttc}{{\large $\triangle$}\kern-0.86em\raisebox{0.3ex}{$\scaleobj{0.78}\otimes$}\hspace{1ex}}
\DeclareMathOperator{\Spc}{Spc}
\DeclareMathOperator{\Pol}{Pol}

\begin{document}
%make the title page
\title{Schur Duality and Strict Polynomial Functors\\\vspace{1ex} \normalsize General Exam Paper}
\author{Nico Courts\footnote{University of Washington, Seattle. Email: ncourts@uw.edu}}
\date{Exam Presentation: March 10th, 2020, 10:15am, DEN 213}
\maketitle

\begin{abstract}
	We begin by going through a considerable amount of domain knowledge concerning representations of $\GL_n$,
	representations of $\frakS_n$, tracing the development from the classical study of group representations by Schur
	and Weyl and the transformation of this theory in the more robust language of affine group schemes. From there,
	the story takes on a more categorical flavor as we discuss different manifestations of polynomial representations of $\GL_n$, 
	following the work of Friedlander and Suslin as well as Krause, Aquilino, and Reischuk. In the latter case, 
	we show how they determined that the Schur-Weyl functor is monoidal, opening up the theory to the machinery of 
	monoidal categories. We take some time to develop the theory of tensor triangulated geometry from Balmer as well 
	as discuss some standard constructions necessary for the theory. We end our paper by talking about how all of this work 
	comes together to elucidate some problems at the boundaries of modern representation theory as well as techniques with 
	which we can solve them.\vspace{0.5in}

	{\begin{center}
		\footnotesize The most up-to-date version of this paper can be downloaded at the following link:\\
		\url{https://github.com/NicoCourts/General-Exam-Paper/raw/master/General-Paper.pdf}
	\end{center}}
\end{abstract}

\newpage
%\setcounter{tocdepth}{3}
\tableofcontents

\newpage
\section{Introduction}
\subsection{Issai Schur and polynomial representations}
The story of this project (more-or-less) begins with Schur's doctoral thesis \cite{schur-thesis} in which he defines
the polynomial representations of the group $\GL_n(k)$---a theory which he developed more completely in his later paper \textit{\"Uber die 
rationalen Darstellungen der allgemeinen linearen Gruppe}\footnote{English: \textit{On the rational representations of the general linear group}}
\cite{schur-rational}. In these papers, Schur develops the idea of a \textbf{polynomial representation of $\GL_n(k)$},
meaning a (finite dimensional) representation where the coefficient functions of the representing map 
\[\rho:\GL_n(k)\to \GL_m(k)\]
is polynomial in each coordinate. For example, the map sending 
\[A=\begin{pmatrix}
	a&b\\
	c&d
\end{pmatrix}\mapsto \begin{pmatrix}
	a^2d-abc & acd-c^2b & 0\\
	abd-b^2c & ad^2-bcd & 0\\
	0 & 0 & ad-bc
\end{pmatrix}=\rho(A)\]
is a three-dimensional polynomial representation of $\GL_2(\bbR)$.

The block-diagonal form above demonstrates a direct sum decomposition of our representation into two parts: one two-dimensional homogeneous degree 3
and one one-dimensional homogeneous degree 2 (in the entries of $A$). A result in \cite{schur-thesis} tells us that, in fact, this can always be done: 
if $V$ is a polynomial representation of $\GL_n(k)$,
then $V$ decomposes as a direct sum of representations 
\[V=\bigoplus_\delta V_\delta\]
where each $V_\delta$ is a polynomial representation where the coefficient functions are \textit{homogeneous degree $\delta$}. 
This allows us to focus our attention to the structure of these $V_\delta$ as the fundamental building blocks of the theory.

The key insight made in this theory comes from the observation that the vector space 
\[E^{\otimes d}\eqdef (k^n)^{\otimes d}\]
can made into a $(\GL_n(k),\frakS_d)$-bimodule in a very natural way, and that this bimodule gives us a way to relate 
$\rmod {\frakS_d}$ with $\lmod {\GL_n(k)}$ via the so-called \textbf{Schur-Weyl functor.}

\subsection{A more modern treatment: affine group schemes}
Schur's discovery, while already interesting enough by itself, takes on a new level of depth when one puts 
things in the right context. More modern mathematicians realized that this phenomenon is best stated as a property of \textit{affine group schemes}, rather than as groups. 

When put into this context, the classification of rational representations (group scheme morphisms into $\GL(V)$) comes as given and the classification comes as a very 
natural condition put on the corresponding map between coordinate algebras. This better motivates many of the constructions 
that Schur made and opens up his theory to analysis using the tools of category theory and algebraic geometry.

\subsection{The Schur-Weyl functor}
Clearly a connection between representations of two groups that are so ubiquitous in group theory and math in general 
is a stunning observation, and much effort has been expended since the late 20th century to study this functor and its 
properties---especially in how it relates the representation theory of these two groups. 

For instance, Friedlander and Suslin \cite{friedlander-suslin}
originally discussed the idea of \textbf{strict polynomial functors} and showed that the category of repesentations 
of the Schur algebra $S(n,d)$ was equivalent to the category $\calP_d$ of homogeneous degree $d$ strict polynomial functors.

In later work, Krause \cite{krause-strict-poly-func} used an alternative construction of $\calP_d$ as the category of
of reprsentations of the $d$-divided powers of the category of finitely generated projective $k$-modules. This category is 
denoted $\Gamma^d P_k$ (or $\Gamma^d_k$ for short) and his version of strict polynomial functors is $\Rep \Gamma^d_k$. The upshot of this defintion is that 
the polynomial structure we desire is better encapsulated in the domain category, rather than placing awkward conditions on the functors themselves.
This also enables Krause to define monoidal structure on $\Rep \Gamma^d_k$ using the fact that presheaves are canonical limits of representable presheaves. 

Krause's students Aquilino and Reischuk, in their paper \cite{aquilino-reischuk}, prove, among other facts, that 
under these natural monoidal structures the Schur-Weyl functor is in fact monoidal. This puts the theory of representations 
of these groups and algebras firmly in the realm of monoidal categories, opening up the area to new questions using 
tools from category theory.

\subsection{Tools and further directions}
Sections 5,6, and 7 are devoted to reproducing the core aspects of some tools that can be used in solving problems in representation theory
including methods from homological algebra (the derived category) and triangulated categories (the Balmer spectrum). In fact, these two 
tools coalesce to give a way to analyze otherwise the (sometimes unweildy) categories $\Rep S(n,r)$. We finish our discussion with some ideas 
of how to proceed from this knowledge to solving new problems.

\subsection{Notation and conventions}\label{subsec:notation}
Throughout this paper we will define $k$ to be an infinite field (not necessarily of characteristic zero or algebraically closed unless otherwise noted).
d%Let $\Alg_k$ be the category of $k$-algebras and $\Grp$ denote the category of groups with homorphisms.

We will use $\Gamma=\GL_n$ to denote the affine group scheme $\Hom_{\Algk}(k[x_{ij}|1\le i,j\le n]_{\det},-)$ and $\GL_n(k)$ to denote either the $k$-points of $\GL_n$
or the abstract group, depending on which viewpoint best suits the discussion.


%%%%%%%%%%%%%%%%%%%%%%%%%%%%%%%%%%%%%%%%%%%%%%%%%
%%%%%%%%%%%%%%%%%%%%%%%%%%%%%%%%%%%%%%%%%%%%%%%%%
%%%%%%%%%%%%%%%%%%%%%%%%%%%%%%%%%%%%%%%%%%%%%%%%%
%%%%%%%%%%%%%%%%%%%%%%%%%%%%%%%%%%%%%%%%%%%%%%%%%
%%%%%%%%%%%%%%%%%%%%%%%%%%%%%%%%%%%%%%%%%%%%%%%%%

\newpage
\section{The classical theory: Representations of \texorpdfstring{$\GL_n$}{GLn} and of \texorpdfstring{$\frakS_n$}{Sn}}
We begin by detailing the theory behind the (polynomial) representations of $\GL_n$ as well as the representations of $\frakS_n$ to 
familiarize ourselves with the classical representation theory associated to these groups.

\subsection{Representations of \texorpdfstring{$\frakS_n$}{Sn}}
The representation theory for $\frakS_n$ over the complex numbers is a subject that has been widely studied by representation theorists 
and combinatorialists alike for over a century. Before we dive into specifics, we write down the idea originally worked out by Frobenius \cite{frobenius-charaktere}
in his work in 1900 on the characters of $\frakS_n$:
\begin{thm}\label{thm:frob-conj}
	The conjugacy classes (and thus isomorphism classes of irreducible representations over $\bbC$) of $\frakS_n$
	are in bijection with partitions of $n$.
\end{thm}

\begin{rmk}
	In what follows we attempt to give a tangible, minimalistic overview of the nicest case of representations of $\frakS_n$.
	Some of the arguments below appeal more to intuition and examples than rigor, but we feel this better prepares the reader 
	for computations in $\frakS_n$ without being weighed down by unnecessary details. This can all be made rigorous, of course, 
	at the expense of some clarity and conciseness. 
\end{rmk}
Let's get some sense first about how we can relate these two ideas by recalling some easy lemmas from 
group theory. Recall that each element of $\frakS_n$ can be written 
as a product of disjoint cycles and that this representation is unique up to reordering the cycles. We can make this 
representation unique by writing each cycle as one starting at its least element and then ordering the cycles by these least elements. 
For instance, the permutation (in two-line notation)
\[\sigma=\begin{pmatrix}1&2&3&4&5&6&7&8\\ 2&1&7&5&3&8&4&6\end{pmatrix}\in\frakS_8\]
is represented uniquely in this way as the product of cycles:
\[\sigma=(1\,2)(3\,7\,4\,5)(6\,8).\]

The next observation to recover: if $\tau,\eta\in\frakS_n$ and $\tau=(\tau_1\,\tau_2\,\cdots\,\tau_k)$ is a cycle, 
\[\eta^{-1}\tau\eta=(\eta(\tau_1)\,\eta(\tau_2)\,\cdots\,\eta(\tau_k)).\]
We can see this demonstrated in the computation
\begin{align*}
	(1\,3\,5)\sigma(1\,3\,5)^{-1}&=(1\,3\,5)(1\,2)(1\,5\,3)(1\,3\,5)(3\,7\,4\,5)(1\,5\,3)(1\,3\,5)(6\,8)(1\,5\,3)\\
	&=(3\,2)(5\,7\,4\,1)(6\,8)\\
	&=(1\,5\,7\,4)(2\,3)(6\,8)
\end{align*}
The important observation here is that the ``shape'' (the lengths of the cycles when written as a product of disjoint cycles) is preserved under conjugation.
In fact,
\begin{lem}\label{lem:conj-classes}
	The conjugacy classes of $\frakS_n$ are in one-to-one correspondence with the partitions of $n$.
\end{lem}
\begin{prf}
	Let $\scrP_n$ denote the partitions of $n$ and let $C_n$ denote the conjugacy classes in $\frakS_n$.
	We construct the set map 
	\[\varphi:C_n\to \scrP_n\]
	by sending a conjugacy class to the weakly-decreasing list of cycle lengths (including trivial cycles, if necessary). For instance in $\frakS_8$,
	\[(1\,5\,3)(2\,7)\qquad\text{cooresponds to}\qquad (3,2,1,1,1).\]

	The results cited and demonstrated above shows that this map is well defined---conjugation preserves the cycle length in 
	the disjoint cycle representation of an element. Furthermore if $p\in \scrP_n$ is a partition, the adjoint action of $\frakS_n$ on $\varphi^{-1}(p)$
	is transitive, since if two elements have the same cycle lengths when written as disjoint cycles, we can line 
	the cycles up according to length and act by the permutation that ``puts labels in the right place''. If we look at $\sigma$ and the 
	element we found by conjugation above, we have 
	\begin{align*}
		(1\,2)(3\,7\,4\,5)(6\,8)\\
		(3\,2)(5\,7\,4\,1)(6\,8)
	\end{align*}
	where we notice that $1\mapsto 3$, $3\mapsto 5$, and $5\mapsto 1$, meaning that the cycle that takes the top element to the 
	bottom is $(1\,3\,5)$---although of course we already knew that. Another example are the elements $(1\,4\,5)$ and $(3\,2\,1)$. Here we want $1\mapsto 3$, $4\mapsto 2$ and $5\mapsto 1$.
	Thus one element that takes the first to the second is $(5\,1\,3)(2\,4)$. This demonstrates that the action is not faithful since we could also act by $(5\,1\,3\,7)(2\,4)(6\,8)$ and get the same element. 
	the important fact here is that $\frakS_n$ acts transitively on the elements of $1,\dots,n$, so there is always such an element.

	The surjectivity of this map is clear since we can write from any partition of $n$ a product of disjoint cycles corresponding to 
	this partition (which then must map to it) and injectivity is clear since the disjoint cycle representation is unique (up to reordering cycles, which doesn't affect the image $\varphi(x)$).
	This proves the lemma.
\end{prf}

From here, the standard result that (again, over $\bbC$) the conjugacy classes of a group are in bijection with the irreducible representations finishes demonstrating how 
theorem \ref{thm:frob-conj} is true. But a simple set bijection belies the depth of the connection here. 

\subsubsection{Construction of the irreducible representations}
It is possible, through the idea of a Young symmetrizer, to directly link a Young diagram to the corresponding 
irreducible representation. Throughout this subsection, we will be relying on facts developed in \cite{fulton-harris}, 
although there is also a more complete combinatorial picture painted in Fulton's book \textit{Young Tableaux} \cite{fulton-tableaux}. 

To begin our discussion, consider the trivial representation within the left regular representation $\bbC\frakS_n$: 
it is a one-dimensional subspace spanned by the element 
\[x_1=\sum_{\sigma\in\frakS_n}\sigma\]
where you can see that this element is fixed by left multiplication, demonstrating that it has the trivial $\frakS_n$ action.
The subspace spanned by the element 
\[x_{-1}=\sum_{\sigma\in\frakS_n}(-1)^{\operatorname{sign}(\sigma)}\sigma\]
is the sign representation, where an element with sign 1 acts by -1. This is because
\[\operatorname{sign}(\tau\sigma)=\operatorname{sign}(\tau)+\operatorname{sign}(\sigma)\pmod{2}.\]

It ends up that these two representations form the two ``endpoints'' of the representation theory of $\frakS_n$. The exact sense in which this is 
true is captured through Young diagrams! For the purposes of illustration, let us return to our example above of $\frakS_8$. Here the trivial and sign representations 
correspond (repsectively) to the tableaux
\[\ytableausetup{smalltableaux,centertableaux}\ydiagram{8}\qquad\text{and}\qquad\ydiagram{1,1,1,1,1,1,1,1}\]
which, in turn, correspond to partitions $(8)$ and $(1,1,1,1,1,1,1,1)$ of $8$. The way to make this connection is through the definition 
of a \textit{Young symmetrizer:}
\begin{defn}
	Fix an $n\ge 1$ and let $\lambda$ be a partition of $n$. Then define two elements of $\bbC\frakS_n$, $a_\lambda$ and $b_\lambda$ in the following way:
	\[a_\lambda=\sum_{\sigma\in R(T_\lambda)}\sigma\qquad\text{and}\qquad b_\lambda=\sum_{\sigma\in C(T_\lambda)}(-1)^{\operatorname{sign}(\sigma)}\sigma\]
	where $T_\lambda$ is the Young diagram corresponding to $\lambda$ and given some labeling (say the canonical one that labels boxes left-to-right and top-to-bottom)
	$R(T_\lambda)$ (resp. $C(T_\lambda)$) denote the subgroups of $\frakS_n$ stabilizing the rows (resp. columns) of $T_\lambda$ under the action of $\frakS_n$ on the labels.

	Then the \textbf{Young centralizer} of $\lambda$ is 
	\[c_\lambda=a_\lambda b_\lambda\in\bbC\frakS_n.\]
\end{defn}

The canonical fillings of the diagrams above are\footnote{Here you can see yet another connection to disjoint cycle representations. Notice, under the map 
defined in lem.~\ref{lem:conj-classes}, that the conjugacy class corresponding to the trivial representation is the one consisting of ``long'' (length $n$) cycles. Using the 
unique ordering on products of disjoint cycles described after the statement of thm.~\ref{thm:frob-conj}, we can identify fillings with long cycles and we see 
that the cycle $(1\,2\,3\,4\,5\,6\,7\,8)$ is the only one in ``standard form'' in that it gives us a standard Young tableau. The complexity of the Young diagram (meaning how many 
different standard fillings it admits) gives us some information about the dimensionality of the corresponding irreducible representation, as we will see later.}
\[\ytableaushort{12345678}\qquad\text{and}\qquad\ytableaushort{1,2,3,4,5,6,7,8}\]
and so since the column stabilizer of the first diagram is trivial and the row stablizer is everything,
\[c_{(8)}=\left(\sum_{\sigma\in R(T_{(8)})}\sigma\right)\left(\sum_{\sigma\in C(T_{(8)})}(-1)^{\operatorname{sign}(\sigma)}\sigma\right)=\sum_{\sigma\in\frakS_n}\sigma=x_1\]
and since the roles of the column and row stabilizing elements are reversed for the sign representation, we get 
\[c_{(1,1,1,1,1,1,1,1)}=\left(\sum_{\sigma\in R(T_{(1,1,1,1,1,1,1,1)})}\sigma\right)\left(\sum_{\sigma\in C(T_{(1,1,1,1,1,1,1,1)})}(-1)^{\operatorname{sign}(\sigma)}\sigma\right)=\sum_{\sigma\in\frakS_n}(-1)^{\operatorname{sign}(\sigma)}\sigma=x_{-1}.\]

That the Young symmetrizers correspond with the elements spanning the corresponding representations of is no coincidence! 
\begin{defn}
	The module $V_\lambda$ is a $\bbC\frakS_n$-module generated by the Young symmetrizer $c_\lambda$.
\end{defn}
Notice that the dimension of each $V_\lambda$ is determined by number of linearly-independent elements that lie in the orbit of $c_\lambda$. We compute another example that 
gives a general pattern:
\begin{ex}
	Let $\lambda=(2,1,1)$ be the partition of $5$, so 
	\[T_\lambda=\ydiagram{2,1,1}.\]
	Given the canonical filling of $T_\lambda$,
	\[\ytableaushort{12,3,4},\]
	we have 
	\[a_\lambda=e+(1\,2)\qquad\text{and}\qquad b_\lambda=e-(1\,3)-(1\,4)-(3\,4)+(1\,3\,4)+(1\,4\,3)\]
	and so we can compute that the Young symmetrizer for this partition is 
	\[c_\lambda=e-(1\,3)-(1\,4)-(3\,4)+(1\,2)+(1\,3\,4)+(1\,4\,3)-(2\,1\,4)-(1\,2)(3\,4)+(1\,3\,4\,2)+(1\,4\,3\,2)\]
	and one can show (c.f. \cite[48]{fulton-harris}) that this is the representation $V\wedge V$ where $V$ is the standard representation 
	(the complement of copy of the trivial representation spanned by the vector $(1,1,1,1)\in\bbC^4$ under the usual embedding of 
	$\frakS_4$ in $\GL_4$ as permutation matrices).
\end{ex}

This completes the description of the representations of $\frakS_n$ over $\bbC$, but in fact everything we have done here holds over the splitting field 
of $\frakS_n$, that is, the minimal field such that representations don't split further under field extension. We haven't proved here that
\begin{enumerate}
	\item the $V_\lambda$ are irreducible; or 
	\item the $V_\lambda$ are pairwise nonisomorphic,
\end{enumerate}
but one can look up any of the standard texts (including the ones cited in this section) for more rigorous and thorough treatments of these facts.

\subsection{Polynomial representations of \texorpdfstring{$\Gamma$}{Gamma}}
Let $k$ be an infinite field\footnote{In some cases we will be able to allow $k$ to be a ring, but we will still need that $k$ be infinite so that 
polynomials over it are determined by their values.} and $\Gamma$ be the affine group scheme $\GL_n$. This can be thought of as the functor 
\[\Gamma:\Alg_k\to \Grp\quad\text{sending}\quad A\mapsto \GL_n(A).\]
Then 
\begin{defn}
	A (finite dimensional) \textbf{representation} of $\Gamma$ is a (finite dimensional) vector space $V$ along with a group scheme homomorphism
	\[\rho:\Gamma\to \GL(V)\eqdef \Aut(V\otimes_k -)\]
\end{defn}
\begin{rmk}
	Representations of (the group, which can be thought of as the $k$ points of the $k$-scheme) $\GL_n(k)$ can be, in general, ``analytic.'' One can check that the map 
	\[\rho:k^\times=\GL_1(k)\to \GL(k^2)\qquad\text{via}\qquad x\mapsto\begin{pmatrix}
		1 & \ln |x|\\ 0 & 1
	\end{pmatrix}\]
	gives a group homomorphism (and thus representation) between these two groups, but the logarithm makes this representation decidedly \textit{not algebraic.}

	This leads to slightly more awkward definitions in more classical treatments of the theory (e.g. \cite{green}), where one has to 
	specifically rule these out. The upshot to using a more algebro-geometric approach is that we start off in the world of rational maps where 
	such a representation doesn't make sense.
\end{rmk}
Recall that the affine group scheme $\GL_n$ is represented by the algebra 
\[k[x_{ij}]_{\det}\]
where $1\le i,j\le n$ and $\det$ is the polynomial corresponding to the determinant of the matrix $A=(x_{ij})$. Since $\GL_n$ is an affine scheme,
we know that the global functions are 
\[k[\Gamma]=k^\Gamma\cong k[x_{ij}]_{\det}\]
where we will (for clarity) use the notation $c_{ij}:\Gamma\to k$ to denote the function corresponding to $x_{ij}$.

\begin{defn}\label{def:poly-rep}
	A \textbf{polynomial representation} of $\Gamma$ is a representation $\rho:\Gamma\to \GL(V)$ (where $\dim_k V=m$) such that (on points) the structure maps (\ref{rmk:structure-maps}) of 
	\[\rho_A:\Gamma(A)\to \GL(V)(A)\cong\GL_m(A)\] 
	are polynomials in the functions $c_{ij}:\Gamma(A)\to A$ that extract the $(i,j)^{th}$ entry.

	If all the structure maps are homogeneous of degree $r$ for some fixed $r$, we say that $\gamma$ is a \textbf{homogeneous degree $r$
	polynomial representation of $\Gamma$.}
\end{defn}
\begin{rmk}\label{rmk:structure-maps}
	Recall (or learn for the first time!) that the \textit{structure maps} of a representation $(\rho,V)$ are a collection 
	of maps $r_{ij}$ for $1\le i,j,\le n$ from $\Gamma$ to $k$ such that for all $g\in \Gamma$:
	\[g\cdot v_i=\sum_{j=1}^n r_{ij}(g)v_j\]
	where we have picked a basis $\{v_1,\dots,v_n\}$ for $V$. Of course changing basis may change our 
	$r_{ij}$, but their \textbf{span} $\langle r_{ij}\rangle$ is an invariant of the representation.
\end{rmk}
\begin{defn}\label{def:Mnr}
	Let $\Pol_k(n)=\Pol(n)$ be the collection of all polynomial representations of $\GL_n$ and let $\Pol_k(n,r)=\Pol(n,r)$ 
	be the collection of all homogeneous degree $r$ polynomial representations of $\GL_n$.
\end{defn}
It is the \textit{polynomial} representations that we will concern ourselves with in the following sections. 

\subsubsection{Reducing scope}
In what follows we (temporarily) restrict to the case of considering the $R$-points of the scheme, where $R\in\Alg_k$. 
Using some of our familiar friends from representation theory (as well as some clever twists), 
we can simplify this picture considerably by proving the following structural result:
\begin{thm}[{\cite[pp.7-10]{schur-thesis}}]\label{thm:decomp}
	Every polynomial representation $V$ of the group $\GL_n(R)$ (where $R$ is an algebra over an infinite field $k$) decomposes as a direct sum 
	\[V\cong\bigoplus_{\delta\in\bbN}V_\delta\]
	where $V_\delta$ is a \textit{homogeneous} polynomial representation of degree $\delta.$
\end{thm}
Clearly, then, it suffices to understand the \textit{homogeneous degree $r$} polynomial representations of $\Gamma(A)$ if we are looking
to understand the larger structure.

We begin with a useful lemma extracted from a proof in \cite{schur-thesis} echoing the general theory of 
orthogonal decomposition of Artinian algebras.
\begin{lem}\label{lem:orth-decomp}
	Let $C_0,\dots,C_m\in M_n(R)$ be mutually orthogonal idempotent matrices that sum to the identity. That is, 
	\[I_n=\sum_i C_i\quad\text{and}\quad C_iC_j=\delta_{ij}C_i\]
	for all $0\le i,j\le m$. Then there exists an invertible matrix $P$ such that for some positive integers $d_0,\dots,d_m$ with $\sum_k d_k=n$ and for all $i$,
	\[P^{-1}C_iP=\begin{pmatrix}
		\mathbf{0}_{N_i} & &\\
		& I_{d_i} & \\
		& & \mathbf{0}_{M_i}
	\end{pmatrix}\]
	Where $N_i=\sum_{0\le j<i}d_j$ and $M_i=n-d_i-N_i$
\end{lem}
\begin{prf}[of lem~\ref{lem:orth-decomp}]
	We set $S_k=\{C_0,C_1,\dots,C_k\}$ and we proceed by induction on $k$. When $k=0$, $S_k=\{C_0\}$. Now since 
	$C_0^2=C_0$, we get that 1 and 0 are the only eigenvalues of $C_0$, so there is an $r\times r$ matrix $P_0$ 
	and a positive integer $d_0$ such that
	\[P_0^{-1}C_0P_0=\begin{pmatrix}
		I_{d_0} & \\
			& \mathbf{0}_{n-d_0}
	\end{pmatrix}.\]
	which establishes the base case.

	Now assume that we have a matrix $P_{k-1}$ such that this property holds for all elements of $S_{k-1}$.
	Define, for each $0\le i\le k$, 
	\[C_i'\eqdef P^{-1}_{k-1}C_iP_{k-1}\]
	and since the $C_k$ is assumed to be orthogonal to all other $C_i$,
	\[C_k'=\begin{pmatrix}
		\mathbf{0}_{N_k} & \\
		& D_k
	\end{pmatrix}\]
	for some $D_k$.

	Now by properties of block diagonal matrices, we have 
	\[D_k^2=D_k\]
	so the eigenvalues of $D_k$ are again one and zero. Thus there is an invertible $Q\in \GL_{n-N_k}$ such that 
	\[Q^{-1}D_kQ=\begin{pmatrix}I_{d_k} &\\ & \mathbf{0}_{M_k}\end{pmatrix}\]
	and so by setting
	\[P_k\eqdef P_{k-1}\begin{pmatrix}I_{N_k} &\\ & Q\end{pmatrix}\]
	we can define
	\[C''_i\eqdef P_k^{-1}C_i P_k=\begin{pmatrix}I_{N_k} &\\ & Q\end{pmatrix}^{-1}C'\begin{pmatrix}I_{N_k} &\\ & Q\end{pmatrix}\]
	for $0\le i\le k$, we see immediately that $C_i'=C_i''$ for $0\le i<k$ and furthermore 
	\[C_k''=\begin{pmatrix}
		\mathbf{0}_{N_k} & \\
		& Q^{-1}D_kQ
	\end{pmatrix}=\begin{pmatrix}
		\mathbf{0}_{N_k} & &\\
		& I_{d_k} & \\
		& & \mathbf{0}_{M_k}
	\end{pmatrix}\]
	completing the inductive step. This this result holds for all $S_i$ and in particular for $S_m$, so the result is proven.
\end{prf}
As well as another result on a special class of commuting block diagonal matrices:
\begin{lem}\label{lem:block-diag}
	Let $R\in\Alg_k$ ($k$ be an infinite field) and let $A$ be a block diagonal matrix over $k$ of the form
	\[A=\operatorname{diag}(x^mI_{d_m},x^{m-1}I_{d_{m-1}},\dots,I_{d_0})\]
	where $d_i$ is (clearly) the dimension of the $(m-i)^{th}$ block and let $B$ be any matrix that commutes with $A$
	for every choice of $x\in k$. Then $B$ is block diagonal of the same shape as $A$. 
\end{lem}
\begin{prf}[of lem~\ref{lem:block-diag}]
	We proceed by comparing the entries in $AB$ and $BA$: notice that 
	\[(AB)_{ij}=\sum_k A_{ik}B_{kj}=A_{ii}B_{ij}=x^aB_{ij}\]
	and 
	\[(BA)_{ij}=\sum_k B_{ik}A_{kj}=B_{ij}A_{jj}=x^bB_{ij}.\]
	We will show that if the $(i,j)^{th}$ postion is not in one of the blocks of $A$, then it is zero.

	But if $(i,j)$ is not in one of the blocks of $A$, then the nonzero element in the $i^{th}$ row and the nonzero element in 
	the $j^{th}$ column ($x^a$ and $x^b$ in the above equations) are not the same! Since $x$ is arbitrary, this forces $B_{ij}=0$,
	so $B$ is block diagonal with blocks the same as $A$.
\end{prf}
\begin{rmk}
	Notice that in the above proof we used implicitly that there is an $x\in R$ such that for all $a,b$
	\[x^a=x^b\quad\Rightarrow\quad a=b\]
	which is true since $k$ is infinite. This can cause a problem for finite fields since, for instance, every element in $\bbF_p$ satisfies $x^p=x$.
\end{rmk}

And finally using these two lemmas allows us to prove our main result:
\begin{prf}[of thm~\ref{thm:decomp}]
	We recreate the argument in Schur's thesis, translated from German and reinterpreted in more modern parlance. 
	
	Let $(\rho,V)$ be a polynomial representation of $\GL_n(R)$ with $\dim_k V=r$. Then let $x\in R^\times$ be arbitrary (thought of as an indeterminate)
	and consider the matrix $xI_n\in\Gamma$. The image of this matrix under $\rho$ is a matrix 
	\[\rho(A)=\begin{pmatrix}
		p_{11}(x) & \cdots & p_{1r}(x)\\
		\vdots & \ddots & \vdots\\
		p_{r1}(x) & \cdots & p_{rr}(x)
	\end{pmatrix}\]
	where each $p_{ij}$ is a polynomial in $x$. Let $m=\max_{i,j}\deg p_{ij}$, and this gives us a decomposition 
	\[\rho(A)=x^m C_0+x^{m-1}C_1+\cdots+ xC_{m-1}+C_m\]
	where each $C_i$ is an $r\times r$ matrix.

	Let $y$ be another indeterminate and $B=yI_n$. By virtue of being a representation of $\GL_n(R)$, we get 
	\[\rho(A)\rho(B)=\rho(xI_n)\rho(yI_n)=\rho(xyI_n)=\rho(AB)\]
	and using this setup we prove the following result: 
	\[\text{For all $0\le i,j\le m$, with the $C_l$ as above,}\quad C_iC_j=\delta_{ij}C_i\]
	That this is true can be established by comparing coefficients in the equation
	\begin{align*}
		\rho(AB)&=\rho(A)\rho(B)\\
		C_0(xy)^m+\cdots+C_i(xy)^{m-i}+\cdots+C_m&=C_0^2x^my^m+\cdots+C_iC_jx^{m-i}y^{m-j}+\cdots+C_m^2
	\end{align*}
	Indeed, we immediately get that $C_i=C_i^2$ and furthermore the coefficients on $x^iy^j$ when $i\ne j$ give us
	\[0=C_{m-i}C_{m-j}.\]
	
	Thus we have shown that the $C_i$ form a set of orthogonal idempotent matrices and evaluating our original equation at $x=1$,
	we get (since $\rho$ is a homomorphism)
	\[I_r=1C_0+\cdots+1C_m=\sum C_i\]
	so the result from lemma~\ref{lem:orth-decomp} applies: we get a matrix $P$ such that 
	\[P^{-1}\rho(xI_n)P=\begin{pmatrix}
		x^mI_{d_0} & & & &\\
		& x^{m-1}I_{d_1} & & &\\
		& & \ddots & &\\
		& & & xI_{d_{m-1}} & \\
		& & & & I_{d_m}
	\end{pmatrix}\]
	Now let $\rho'(g)=P^{-1}\rho (g)P$ for all $g\in\GL_n(R)$. This is a representation of $\Gamma$ since it it differs from $\rho$ by 
	an automorphism of $\GL(V)$. Since matrix multiplication is an algebraic operation, $\rho'$ is still a polynomial representation of $\GL_n(R)$. 
	But notice that for all $g\in\GL_n(R)$
	\[\rho'(g)\rho'(xI_n)=\rho'(xg)=\rho'(xI_n)\rho'(g)\]
	Then lemma \ref{lem:orth-decomp} gives us that $\rho'(g)$ decomposes in the same way for all $g\in \GL_n(R)$, so 
	we know that $\rho'$ decomposes as a direct sum of representations 
	\[\rho'=\sum_{i=0}^m \rho'_i\]
	where for each $i$ and $\lambda\in k$,
	\[\rho'_i(\lambda g)=\rho_i'(\lambda I_{d_i})\rho_i'(g)=\lambda^i\rho'_i(g)\]
	so each $\rho_i'$ is a homogeneous degree $i$ polynomial representation of $\Gamma$.

	But of course the decomposition of a representation is independent of choice of basis,
	so we get a decomposition of $\rho$ into homogenous pieces, as desired.
\end{prf}

This is wonderful if one is just interested in the representation theory of $\GL_n(A)$ for a specific $A$, but 
here we are interested in group \textit{scheme} representations. The result above effectively tells us how the scheme splits 
up on points, but how about the global structure?

What we need is that this splitting is functorial. That is, if $\rho:\Gamma\to\GL(V)$ is a polynomial representation, there is a subscheme $\rho_r$
that, for all $A$, $(\rho_r)_A$ is the homogeneous degree $r$ part of $\rho_A$. Let $\rho_r$ be such a map and let $\varphi:A\to B$ be an 
algebra morphism. Then we want that this induces a map
\[\hat\varphi:(\rho_r)_A\to (\rho_r)_B.\]
Fix a basis $v_1,\dots,v_n$ for $V$ and let $g\in \Gamma(A)$. Then since $(\rho_r)_A$ is homogeneous degree $r$, for all $v_i$,
\[g\cdot_A v_i=\sum_j f_{ij}(g)v_i\]
where the $f_{ij}$ is a homogeneous degree $r$ polynomial in the $c_{ij}$. Now if we write $(\rho_r)_A(g)=M_{g,A}=(m_{ij})_{i,j}$, this means that 
if $\lambda\in k$,
\[M_{\lambda g, A}=(\lambda^rm_{ij})_{i,j}=\lambda^rM_{g,A}\]

The map $\hat \varphi$ is such that 
\[\hat\varphi(M_{g,A})=(\varphi(m_{ij}))_{i,j}=M_{g,B}\]
and so
\[M_{\lambda g,B}=\hat\varphi(M_{\lambda g,A})=(\varphi(\lambda^r m_{ij}))_{i,j}=(\lambda^r \varphi(m_{ij}))_{i,j}=\lambda^rM_{g,B}\]
which tells us that the map $\rho_r$ is indeed a functor $\Gamma\to\GL(V)$, so is a subgroup scheme of $\rho$.
Thus the pointwise splitting shown above lifts to a splitting of the entire representation $\rho$.

\subsubsection{Monomials and multi-indices}\label{subsubsec:indices}
All of the discussion up to this point has revolved around polynomials in $n^2$ variables, which quickly gets unwieldy unless one 
uses some better notation. To that end, 
\begin{defn}
	An $(n,r)$-\textbf{multi-index} $i$ is an $r$-tuple $(i_1,\dots,i_r)$ where each $i_j\in\underline n\eqdef\{1,\dots,n\}$.
	The collection of all $(n,r)$-multi-indices is denoted $I(n,r)$.
\end{defn}
\begin{rmk}
	One can also think of an element $i\in I(n,r)$ as a (set) map 
	\[i:\underline r\to\underline n.\]
\end{rmk}
The idea here is to associate to each monomial in a polynomial ring in many variables a tuple indicating its multidegree. That is we think of 
\[(i_1,\dots,i_r)\quad\leftrightsquigarrow\quad x_{i_1}\cdots x_{i_r}\]
as corresponding to the same object. Which is wonderful except for one small flaw: polynomials are commutative 
and multi-indices (as we have defined them) aren't! For example, in $I(3,4)$,
\[(2,2,1,3)\quad\leftrightsquigarrow\quad x_1x_2^2x_3\quad\leftrightsquigarrow\quad (3,2,1,2).\]

To handle this disparity, we define an equivalence relation on $I(n,r)$ where we say that $i\sim j$ if they are in the 
same orbit under the natural $\frakS_r$ action. That is, if there exists $\sigma\in\frakS_r$ such that
\[(i_1,\dots,i_r)=(j_{\sigma(1)},\dots,j_{\sigma(r)})\]

In the context of polynomial representations of $\Gamma$, we want to consider polynomials in the coordinate functions $c_{ij}$,
so as a matter of notation if $i,j\in I(n,r)$, let $c_{i,j}$ denote the monomial 
\[c_{i,j}=c_{i_1j_1}\cdots c_{i_rj_r}.\]
Again, we want to take into account that we can permute the order on the right hand side, but now we need that $i_k$ and $j_k$ 
remain linked to the same function. To deal with this, we define an equivalence relation $\sim$ on $I(n,r)\times I(n,r)$ such that 
\[(a,c)\sim (b,d)\]
if there exists a $\sigma\in\frakS_r$ such that 
\[(a_1,\dots,a_r)=(b_{\sigma(1)},\dots,b_{\sigma(r)})\quad\text{and}\quad(c_1,\dots,c_r)=(d_{\sigma(1)},\dots,d_{\sigma(r)}).\]
The upshot of this work is that it gives us a bijection between (total) degree $r$ monomials in the $c_{ij}$ and the set
\[I(n,r)\times I(n,r)/\sim\]

\subsubsection{\texorpdfstring{$A_k(n,r)$}{Ak(n,r)}}
Notice that if $V\in \Pol(n,r)$, each of its structure maps are homogeneous degree $r$ polynomials. As the first object of study, consider 
\begin{defn}
	Let $A_k(n,r)=A(n.r)$ denote the collection of all homogeneous degree $r$ polynomials in the 
	coordinate functions $c_{ij}:\Gamma\to k$.
\end{defn}
It is not too hard to see that 
\begin{prop}
	$A_k(n,r)$ is spanned by the elements 
	\[\{c_{i,j}|(i,j)\in I(n,r)\times I(n,r)\}\]
\end{prop}
however it takes a short argument to see 
\begin{lem}
	The dimension of $A_k(n,r)$ over $k$ is $\binom{n^2+r-1}{n^2-1}=\binom{n^2+r-1}{r}$.
\end{lem}
\begin{prf}
	The following is a ``stars and bars'' argument that is pervasive in combinatorics. See for example \cite{stanley} if unfamiliar with these techniques.
	
	Fix an ordering of the $c_{ij}$ (say the dictionary order)
	and relabel them $\{\gamma_1,\dots,\gamma_{m}\}$ (here $m=n^2$) according to this order. Then the degree $r$ monomials are in bijection with $m$-tuples $(a_1,\dots,a_{m})\in\bbN^m$ such that $\sum_i a_i=r$ via the map which sends 
	\[(a_1,\dots,a_{m})\mapsto \gamma_1^{a_1}\cdots\gamma_{m}^{a_{m}}.\]

	But choosing such an element is the same as inserting $m-1$ bars into a line of $r$ stars (that is an ordered partition of $r$ into $m$ parts, 
	where parts are allowed to be zero). But this is equivalent to choosing $m-1$ bars in a field of $m+r-1$ symbols. This is just 
	\[\binom{m+r-1}{m-1}\]
	and a well-known identity for binomial coefficients gets us the final equality.
\end{prf}
\begin{ex}
	In case the reader is unfamiliar with this kind of reasoning, consider the case when $n=5$ and $r=4$. Then the composition $(1,0,0,2,1)$ corresponding to 
	$\gamma_1\gamma_4^2\gamma_5$ corresponds to the stars-and-bars diagram 
	\begin{center}
		$\ast|||\ast\ast|\ast$
	\end{center}
	where there are $m+r-1=8$ symbols, $r=4$ of which are stars.
\end{ex}

\subsubsection{Hopf algebras and group schemes}
$A(n,r)$ lies within $k^\Gamma=k[\Gamma]$, which has the structure of a Hopf algebra induced from the group 
structure on $\Gamma$. More precisely, the functor $\Gamma:\Alg_k\to \Grp$ that assigns to every $k$-algebra $A$ the group $\GL_n(A)$ is representable. In other words, 
\[\GL_n(-)\simeq \Hom_{\Alg_k}(R,-)\]
where $R=k[\Gamma]$.

The anti-equivalence of the categories of affine group schemes over $k$ and finite dimensional commutative $k$-Hopf algebras (of which this is a particular instance) follows from Yoneda lemma 
(c.f. \cite[chp. 1]{waterhouse}). The resulting Hopf algebra will be (as an algebra) $R$, and along with a coalgebra structure induced the group structure on $\Gamma$: 
we have maps $\mu,\epsilon$, the multiplication and unit maps on $\Gamma$ satisfying the diagrams 
\begin{center}
	\begin{tikzcd}
		\Gamma\times\Gamma\times\Gamma\ar[r,"\mu\times\id"]\ar[d,"\id\times\mu"] & \Gamma\times\Gamma\ar[d,"\mu"]\\
		\Gamma\times\Gamma\ar[r,"\mu"] & \Gamma
	\end{tikzcd}
	\quad \begin{tikzcd}
		\ast\times G\ar[r,"\epsilon\times\id"] &G\times G\ar[d,"\mu"]& G\times \ast\ar[l,"\id\times\epsilon",swap]\\
		& G\ar[ur,leftrightarrow,"\sim"]\ar[swap,ul,leftrightarrow,"\sim"] &
	\end{tikzcd}
\end{center}
(where $\ast$ is the trivial group and initial object in the category of group schemes) giving us associativity and identity. Yoneda tells us that the maps between schemes
\[\mu:\Gamma\times\Gamma\to \Gamma\quad\text{and}\quad \epsilon:\ast\to\Gamma\]
give rise to maps in $\Alg_k$:
\[\Delta\eqdef\mu^\ast:R\to R\otimes_k R\quad\text{and}\quad \varepsilon\eqdef\epsilon^\ast: R\to k\]
satisfying diagrams 
\begin{center}
	\begin{tikzcd}
		R\otimes R\otimes R & R\otimes R\ar[l,"\Delta\otimes\id",swap]\\
		R\otimes R\ar[u,"\id\otimes \Delta"] & R\ar[u,"\Delta"]\ar[l,"\Delta"]
	\end{tikzcd}
	\quad\begin{tikzcd}
		k\otimes R\ar[dr,"\sim",leftrightarrow,swap] & R\otimes R\ar[l,"\varepsilon\otimes \id",swap]\ar[r,"\id\otimes\varepsilon"] & R\otimes k\ar[dl,"\sim",leftrightarrow]\\
		& R\ar[u,"\Delta"] &
	\end{tikzcd}
\end{center}
\begin{prop}
	The maps $\Delta$ and $\varepsilon$ which, in coordinates, for $1\le i,j\le n$, are
	\[\Delta(c_{ij})=\sum_k c_{ik}\otimes c_{kj}\quad\text{and}\quad \varepsilon(c_{ij})=\delta_{ij}\]
	give a coalgebra structure on $R$
\end{prop}

That these maps satisfy the diagrams above is a straightforward computation. That, furthermore, these maps make $R$ into a bialgebra amounts 
to checking that $\Delta$ and $\varepsilon$ are algebra morphisms.
But what is not immediately obvious is why \textit{these particular maps} are the ones we use on $R$. To 
see this, one must dig into the Yoneda correspondence a bit to see what happens to the multiplication and unit morphisms.

In service of this, let's translate matrix multiplication into a statement about representable functors. We want to define $m$ as a map 
\[m:\Hom(R,-)\times\Hom(R,-)\to \Hom(R,-)\]
and to see what $m$ should do in this context, we evaluate at a $k$-algebra 
\[m_A:\Hom(R,A)\times\Hom(R,A)\to \Hom(R,A)\]
where we interpret each map $f:R\to A$ as a matrix with entries in $A$ by saying $f$ corresponds to a matrix $A_f$ such that 
\[(A_f)_{ij}=f(c_{ij}).\]

Then if $(f,g)\in \Hom(R,A)\times\Hom(R,A)$, we want that the algebra structure is the usual matrix multiplication, so
\[m_A(f,g)=A_fA_g\]
and by computing the $(i,j)^{th}$ entry everywhere, we get 
\[m_A(f,g)(c_{ij})=(A_fA_g)_{ij}=\sum_{k=1}^n(A_f)_{ik}(A_g)_{kj}=\sum_k f(c_{ik})g(c_{kj}).\]

This gives us the values of our component maps everywhere, so this defines the natural transformation $m$. Then (the proof of)
Yoneda tells us that we can compute the corresponding algebra morphism as 
\[\mu(c_{ij})=m_{R\otimes R}(\iota_l\otimes\iota_r)(c_{ij})=\sum_k \iota_l(c_{ik})\iota_r(c_{kj})=\sum_k c_{ij}\otimes c_{kj}.\]
Above we call $\iota_l$ (resp. $\iota_r$) to be the map $R\to R\otimes R$ which embeds $R$ into the left (resp. right) tensor factor. Notice 
that $\iota_l\otimes\iota_r=\id_{R\otimes R}$.

Using the same identification between maps and matrices over $A$, let $\ast:k\to A$ be the unique map sending $1_k\mapsto 1_A$. Then we want
\[u_A(\ast)=f:R\to A\]
corresponding to the identity $(n\times n)$ matrix over $A$. So 
\[u_A(\ast)(c_{ij})=f(c_{ij})=(I_n)_{ij}=\delta_{ij}\cdot 1_A.\]
Again applying Yoneda, we have 
\[\varepsilon(c_{ij})=u_k(\id_k)(c_{ij})=\delta_{ij}1_k\]
and we have our counit map.

In fact, as mentioned before, $R$ becomes a bialgebra (a Hopf algebra even, although we won't need the antipode here). This means that 
$\Delta$ and $\varepsilon$ are algebra morphisms for the natural algebra structure given by multiplication $m$ on $R$. In diagrams:
\begin{center}
	\begin{tikzcd}
		R^{\otimes 4}\ar[r,"\id\otimes\tau\otimes 1"] & R^{\otimes4}\ar[r,"m\otimes m"]  & R\otimes R\\
		R\otimes R\ar[u,"\Delta\otimes \Delta"]\ar[rr,"m"] & & R\ar[u,"\Delta"]
	\end{tikzcd}
	\quad\begin{tikzcd}
		R\otimes R\ar[r,"m"]\ar[d,"\varepsilon\otimes\varepsilon"] & R\ar[d,"\varepsilon"]\\
		k\otimes k\ar[r,"m"] & k
	\end{tikzcd}
\end{center}
where $\tau:R\otimes R\to R\otimes R$ is the twist map $a\otimes b\mapsto b\otimes a$. 
Chasing an element through the diagram on the left, we get
\[\tilde m\circ (\Delta\otimes \Delta)(c_{ij}\otimes c_{ab})=\sum_{1\le k,l\le n}c_{ik}c_{al}\otimes c_{kj}c_{lb}=\Delta(c_{ij}{c_{ab}})\]
or using our multi-index notation,
\[\Delta(c_{(i,a),(j,b)})=\sum_{(k,l)\in I(n,2)}c_{(i,a),(k,l)}\otimes c_{(k,l),(j,b)}.\]

Written more simply, the fact that $\Delta$ is an algebra morphism can be written 
\[\Delta(a\cdot b)=\Delta(a)\ast\Delta(b)\]
under suitable definitions of $\cdot$ and $\ast$. In a way that can be made precise, this means in particular that 
\[\Delta(a\cdot b\cdot c)=\Delta(a)\ast\Delta(b\cdot c)=\Delta(a)\ast\Delta(b)\ast\Delta(c)\]
and so on (since multiplication everywhere is associative) and therefore we can define this for arbitrary monomials and extend $k$-linearly: 
\begin{prop}
	If $i,j\in A(n,r)$, then 
	\[\Delta(c_{i,j})=\sum_{k\in I(n,r)}c_{i,k}\otimes c_{k,j}\quad\text{and}\quad \varepsilon(c_{i.j})=\delta_{i,j}\]
\end{prop}
One can easily see that degree is preserved by $\Delta$, meaning that 
\begin{prop}
	$\Delta$ and $\varepsilon$ descend to a coalgebra structure on $A(n,r)$. That is, $A(n,r)$ is a ($k$-)coalgebra.
\end{prop}

\subsubsection{The structure maps of \texorpdfstring{$\rho$}{rho}}
This context empowers us to better understand what is meant by the structure maps of a representation. At the moment, we 
define a homogeneous polynomial representation by how it looks on points and simply use the fact it coalesces to a functor. 
A natural question to ask is how the structure morphisms relate to the entries of the matrices $\rho_A(g)=M_{g,A}$.

To understand the answer to this question, we need to uncover how the entries of a matrix come about. We have been thinking of an element of 
$\Gamma(A)$ as a matrix, but what it actually is is a morphism 
\[k[x_{ij}]_{\det}\to A\]
and then thinking of $\Aut(V\otimes A)\cong\GL_m(A)$ in the same way, we interpret a representation 
as a map 
\[\rho:\Hom(k[x_{ij}]_{\det},-)\to \Hom(k[y_{kl}]_{\det},-)\]
where $i$ and $j$ run from 1 to $n$ and $k$ and $l$ run from 1 to $m$. 

Then Yoneda lemma tells us that $\rho$ corresponds to an algebra map
\[\rho^\ast:k[y_{lk}]_{\det}\to k[x_{ij}]_{\det}\]
This map is the one such that if $f\in\Hom(k[x_{ij}]_{\det},A)$,
\[\rho_A(f)=f\circ\rho^\ast:k[y_{lk}]_{\det}\to k[x_{ij}]_{\det}\to A.\]

What are the structure maps for this representation? We can compute for any $g\in \GL_n(A)$ (which we, though a mild
abuse of notation, think of as a map $g:k[x_{ij}]_{\det}\to A$ where $g(x_{ij})=g_{ij}$)
\[\rho_A(g)\cdot v_i=\rho_A(g(x_{ij}))_{i,j}v_i=((g\circ\rho^\ast)(y_{ij}))_{i,j}v_i=\sum_j g(\rho^\ast(y_{ij}))v_j=\sum_j\rho^\ast(y_{ij})(g)v_j\]
from which we can see
\begin{lem}
	Let $\rho:\Gamma\to \GL(V)$ be a polynomial representation. Then for any $A\in\Alg_k$, the structure maps $f_{ij}$ 
	of the group representation $\rho_A:\GL_n(A)\to \GL_m(A)$ are precisely the $\rho^\ast(y_{ij})$.
\end{lem}

This enables us to re-define polynomial representations 
in the following way (compare with definition \ref{def:poly-rep}):
\begin{defn}\label{def:poly-rep-new}
	A finite dimensional \textbf{polynomial representation of $\Gamma$ of degree $r$} is a finite dimensional vector space $V$ 
	over $k$ along with a scheme map $\rho:\Gamma\to \GL(V)$ such that the associated algebra map $\rho^\ast:k[\GL(V)]\to k[\Gamma]\cong k[x_{ij}]_{\det}$ is homogeneous 
	degree $r$. In other words, the image of $\rho^\ast$ is entirely contained within the degree $r$ graded piece of $k[x_{ij}]\subseteq k[\Gamma]$.
\end{defn}

\begin{rmk}
	In the following section, we freely identify $k[x_{ij}]_{\det}$ with the ring of functions on $\Gamma$, and $k[y_{lk}]_{\det}$ with 
	the ring of functions on $\GL(V)$ (where $m=\dim V$).
\end{rmk}

\subsubsection{Comodules}
In this section let $A=A(n,r)$, which we have just established is a coalgebra with $\Delta$ and $\varepsilon$ defined above.
\begin{defn}
	A (left) \textbf{$A$-comodule} is a vector space $V$ over $k$ along with a (left) \textbf{$A$-coaction} given by a ($k$-)morphism 
	\[\phi:V\to A\otimes_k V\]
	that is both \textbf{coassociative and counital} in the sense that the diagrams in Figure~\ref{fig:comodule} commute.
\end{defn}
\begin{figure}
	\centering
	\begin{tikzcd}
		V\ar[r,"\phi"]\ar[d,"\phi"] & A\otimes V\ar[d,"\id\otimes\phi"]\\
		A\otimes V\ar[r,"\Delta\otimes \id",swap] & A\otimes A\otimes V
	\end{tikzcd}\qquad 
	\begin{tikzcd}
		V\ar[r,"\phi"]\ar[rd,"\sim",swap] & A\otimes V\ar[d,"\varepsilon\otimes\id"]\\
		& k\otimes V
	\end{tikzcd}
	\caption{The coassociative and counital axioms}
	\label{fig:comodule}
\end{figure}

Given two $A$-comodules $V$ and $W$, a comodule morphism $\varphi:V\to W$ is one that preserves the coaction. That is 
\[(\id\otimes\varphi)\phi_V(v)=\phi_W(\varphi(v))\]
for all $v\in V$.

\begin{defn}
	Let $A$ be any coalgebra. Then $\lcomod A$ denote the category of (finite-dimensional, left) $A$-comodules along with comodule morphisms.
\end{defn}

Sometimes a more useful way to think of polynomial representations is as comodules. That idea is made more formal
in the following theorem:
\begin{lem}\label{lem:comod-map}
	Every homogeneous degree $r$ polynomial representation of $\GL_n$ gives rise to an $A(n,r)$-comodule
	in the following way: the underlying vector space is the same and the $A(n,r)$ coaction is given by 
	\[\phi(v_i)=\sum_j\rho^\ast(y_{ij})\otimes v_j\]
	where in the above we identify the algebra $k[y_{ij}]_{\det}$ with the ring of functions on $\GL(V)$. As a matter of notation, 
	we call this map 
	\[\Psi:\Pol(n,r)\to \lcomod{A(n,r)}.\]
\end{lem}
\begin{prf}
	By definition this gives us a map into $\lcomod{k[\Gamma]}$, but we can see that the image is entirely contained within 
	$\lcomod{A(n,r)}$ since, in light of definition \ref{def:poly-rep-new}, $\rho^\ast(y_{ij})$ is homogeneous degree $r$.
	Then it remains to show that the given map is legitimately a coaction. We can compute (identifing the map $\varepsilon:k[\Gamma]\to k$ as the matrix $I_n$ over $k$)
	\[(\varepsilon\otimes\id)\circ\phi(v_i)=\sum_j \varepsilon(\rho^\ast(y_{ij}))\otimes v_j=\sum_j(\varepsilon\circ\rho^\ast)(y_{ij})\otimes v_j=\sum_j\rho_k(I_n)_{ij}\otimes v_j=1_k\otimes v_i\]
	so $\phi$ satisfies the counit identity. For coassociativity, identify the morphism $k[\Gamma]\to k[\Gamma]\otimes k[\Gamma]$ with the matrix $D$ whose entries are $\Delta(x_{ij})$. Then
	\begin{align*}
		(\Delta\otimes\id)\circ\phi(v_i)&=\sum_j\Delta(\rho^\ast(y_{ij}))\otimes v_j\\
		&=\sum_j\rho(D)_{ij}\otimes v_j\\
		&=\sum_j \left(\sum_k \rho^\ast(y_{ik})\otimes\rho^\ast(y_{kj})\right)\otimes v_j\\
		&=\sum_k \rho^\ast(y_{ik})\otimes\left(\sum_j \rho^\ast(y_{kj})\otimes v_j\right)\\
		&=(\id\otimes\phi)\sum_k\rho^\ast(y_{ik})\otimes v_k\\
		&=(\id\otimes\phi)\circ\phi(v_i)
	\end{align*}
	which shows that $\phi$ gives a $A(n,r)$-comodule structure on the underlying vector space of a representation $\rho$ of $\Gamma.$
\end{prf}
\begin{rmk}
	Above we used the fact that, for all $g\in\GL_n(A)$, 
	\[\Delta(\rho^\ast(y_{ij}))(g)=\sum_k\rho^\ast(y_{ik})\otimes\rho^\ast(y_{kj})\]
	which is true since the map
	\[\Delta\circ\rho^\ast:k[\GL(V)]\to k[\Gamma]\otimes k[\Gamma]\]
	corresponds to the map 
	\[\rho\circ m:\Gamma\times\Gamma\to \GL(V)\]
	where, after evaluating at $A\in\Alg_k$,
	\[(\rho\circ m)(M,N)=\rho(MN)=\rho(M)\rho_A(N)=m\circ(\rho\times\rho)(M, N)\]
	which implies we have the equation
	\[\Delta\circ \rho^\ast=(\rho^\ast\otimes \rho^\ast)\circ \Delta\]
	and the equality follows.
\end{rmk}
The preceeding lemma is in service of reframing the problem in terms of the comodules of a nicely-behaved (e.g. finite dimensional!) coalgebra:
\begin{lem}
	The map $\Psi$ defined in lemma~\ref{lem:comod-map} is a functor.
\end{lem}
\begin{prf}
	Let $f:(V,\rho)\to (W,\eta)$ be a map of homogeneous degree $r$ polynomial representations of $\Gamma$. This is a linear map $f:V\to W$ 
	satisfying the usual property that for any $A\in\Alg_k$, $a\in V\otimes A$ and $g_A\in \Gamma(A)$,
	\[f(\rho(g_A)a)=\eta(g_A)f(a).\]
	%This corresponds to a diagram
	%\begin{center}
%		\begin{tikzcd}[column sep=large]
%			\Gamma\times V\ar[r,"{(g,v)\mapsto \rho(g)v}"]\ar[d,"\id\times f",swap] & V\ar[d,"f"]\\
%			\Gamma\times W\ar[r,"{(g,w)\mapsto \eta(g)w}",swap] & W
%		\end{tikzcd}
%	\end{center}
%	and upon evaluating at $k[\Gamma]$ the action gives us a coaction defined by (for all $v\in V$)
%	\[\phi_\rho(v)=\rho_{k[\Gamma]}(\id_{k[\Gamma]})(1_{k[\Gamma]}\otimes v)\in k[\Gamma]\otimes V\]

	We want to show that $f$ is a map of $A(n,r)$-comodules. Let $v_i\in V$ be a basis element as we have used before. Then if $f(v_i)=\sum_k a_{ik}w_{k}$ where $W=\langle w_k\rangle$,
	\[\phi_W\circ f(v_i)=\phi_W\left(\sum_ka_{ik}w_k\right)=\sum_ka_{ik}\phi_W(w_k)=\sum_ka_{ik}\left(\sum_j \eta^\ast(y_{kj})\otimes w_j\right)\]
	and on the other hand 
	\[(\id\otimes f)\circ\phi(v_i)=(\id\otimes f)\left(\sum_j\rho^\ast(y_{ij})\otimes v_j\right)=\sum_j\rho^\ast(y_{ij})\otimes f(v_i)=\sum_j\rho^\ast(y_{ij})\otimes\left(\sum _ka_{jk}w_k\right)\]
	and using the identification of 
	\[k[\Gamma]\otimes W\cong \Hom_k(\Gamma,W)\]
	we see the first line corresponds to the map 
	\[g\mapsto \sum_ka_{ik}\sum_j\eta(g)_{kj}w_j=\sum_ka_{ik}\eta(g)w_k=\eta(g)f(v_i)\]
	and the bottom line corresponds to 
	\[g\mapsto \sum_j\rho(g)_{ij}f(v_j)=f\left(\sum_j\rho(g)_{ij}v_j\right)=f(\rho(g)v_i)\]
	and these two values are equal by virtue of of $f$ being a $G$-module morphism.
\end{prf}
Finally we prove that these are the same category!
\begin{thm}
	The map
	\[\Psi:\Pol(n,r)\to \lcomod{A(n,r)}\]
	is an equivalence of categories.
\end{thm}
\begin{prf}
	To prove essential surjectivity, let $V$ be an $A(n,r)$ comodule with coaction $\phi:V\to A(n,r)\otimes V$.
	We define from this an object in $\Pol(n,r)$ via the action 
	\[g\cdot v=(e_g\overline\otimes \id)\circ\phi(v)\]
	where $e_g:k[\Gamma]\to k$ is evaluation at $g\in \Gamma$ and 
	\[e_g\overline\otimes\id (f\otimes v)=f(g)v.\]
	This defines the map 
	\[\rho:\Gamma\to \GL(V)\quad\text{via}\quad g\mapsto (e_g\overline\otimes \id)\circ\phi\]
	which is a representation of $\Gamma$. To check that this is homogeneous degree $r$, we just compute (where here we write $\phi(v)=\sum_i f_i\otimes v_i$)
	\[\rho(\lambda g)(v)=(e_{\lambda g}\overline\otimes\id)\circ\left(\sum f_i\otimes v_i\right)=\sum f_i(\lambda g)v_i=\lambda^r\sum f_i(g)v_i=\lambda^r\rho(g)(v)\]
	where we used that the $f_i$ are homogeneous degree $r$ maps.

	Our category is concrete (we have a faithful functor to $\Set$) and therefore the comodule map $\Psi(f)$
	is completely determined by its underlying map on sets. But $\Psi(f)(v)=f(v)$ after passing to sets for all $v\in V$,
	so $\Psi$ is automatically fully faithful. 

	Thus $\Psi$ is an equivalence of categories, as desired.
\end{prf}
\begin{rmk}
	Actually, the above proof can be modified slightly to show that $\Psi$ has a functorial inverse--that is, $\Psi$ is an \textit{isomorphism of categories}.
	Since we are only interested in representations up to isomorphism, however, equivalence is all we need.
\end{rmk}

\subsubsection{The Schur algebra}
Finally we get to the actual object of study:
\begin{defn}\label{def:schur-alg}
	A \textbf{Schur algebra} is an element of the two-parameter family $\{S(n,r)\}=\{S_k(n,r)\}$ where $n$ and $r$ are any positive integers.
	As a set, $S(n,r)$ is the linear dual of $A(n,r)$:
	\[S(n,r)=A(n,r)^\ast=\Hom_k(A(n,r),k)\] 

	Let $\xi_{i,j}$ denote the element dual to $c_{i.j}\in A(n,r)$. In other words:
	\[\xi_{(a,b)}(c_{i,j})=\begin{cases}
		1, & (a,b)\sim(i,j)\\
		0, & \text{otherwise}
	\end{cases}\]
\end{defn}

\begin{lem}
	The coalgebra structure $(\Delta,\varepsilon)$ on $A(n,r)$ defines an algebra structure on $S(n,r)$.
\end{lem}
\begin{prf}
	Since $k$ is an initial object in $\Alg_k$, there is a unique map $u:k\hookrightarrow S(n,r)$ sending $1$ to the unit function $\1$, which is given by 
	\[\1(c_{i,j})=c_{i,j}(I_n)=\delta_{i,j}\]
	Define multiplication $(\cdot)$ in $S(n,r)$ as follows: if $f,g\in S(n,r)$ then for any $x\in A(n,r)$ define 
	\[(f\cdot g)(x)=m_k\circ (f\otimes g)\circ \Delta(x)=\sum f(x_{(1)})g(x_{(2)})\]
	where $m_k:k\otimes k\to k$ denotes multiplication in $k$ and $\Delta(x)=\sum x_{(1)}\otimes x_{(2)}$ in Sweedler notation.

	Then we must just confirm that these maps satisfy the properties of a $k$-algebra. $(\cdot)$ is $k$-bilinear because (for instance)
	\begin{align*}
		((af+bg)\cdot h)(x)&=\sum (af+bg)(x_{(1)})\otimes h(x_{(2)})\\
		&=\sum a(f(x_{(1)})\otimes h(x_{(2)}))+b(g(x_{(1)})\otimes h(x_{(2)}))\\
		&= a\sum f(x_{(1)})\otimes h(x_{(2)})+ b\sum g(x_{(1)})\otimes h(x_{(2)})\\
		&=(a(f\cdot h)+b(g\cdot h))(x).
	\end{align*}
	
	By $k$-linearity, it suffices to show that the unit $\1$ acts as it should on the spanning set $\xi_{i,j}$ for a basis element $c_{a,b}$:
	\[(\1\cdot \xi_{i,j})(c_{a,b})=\sum_{k=1}^n \1(c_{a,k})\cdot\xi_{i,j}=\1(c_{a,a})\cdot\xi_{i,j}(c_{a,b})=\xi_{i,j}(c_{a,b})\]
	and a similar identity holds on the right.

	Then it remains to show that this multiplication is associative. Again by linearity it suffices to check that this works on the spanning set $\{c_{i,j}\}$:
	\begin{align*}
		((\alpha\cdot \beta)\cdot\gamma)(c_{i,j})&=\sum_{k\in I(n,r)}(\alpha\cdot\beta)(c_{i,k})\gamma(c_{k,j})\\
		&=\sum_k\left(\sum_{l\in I(n,r)}\alpha(c_{i,l})\beta(c_{l,k})\right)\gamma(c_{k,j})\\
		&=\sum_l\alpha(c_{i,l})\left(\sum_k \beta(c_{l,k})\gamma(c_{k,j})\right)\\
		&=\sum_l\alpha(c_{i,l})(\beta\cdot\gamma)(c_{l,j})\\
		&=(\alpha\cdot(\beta\cdot\gamma))(c_{i,j}).
	\end{align*}

	Thus since we have $k$-linear maps $\1$ and $m=(\cdot)$ satisfying the usual identity and associativity diagrams, $S(n,k)$ is a $k$-algebra 
	with $\1$ and $m$ as its unit and multiplication.
\end{prf}


There is a standard result that says 
\begin{prop}
	The finite dimensional left comodules of a coalgebra $\Lambda$ are the same as finite dimensional right modules 
	over $\Lambda^\vee=\Hom(\Lambda, k)$.
\end{prop}
\begin{prf}[sketch]
	The key idea here is as follows: an right comodule over $\Lambda^\vee$ is, equivalently, a $k$-linear map 
	\[V\otimes \Lambda^\vee\to V\]
	satisfying the usual associativity and identity axioms.	But notice that 
	\[\Hom(V\otimes \Lambda^\vee,V)\cong\Hom(V,\Hom(\Lambda^\vee,V))\subseteq\Hom(V,\Lambda\otimes V)\]
	which, in turn, correspond to $\Lambda$ comodules. It remains to show that the associativity and unit axioms restrict 
	the collection on the left in the right way to give us a map satisfying the coassociativity and counit axioms on the right.
\end{prf}

Since $S(n,r)$ is commutative ($A(n,r)$ is cocommutative), this tells us that there is an equivalence between $\lcomod{A(n,r)}$ and $\rmod{S(n,r)}\cong\lmod{S(n,r)}$, and so 
\begin{cor}
	The categories $\Pol(n,r)$ and $\lmod{S(n,r)}$ are equivalent.
\end{cor}
\begin{rmk}
	Using this equivalence, we identify $\lmod{S(n,r)}$ with $\Pol(n,r)$ whenever it suits us.
\end{rmk}

The following result has been proven in many different contexts, but one source is a paper of Doty and Nakano which completely 
categorized the semisimple Schur algebras.
\begin{cor}[{\cite[Thm. 2]{doty-nakano}}]\label{cor:semisimple}
	If $\ch k=p$, the algebra $S_k(n,r)$ is semisimple if and only if one of the following hold:
	\begin{itemize}
		\item $p=0$
		\item $p>r$
		\item $p=n=2$ and $r=3$
	\end{itemize}
\end{cor}

\subsubsection{Weights and characters}
The discussion in section~\ref{subsubsec:indices} highlights an important idea: while we care about the \textit{quantities} in which each $c_{ij}$ occurs 
in a monomial, we are not particularly interested in the \textit{order}. Sometimes it is easier, then, to simply regard these as weak compositions:
\begin{defn}
	Let $n$ and $r$ be integers as usual. Then denote by $[a_1,\dots,a_n]$ the \textbf{weight} corresponding to 
	$(i_1,\dots,i_r)\in I(n,r)$ where for each $i$,
	\[a_i=\#\{k\in\underline r| i_k=i\}\]
	Denote by $\Lambda(n,r)$ the collection of all weights. 
\end{defn}
\begin{rmk}
	Another way to realize $\Lambda(n,r)$ is in the presentation 
	\[\Lambda(n,r)=\left\{[a_1,\dots,a_n]\left|\sum_i a_i=r\right.\right\},\]
	or as the set of compositions of $r$ into $n$ parts (allowing zeros).
	
	Yet another is to think of $\Lambda(n,r)$ as the set of $\frakS_r$ orbits in $I(n,r)$ (where now two objects 
	are distinguished only if their ``contents'' vary).
\end{rmk}
Recall (c.f. \ref{def:schur-alg}) that we had that $\xi_{i,j}(c_{a,b})=1$ if and only if $(i,j)\sim(a,b)$. Because of this, it makes sense (if $\alpha$ is the 
weight of $i$) to write 
\[\xi_{\alpha}\eqdef \xi_{\alpha,\alpha}\eqdef \xi_{i,i}\]
since the action is the same irrespective of the choice of representative $i$ of $\alpha.$

Notice that the weights admit a $\frakS_n$ action 
\[\sigma\cdot [a_1,\dots,a_n]=[a_{\sigma(1)},\dots,a_{\sigma(n)}]\]
then 
\begin{defn}
	$\Lambda_+(n,r)$ is the orbit space of $\Lambda(n,r)$ under the above $\frakS(n)$ action.
\end{defn}
\begin{rmk}
	The above are called the \textbf{dominant weights} in $\Pol(n,r)$. Since each orbit $\alpha$ contains an element $[a_1,\dots,a_n]\in\alpha$ such that 
	\[a_1\ge a_2\ge\cdots\ge a_n\]
	we will often identify weights with their weakly-decreasing representative.

	Sometimes we will refer to the dominant weight representing the orbit of $i\in I(n,r)$ as the \textbf{shape of $i$.}
\end{rmk}

The theory of weights in representations of $\Gamma$ closely mirrors similar decompositions in other 
Artinian algebras: first we identify a family of (mutually orthogonal) idempotents:
\begin{lem}
	For $\alpha\in\Lambda(n,r)$ and $i,j\in I(n,r)$,
	\[\xi_\alpha\xi_{i,j}=\begin{cases}
		\xi_{i,j}, & i\in\alpha\\
		0, &\text{otherwise}
	\end{cases}\quad\text{and}\quad\xi_{i,j}\xi_\alpha=\begin{cases}
		\xi_{i,j}, & j\in\alpha\\
		0, &\text{otherwise}
	\end{cases}\]
\end{lem}
\begin{prf}
	We can compute the image of these on the $c_{a,b}\in A(n,r)$:
	\begin{align*}
		\xi_\alpha\cdot \xi_{i,j}(c_{a,b})&=\sum_k \xi_\alpha(c_{a,k})\xi_{i,j}(c_{k,b})\\
		&= \xi_\alpha(c_{a,a})\xi_{i,j}(c_{a,b})
	\end{align*}
	where above we used that $\xi_{\alpha}(c_{i,j})=0$ unless $i=j$. But 
	\[\xi_\alpha(c_{a,a})=\begin{cases}
		1, & a\in\alpha\\ 0, & \text{otherwise}
	\end{cases}\]
	so 
	\[\xi_\alpha\cdot \xi_{i,j}(c_{a,b})=\begin{cases}
		\xi_{i,j}(c_{a,b}),& a\in\alpha\\ 0,& \text{otherwise}
	\end{cases}\]
	but in the case where $a\in\alpha$ and $\xi_{i,j}(c_{a,b})\ne 0$, this implies that $i\sim a$, so $i\in \alpha$. So finally,
	\[\xi_\alpha\cdot \xi_{i,j}(c_{a,b})=\begin{cases}
		\xi_{i,j}(c_{a,b}),& i\in\alpha\\ 0,& \text{otherwise}
	\end{cases}\]
	and since this holds for any $c_{a,b}$, the left-hand side is proven. A symmetric argument goes through for the right-hand side.
\end{prf}

For the next step, we decompose the identity into a sum of these idempotents:
\begin{lem}\label{lem:decomp-one}
	We have the decomposition 
	\[\1 = \sum_{\alpha\in\lambda(n,r)} \xi_\alpha.\]
\end{lem}
\begin{prf}
	On the one hand, for any $c_{a,b}\in A(n,r)$, $\1(c_{a,b})=\delta_{a,b}$. On the other hand, for any $\alpha$,
	\[\xi_\alpha(c_{a,b})=0\]
	when $a\ne b$ \textit{or when $a\notin \alpha$}. 
	
	Therefore when $a=b$, there is precisely one $\alpha$ (the orbit of $a=b$)
	such that $\xi_\alpha(c_{a,b})=1$, so putting this all together,
	\[\sum_{\alpha\in\Lambda(n,r)}\xi_\alpha(c_{a,b})=\delta_{a,b}\]
	whence these two functions are equal.
\end{prf}
\begin{rmk}\label{rmk-weight-spaces}
Using lemma~\ref{lem:decomp-one}, we can then decompose any $V\in \Pol(n,r)$ into weight spaces:
\[V=\1\cdot V=\sum_{\alpha\in\Lambda(n,r)}\xi_\alpha V\]
which we will denote 
\[\xi_\alpha V=V^\alpha.\]
\end{rmk}

\begin{defn}\label{defn:character}
	The \textbf{formal character} of a representation $V\in \Pol(n,r)$ is a polynomial 
	\[\Phi_V(X_1,\dots,X_n)=\sum_{\alpha\in\Lambda(n,r)}(\dim V^\alpha)X_1^{\alpha_1}\cdots X_n^{\alpha_n}=\sum_{\alpha\in\Lambda_+(n,r)}(\dim V^\alpha)m_\alpha(X_1,\dots,X_n)\]
	where $m_\alpha$ is the \textit{monomial symmetric polynomial}
	\[m_\alpha(X_1,\dots,X_n)=\sum_{\sigma\in\frakS_n}X_{\sigma(1)}^{\alpha_1}\cdots X_{\sigma(n)}^{\alpha_n}.\]
\end{defn}

\subsubsection{Irreducible representations}
The irreducible representations in $\Pol(n,r)$ are given by a couple of results by some of the big names in representation theory: the original proof for $k=\bbC$ was proven in \cite[p.37]{schur-thesis} and then 
generalized in a later paper by Weyl \cite{weyl} and in work by Chevalley\footnote{Green \cite{green} mentions a paper by Serre: \textit{Groupes de Grothendieck des Sch\'emas en Groupes R\'eductifs D\'eploy\'es} \cite{serre-chevalley}, which 
makes mention to Chevalley's contributions in proving the existence of modules with prescribed characters. This author was unable to find Chevalley's work.}:
\begin{thm}\label{thm:irreps}
	Fix the usual lexicographical ordering on monomials in $k[X_1,\dots,X_n]$. Let $n$ and $r$ be given integers with $n\ge 1$ and $r\ge 0$ Let $k$ be an infinite field. Then 
	\begin{enumerate}
		\item For each $\lambda\in\Lambda_+(n,r)$, there exists an (absolutely) irreducible module $F_{\lambda,k}$
		in $M_k(n,r)$ whose character $\Phi_{\lambda,k}$ has leading term $X_1^{\lambda_1}\cdots X_n^{\lambda_n}$.
		\item Every irreducible $V\in M_k(n,r)$ is isomorphic to $F_{\lambda,k}$ for exactly one $\lambda\in \Lambda_+(n,r)$.
	\end{enumerate}
\end{thm}
So then the problem of classifying the simple modules (the ``basic building blocks'' in the semisimple case) is completely solved for infinite fields.
It remains to demonstrate a way to construct $F_{\lambda,k}$.
\begin{defn}
	Fix some $\lambda\in \Lambda_+(n,r)$. Notice that this corresponds to a Young diagram with $r$ boxes. Fix any labeling $1,\dots,r$ of the boxes in the 
	Young diagram corresponding to $\lambda$. Let $T$ denote the diagram for $\lambda$ along with this labeling.
	
	Let $i:\underline r\to\underline n$ be any map. Then denote by $T_i$ the \textbf{$\lambda$-tableau}, which is $T$ with the $k^{th}$ entry consisting of $i(k)\in\underline r$.
\end{defn}
\begin{rmk}
	This notation varies slightly (but not in spirit) from the notation in Green's book. He denotes the Young diagram by $[\lambda]$ and lets $T^\lambda$ be 
	the labelling of the boxes in $[\lambda]$--a bijection $[\lambda]\to\underline r$.
\end{rmk}
\begin{ex}
	Let $\lambda=(3,1,1)\in\Lambda_+(3,5)$. Thus $T$ is of shape 
	\[\ydiagram{3,1,1}\]
	Then if we fix the left-to-right/top-to-bottom ordering of the boxes in $T$ and let $i:\{1,2,3,4,5\}\to\{1,2,3\}$
	be given by $(2,1,3,3,2)$, we get the $\lambda$-tableau 
	\[T_i=\ytableaushort{2 1 3, 3, 2}\]
\end{ex}

The core tool in constructing (a basis for) the irreducible modules is in the following definiton:
\begin{defn}
	Let $\lambda\in\Lambda_+(n,r)$ be some shape with a fixed labeling and let $i,j:\underline r\to\underline n$. Then the \textbf{bideterminant of $T_i$ and $T_j$}
	is 
	\[(T_i:T_j)=\sum_{\sigma\in C(T)}\operatorname{sgn}(\sigma)c_{i,j\sigma}\in A_k(n,r)\]
	where $C(T)$ is the column stabilizer of $T$.
\end{defn}
This definition can be a bit difficult to unpack, so we give some examples:
\begin{ex}
	\begin{enumerate}
		\item $\lambda=(2,1,0)\in \Lambda_+(3,3)$\[\ytableausetup{nosmalltableaux}\left(\ytableaushort{1 2,3}:\ytableaushort{3 1,2}\right)=\left|\begin{array}{cc}
			c_{13} & c_{12}\\ c_{33} & c_{32}
		\end{array}\right|c_{21}=(c_{13}c_{32}-c_{12}c_{33})c_{2,1}=c_{(1,2,3),(3,1,2)}-c_{(1,2,3),(2,1,3)}\]
		\item $\lambda=(n,0,\dots,0)\in\Lambda_+(m,n)$\[\left(\begin{ytableau} a_1& a_2& a_3&\none[\dots] &a_n\end{ytableau}:
		\begin{ytableau} b_1& b_2& b_3&\none[\dots]&b_n\end{ytableau}\right)=c_{a_1b_1}\cdots c_{a_nb_n}\]
		\item $\lambda=(1,\dots,1,0,\dots)\in\Lambda_+(m,n)$ where $n\ge m$ \[\left(\begin{ytableau}a_1\\ a_2\\\none[\vdots]\\a_n\end{ytableau}:\begin{ytableau}b_1\\ b_2\\\none[\vdots]\\b_n\end{ytableau}\right)=
			\left|\begin{array}{ccc}c_{a_1b_1} & \cdots & c_{a_1b_n}\\
			\vdots & \ddots & \vdots\\
			c_{a_nb_1} & \cdots & c_{a_nb_n}
			\end{array}\right|\]
	\end{enumerate}
\end{ex}

In the following, let $l:\underline r\to\underline n$ be $(1,\dots,1,2,\dots,2,3,\dots)$ such that for any shape $\lambda$ the 
$\lambda$-tableau $T_l$ is 
\[\begin{ytableau}
	1 & 1 &\none[\dots] &\none[\dots] &1\\
	2 & 2 &\none[\dots] &2\\
	\none[\vdots]\\
	k
\end{ytableau}\ytableausetup{smalltableaux}\]
with $i$ in every box on the $i^{th}$ row from the top.

\begin{defn}
	Define, for every shape $\lambda\in\Lambda_+(n,r)$, the module
	\[D_{\lambda,k}=\langle(T_l:T_i)\rangle_{i\in I(n,r)}\]
	where $l$ is the filling defined above.
\end{defn}
According to \cite{green}, these modules were originally called ``Weyl modules'', while he (and we)
reserve this name for the contravariant dual of these objects. To construct them, define the map 
\begin{equation}\label{eqn:pimap}
	\pi:E^{\otimes r}\to D_{\lambda, k},
\end{equation}
and we get objects originally defined in Carter and Lusztig's treatment of modular representations of $\GL_n$ \cite{carter-lusztig}
and tweaked by Green in \cite{green}:
\begin{defn}
	Given a shape $\lambda$, the \textbf{Weyl module of shape $\lambda$ over $k$} is 
	$V_{\lambda, k}\eqdef N^\perp$ where 
	\[N\eqdef\ker\pi\hookrightarrow E^{\otimes r}\to D_{\lambda,k}\]
	and the orthogonal complement of $N$ is taken with respect to the canonical contravariant form on $E^{\otimes r}$ that has the property $\langle e_i,e_j\rangle=\delta_{ij}$.
\end{defn}

In their original paper \cite[p.218]{carter-lusztig}, Carter and Lusztig showed that these modules are, in fact, generated as $S(n,r)$-modules by a single element:
\begin{thm}\label{thm:weyl-basis}
	Let $\lambda\in\Lambda_+(n,r)$ and $T$ the Young diagram corresponding to $\lambda$. Let $l$ be the labelling above. Then 
	the element 
	\[f_l=e_l\cdot\sum_{g\in C(T)\subset\frakS_n}\operatorname{sign}(\sigma)\sigma\]
	generates $V_{\lambda,k}$ as a $S(n,r)$-module.
\end{thm}
\begin{prf}[sketch.]
	We refer the reader to Green's \cite[p.46]{green} proof for the details, but the idea is as follows: he relies on an earlier result 
	that the modules $D_{\lambda,k}$ have a basis consisting of the bideterminants 
	\[(T_l:T_i)\]
	such that $T_i$ is in ``standard form'' (meaning that it forms a valid Young tableau). One can define a nondegenerate contravariant form 
	\[(\cdot,\cdot):V_{\lambda,l}\times D_{\lambda,k}\to k\]
	by pulling back any element in $D_{\lambda,k}$ to a representative in $E^{\otimes r}$ under the map $\pi:E^{\otimes r}\to D_{\lambda,k}$.

	Recall that $V_{\lambda,k}$ is defined as the orthogonal complement (under the canonical form $\langle\cdot,\cdot\rangle$ on $E^{\otimes r}$) of $\ker\pi.$
	This gives us that $(\cdot,\cdot)$ is indeed well-defined. From there, Green does some computation to show that one can bootstrap the 
	independence of the $(T_l:T_i)$ to prove that of the set
	\[\{\xi_{jl}f_l|j\in I(n,r), T_j\text{ standard}\}\]
	forms a $(k-)$basis for $V_{\lambda,k}$, and therefore $f_l$ generates the entire module under the $S(n,r)$ action.
\end{prf}
\begin{lem}\label{lem:unique-maximal-submod}
	The modules $V_{\lambda,k}$ have a unique maximal submodule $V_{\lambda,k}^{\text{max}}$
\end{lem}
\begin{prf}[{\cite[p.47]{green}}]
	Begin by noticing that the weight space $V_{\lambda,k}^\lambda$ is spanned by the single element $f_l$. This is because
	\[\xi_l\cdot\xi_{il}f_l=\delta_{il}f_l\]
	so the only nonzero basis vector from the proof of thm.~\ref{thm:weyl-basis} is $f_l$ itself. Since $f_l$ generates 
	all of $V_{\lambda,k}$ as an $S(n,r)$-module, however, any proper submodule $M$ of $V_{\lambda,k}$ must be contained in the 
	complement of $V_{\lambda,k}^\lambda.$ Thus the sum of all proper submodules is contained in the complement of this 
	weight space, and is therefore proper! This sum is our $V_{\lambda,k}^\text{max}$
\end{prf}

We are finally in good shape to compute the irreducible modules promised to us in thm.~\ref{thm:irreps}. We define 
\[F_{\lambda, k}=V_{\lambda,k}/V_{\lambda,k}^\text{max}\]
where $V_{\lambda,k}^\text{max}$ is the unique maximal submodule guaranteed to us by lemma~\ref{lem:unique-maximal-submod}. It remains to show 
that the $F_{\lambda,k}$ have the requisite characters $\Phi_{\lambda,k}$. But notice that $V^\lambda_{\lambda,k}$ is one-dimensional, so 
the character (c.f. definition \ref{defn:character}) of $V_{\lambda, k}$ is of the form 
\[m_\lambda(X_1,\dots,X_n)+\sum_{\lambda\ne\alpha\in\Lambda_+(n,r)} \dim V_{\lambda,l}^\alpha m_\alpha(X_1,\dots,X_n)\]
but since each $V_{\lambda,k}^\alpha$ is contained in $V_{\lambda,k}^\text{max}$, it occurs as a weight space of this maximal submodule with the 
same multiplicity. Therefore the character of $V_{\lambda,k}^\text{max}$ is 
\[\sum_{\lambda\ne\alpha\in\Lambda_+(n,r)} \dim V_{\lambda,l}^\alpha m_\alpha(X_1,\dots,X_n)\]
so we can conclude that 
\[\Phi_{V_{\lambda,k}}(X_1,\dots,X_n)=m_\lambda(X_1,\dots,X_n)=X_1^{\lambda_1}\cdots X_n^{\lambda_n}+\cdots\]
which has leading term (under the lexicographic ordering) precisely what we wanted.


\subsection{Explicit examples for comparison}
To demonstrate the theory developed above, we begin a computation (in a simple case) of the isomorphism classes of irreducible 
representations of both $S_\bbC(2,2)$ and $\frakS_2$.

\subsubsection{The symmetric group on two letters}
The representation theory (over $k=\bbC$) of $\frakS_2$ is as simple as it comes: of course $\frakS_2\cong \bbZ/2\bbZ$ and we know that 
there are $|G|$ nonisomorphic irreducible representations of an abelian group $G$ over $\bbC$. Since we are talking about a symmetric group, 
we can realize these as the trivial and sign representations, represented by the Young diagrams:
\[\ydiagram{2}\quad\text{and}\quad\ydiagram{1,1}\]

As submodules of the regular representation $k\frakS_2= k e\oplus k(1\,2)$, we can construct these as $\langle e+(1\, 2)\rangle$ (trivial representation) and $\langle e-(1\,2)\rangle$ (sign representation).

\subsubsection{The Schur algebra \texorpdfstring{$S_\bbC(2,2)$}{S(2,2)}}
Since $\ch\bbC=0$, corollary~\ref{cor:semisimple} implies that $S_\bbC(2,2)$ is semisimple, so it suffices to identify the irreducible submodules therein.
We know 
\[S=S_\bbC(2,2)\cong \bbC^2\otimes\bbC^2\]
so $\dim_\bbC S=4.$ The theory outlined above gives us that isomorphism types of irreducible modules are in bijection with compositions of 2 of length 2, meaning 
we have two isomorphism types: one corresponding to $\lambda_1=(1,1)$ and one corresponding to $\lambda_2=(2,1)$. 

Using the construction of $D_{\lambda,\bbC}$ from above, we can compute these two irreducible modules explicitly:

\begin{ex}[$\mathbf{\lambda_1=(1,1)}$]
 In this case our shape is $(1,1)$, corresponding to the Young diagram 
\[\ydiagram{1,1}\]
and then $D_{\lambda_1,\bbC}$ is spanned by the element
\[(T_l:T_{(2,1)})=\left(\ytableaushort{1,2}:\ytableaushort{2,1}\right)=c_{12}c_{21}-c_{11}c_{22}=c_{(1,2),(2,1)}-c_{(1,2),(1,2)}\in A_\bbC(2,2)\]
since all other bideterminants of this shape are zero or linearly dependent. Thus this is a one-dimensional irreducible representation.
\end{ex}
\begin{ex}[$\mathbf{\lambda_2=(2,0)}$]
Now our shape is $(2,0)$, corresponding to the diagram
\[\ydiagram{2}.\]
The bideterminants here are 
\begin{align*}
	(T_l:T_{(1,1)})=\big(\ytableaushort{1 1}:\ytableaushort{1 1}\big)=c_{11}^2\\
	(T_l:T_{(1,2)})=(T_l:T_{(2,1)})=c_{11}c_{12}\\
	(T_l:T_{(2,2)})=c_{12}^2
\end{align*}
So we have a three-dimensional irreducible representation spanned by $\langle c_{11}^2,c_{11}c_{12},c_{12}^2\rangle$.
\end{ex}
Since these are the only two Young diagrams of size two, these examples form a complete list of isomorphism classes of irreducible representations of $S_\bbC(2,2)$.

If we prefer instead to recognize our irreducibles as submodules of $E^{\otimes 2}=(k e_1\oplus k e_2)^{\otimes 2}$ (giving us a more obvious action by our algebras), 
we can use the short exact sequence 
\[0\to N\hookrightarrow E^{\otimes 2}\twoheadrightarrow D_{\lambda,\bbC}\to 0\]
to define the $N=\ker\pi$, where $\pi$ is the map defined in equation (\ref{eqn:pimap}) above.
Then we can compute the orthogonal complement to $N$ to get $V_{\lambda,\bbC}$.
We can compute:
\[V_{\lambda_1,\bbC}=\langle e_1\otimes e_2-e_2\otimes e_1\rangle\]
and
\[V_{\lambda_2,\bbC}=\langle e_1\otimes e_1, \,e_1\otimes e_2+e_2\otimes e_1, \,e_2\otimes e_2\rangle.\]



%%%%%%%%%%%%%%%%%%%%%%%%%%%%%%%%%%%%%%%%%%%%%%%%%
%%%%%%%%%%%%%%%%%%%%%%%%%%%%%%%%%%%%%%%%%%%%%%%%%
%%%%%%%%%%%%%%%%%%%%%%%%%%%%%%%%%%%%%%%%%%%%%%%%%
%%%%%%%%%%%%%%%%%%%%%%%%%%%%%%%%%%%%%%%%%%%%%%%%%
%%%%%%%%%%%%%%%%%%%%%%%%%%%%%%%%%%%%%%%%%%%%%%%%%


\newpage
\section{The Schur-Weyl Functor}
From the discussion in the last section it is evident that the combinatorics behind the representation theory of $S(n,r)$ and $\frakS_r$ have some intersections
in their use of Young tableaux and this connection is more than superficial. In fact, there is a functor relating the representations
of these two objects in the following way:
\subsection{Construction of the Schur-Weyl functor \texorpdfstring{$\calF$}{F}}

Let $V\in M_k(n,r)$ be a $S(n,r)$-representation and select any weight $\alpha\in\Lambda(n,r)$. Then the weight space (cf. rmk~\ref{rmk-weight-spaces})
\[V^\alpha=\xi_\alpha V\]
becomes a $S(\alpha)\eqdef\xi_\alpha S(n,r)\xi_\alpha$-module using the action from $S(n,r)$. Now if we allow $r\le n$ and let
\[\omega=(1,\dots,1,0,\dots,0)\in\Lambda(n,r)\]
notice that $S(\omega)$ is spanned by the elements
\[\xi_\omega\xi_{i,j}\xi_\omega,\quad i,j\in I(n,r)\]
but by the multiplication rules established in the definition of $S(n,r)$, these are nonzero precisely when 
$i$ and $j$ are both of shape $\omega$. So then since $\xi_{i,j}=\xi_{i\sigma,j\sigma}$ for all $\sigma\in\frakS_r$, we can take as
a basis of $S(\omega)$ the set 
\[\{\xi_{u\pi,u}|\pi\in\frakS_r\}\]
where $u=(1,2,\cdots,r)\in I(n,r)$.

To prove the next statement we require a computational result.
\begin{lem}\label{lem:somega-mult}
	If $u=(1,2,\dots,r)\in I(n,r)$, then for all $\pi,\sigma\in\frakS_r$,
	\[\xi_{u\pi,u}\cdot \xi_{u\sigma,u}=\xi_{u\pi\sigma,u}.\]
\end{lem}
\begin{prf}
	Using the formulas for multiplication in $S(n,r)$, recall that 
	\begin{equation}
		\xi_{u\pi,u}\cdot\xi_{u\sigma,u}=\sum Z_{i,j} \xi_{i,j}\label{eq:1}
	\end{equation}
	where 
	\[Z_{i,j}=\#\{s\in I(n,r)|(u\pi,u)\sim(i,s)\text{ and }(u\sigma,u)\sim (s,j)\}.\]
	Then for each $i,j$, since $u=(1,2,\dots,r)$ has no stabilizer in $\frakS_r$, there is a unique 
	$g$ such that $u\pi g=i$, meaning that $s=ug$. 

	But then this fixes (again a unique) $h\in\frakS_r$ such that $u\sigma h=s=u g$ whence $\sigma h= g$. 
	One computes that 
	\[u\pi\sigma h = u\pi g=i\quad\text{and}\quad uh = j\]
	therefore since in the above computation $s$ was completely determined by $i$, we have
	\[Z_{i,j}=\left\{\begin{array}{lr}
		1, &  (i,j)\sim(u\pi\sigma,u)\\
		0, & \text{otherwise}
	\end{array}\right.\]
	and the result follows.
\end{prf}
Using this result, we prove a more obviously useful statement:

\begin{lem}
	$S(\omega)\cong k\frakS(r)$.
\end{lem}
\begin{prf}
	Define the map $\varphi:S(\omega)\to k\frakS_r$ on the basis above to be 
	\[\varphi (\xi_{u\pi,u})=\pi\]
	and extending $k$-linearly.

	This is a homomorphism since 
	\[\varphi(\xi_{u\pi,u}\xi_{u\sigma,u})=\varphi(\xi_{u\pi\sigma,u})=\pi\sigma=\varphi(\xi_{u\pi,u})\varphi(\xi_{u\sigma,u})\]
	and it is bijective since it is bijective on the respective bases and is thus bijective as a linear map.
\end{prf}
The upshot of these lemmas is that one can define the \textbf{Schur-Weyl functor} 
\[\calF:M_k(n,r)\to \Rep(\frakS_r)\]
via the map that sends any representation $V$ to its $\omega$ weight space $V^\omega\in \lmod {S(\omega)}\simeq \Rep(\frakS_r)$.

\subsection{The general theory}
The idea of the Schur functor fits into a larger context: Let $S$ be a $k$-algebra and let $M\in\lmod S$. Furthermore, let $e\in S$ be a (nonzero)
idempotent. Then one can define a functor 
\[\calF:\lmod S\to\lmod {eSe}\quad\text{via}\quad V\mapsto eV.\]
An important property of this functor is 
\begin{prop}\label{prop:F-irred}
	The image of an irreducible $S$ module under the functor $\calF$ above is zero or irreducible.
\end{prop}
\begin{prf}
	Let $e\in S$ be the idempotent in the discussion above and let $W\subseteq eV$ be any nonzero $eSe$-submodule.
	Then notice that $eW$ is a nonzero $S$-module contained in $e^2V=eV$, so $eW=eV$.
	But since $eW\subseteq W$, this forces $W=eV$, so $\calF(V)$ is irreducible.
\end{prf}

Next, a discussion in Green \cite[p. 56]{green} gives us a natural thought process to follow in constructing a partial inverse to this functor. 
Let $\calG:\lmod{eSe}\to\lmod S$ be an extension of scalars: specifically, if $M\in\lmod{eSe}$, then 
\[\tilde\calG(M)=Se\otimes_{eSe}M.\]
This is clearly functorial and furthermore satisfies the property that 
\[\calF\circ\tilde\calG(M)=\calF(Se\otimes_{eSe}M)=e(Se\otimes_{eSe}M)=eSe\otimes_{eSe} M\cong e\otimes_{eSe}M\cong M\]
so it is a right inverse (up to isomorphism) to $\calF$---a good candidate for our purposes. 
\begin{rmk}
	It is easy to prove the fact, which I glossed over above, that $M\cong e\otimes M$ via the $eSe$-isomorphism $m\mapsto e\otimes m$.
\end{rmk}

What we are really looking for, however, is a functor that sends irreducible modules to irreducibles. It can be shown that $\tilde G$ 
\textit{does not} satisfy this property, so we define 
\begin{defn}
	If $M\in\lmod S$ and $e\in S$ is an idempotent, denote by $M_{(e)}$ the largest $S$-submodule of $(1-e)M$.
\end{defn}
\noindent which enables us to define the functor 
\[\calG:\lmod{eSe}\to \lmod S\quad\text{via}\quad M\mapsto \tilde\calG(M)/\tilde\calG(M)_{(e)}.\]
This leads to the result:
\begin{prop}
	If $M\in \lmod{eSe}$ is irreducible, then so is $\calG(M)$.
\end{prop}
\begin{prf}
	Let $W$ be an $S$-module such that 
	\[\tilde\calG(M)_{(e)}\subseteq W\subseteq \tilde\calG(M)\]
	Then consider multiplying by $e$ in the above inculsions:
	we get
	\[0=e\tilde\calG(M)_{(e)}\subseteq eW\subseteq e\tilde \calG(M)=\calF\circ\tilde\calG(M)\simeq M\]
	which, by the irreducibility of $M$, forces either $eW=0$ (in which case $W\subseteq\tilde\calG(M)_{(e)}$ and we are done)
	or else $eW=e\tilde\calG(M)$.

	In this latter case, we find 
	\[\tilde\calG(M)= Se\otimes M\simeq Se\otimes eSeM=S(eSe\otimes M)=S(e\tilde\calG(M))=SeW\subseteq W\]
	Thus we can conclude that $W=\tilde\calG(M)$, so $\calG(M)$ has no nontrivial proper submodules, so it is simple.
\end{prf}

\subsection{Properties of \texorpdfstring{$\calF$ and $\calG$}{F and G}}
Returning to the specific case of $S=S(n,r)$ and $eSe\cong\frakS_r$, the theory developed in the last part
gives us a pair of functors
\[\calF:\Pol(n,r)\to \lmod{\frakS_r},\qquad \calG:\lmod{\frakS_r}\to \Pol(n,r),\]
each of which preserve irreducibility. We also have that
\begin{prop}
	If $M\in \Pol(n,r)$ is irredicible and if $eM\ne 0$, then $\calG\circ\calF(M)=\calG(eM)\cong M.$
\end{prop}
\begin{prf}
	Notice by prop.~\ref{prop:F-irred} and the following discussion that $eM$ is irreducible and (by assumption) nonzero, so
	\[\calF\circ\calG(eM)\cong eM\]
	and since 
	\[0\ne eM\subseteq M\]
	and $M$ is irreducible, $eM=M$.
\end{prf}

\subsection{In positive characteristic}
Schur's classical work dealt only with the case when $k$ is a field of characteristic zero. In Aquilino and Reischuk's paper \cite{aquilino-reischuk} 
on the monoidal structure of $\lmod{S(n,d)}$, the authors mention that (in general),
\[\lmod{S(n,d)}\not\cong\lmod{\frakS_d}.\]

To fix this problem, the authors restrict attention to the ``nicely behaved ones''. Let $M^\lambda$ denote 
the $\lambda\in\Lambda(n,d)$ weight space of $E^{\otimes d}=(k^n)^{\times d}$. Then one can define 
\begin{defn}
	Let $M=\{M^\lambda|\lambda\in\Lambda(n,d)\}$ and let the category $\mathbf{add}(M)$ be the full subcategory of $\lmod{\frakS_d}$ consisting 
	of modules that are summands of finite direct sums of weight modules $M^\lambda\in M$.
\end{defn}
One can define an analogous subcategory $\mathbf{add}(S(n,d))$, and the usual Schur-Weyl functor 
\[\calF(M)=\xi_\omega M=M^\omega\]
restricts to an equivalence between the categories $\mathbf{add}(M)$ and $\mathbf{add}(S(n,d)).$

%%%%%%%%%%%%%%%%%%%%%%%%%%%%%%%%%%%%%%%%%%%%%%%%%
%%%%%%%%%%%%%%%%%%%%%%%%%%%%%%%%%%%%%%%%%%%%%%%%%
%%%%%%%%%%%%%%%%%%%%%%%%%%%%%%%%%%%%%%%%%%%%%%%%%
%%%%%%%%%%%%%%%%%%%%%%%%%%%%%%%%%%%%%%%%%%%%%%%%%
%%%%%%%%%%%%%%%%%%%%%%%%%%%%%%%%%%%%%%%%%%%%%%%%%

\newpage
\section{Strict polynomial functors}
The theory of strict polynomial functors has its genesis in the idea of \textit{polynomial maps between vector spaces},
or equivalently the rational maps between the schemes they represent. The category of vector spaces with these polynomial maps---and 
more specifically, the representaion category associated to it---gives the category $\Rep \Gamma^d_k$ of strict polynomial functors. 

Originally definied by Friedlander and Suslin in \cite{friedlander-suslin}, the authors there showed that 
the category $\lmod{S(n,r)}$ is equivalent to this category, introducing the language of polynomial functors as 
a way to understand the structure of representations of the Schur algebras. 

This process was carried out by Krause \cite{krause-strict-poly-func} and his students Aquilino and Reischuk \cite{aquilino-reischuk}.
In the former, Krause identifies projective generators $\Gamma^{d,V}$ for $\Rep\Gamma^d_k$ and defines the tensor product 
by defining it for projectives and taking the appropriate colimits. In the latter paper, the construction 
is further elucidated and it is proven that the Schur-Weyl functor $\calF$ is monoidal.

\subsection{Polynomial maps}
Let $V,W$ be vector spaces over a field $k$. There are many equivalent formulations of polynomial maps between such spaces, 
but one that this author of this paper finds particulaly motivating is the scheme-theoretic one:
\begin{defn}\label{defn:poly-maps}
	Let $V,W$ be as above. Then the set of \textbf{polynomial maps from $V$ to $W$} is defined to be 
	\[\Hom_\text{Pol}(V,W)\eqdef \Hom_{\Sch/k}(V,W).\]
\end{defn}

To make sense of this definition, one recalls that every $V\in\Vectk$ corresponds to an affine $k$-scheme 
$\Spec S^\ast(V^\vee)=V\otimes_k-$ (which we, through an abuse of notation, again denote $V$) represented by the symmetric algebra of the dual of $V$. Thus the polynomial maps
are precisely the rational maps one considers between these objects in their algebro-geometric realizations.

For reasons that will become apparent shortly, it is easier to identify the polynomial maps with elements of a vector space in the following way:
\begin{defn}
	If $V,W\in\Vectk$ are finite dimensional, a \textbf{polynomial map $f:V\to W$} can be alternatively defined as an element 
	\[f\in W\otimes S^\ast(V^\vee)\cong\Hom_{\Sch/k}(V,W)\]
	through the identifications above.
\end{defn}
\begin{rmk}
	In Friedlander and Suslin's original paper, they take this to be the first definition of a polynomial map. That this 
	agrees with the geometric definition (assuming that $V$ and $W$ are finite dimensional) follows from the following series of isomorphisms:
	\begin{align*}
		\Hom_\text{Pol}(V,W)&\eqdef\Hom_{\Sch/k}(V,W)\\
		&\simeq\Hom_{\Alg_k}(S^\ast(W^\vee),S^\ast(V^\vee))\\
		&\simeq\Hom_{k}(W^\vee,S^\ast(V^\vee))\\
		&\simeq W\otimes S^\ast(V^\vee)
	\end{align*}
	where we used above properties of affine schemes and standard facts of the linear algebra of finite dimensional vector spaces as well as the fact that 
	a map from $S^\ast(V)$ is determined uniquely by its images on $V$.
\end{rmk}

The upshot to this seemingly more \textit{ad hoc} definition is that, while it introduces the restriction of finite dimensionality (which will suffice for our 
definitions anyways), it enables us to make more simple the following idea:
\begin{defn}\label{def:homog-poly-map}
	Let $V$ and $W$ be vector spaces. Then a map $f\in \Hom_\text{Pol}(V,W)$ is called \textbf{homogeneous degree $d$} if 
	it corresponds (under the isomorphisms above) to an element 
	\[f\in W\otimes S^d(V^\vee).\]
\end{defn}
This is clearly a tangible and sensible way to define a degree $d$ map and it is less obvious how to define a property 
on the map of corresponding varieties that achieves the same goal. We will see in the next subsection other ways to define this 
notion that may appeal more to representation theorists.
\begin{ex}
	Here are some examples of polynomial maps:
	\begin{itemize}
		\item The identity (scheme) map $\id:V\to V$ is a (homogeneous degree 1) polynomial map. This corresponds to the element 
		\[\sum_{i=1}^n v_i\otimes v_i^\vee\in V\otimes S^\ast(V^\vee)\]
		where $v_1,\dots,v_n$ is a basis for $V$.
		\item If $V=\langle v_1,\dots,v_n\rangle$ and $W=\langle w_1,\dots,w_m\rangle$, the element
		\[\sum_1^m w_i\otimes (v_i^\vee\otimes v_i^\vee)\]
		gives rise to a map of algebras that sends basis element
		\[\sum_{\sigma\in\frakS_k}w_{i_{\sigma(1)}}^\vee\otimes \cdots\otimes w_{i_{\sigma(k)}}^\vee\mapsto \sum_{\sigma\in\frakS_k}v_{i_{\sigma(1)}}^\vee\otimes v_{i_{\sigma(1)}}^\vee\otimes \cdots\otimes v_{i_{\sigma(k)}}^\vee\otimes v_{i_{\sigma(k)}}^\vee\]
		which corresponds to a homogeneous degree 2 polynomial (scheme) map $V\to W$.
	\end{itemize}
\end{ex}

\subsection{The category \texorpdfstring{$\calP_k$}{Pk} of strict polynomial functors}
Before we define these categories we should describe the objects in question!
\begin{defn}
	A \textbf{strict polynomial functor} is a functor $T:\Vect_k\to \Vect_k$ such that for any $V,W\in\Vect_k$,
	the map on $\Hom$s
	\[T_{V,W}:\Hom_k(V,W)\to \Hom_k(T(V),T(W))\]
	is a polynomial map. That is,
	\[T_{V,W}\in\Hom_\text{Pol}\big(\Hom_k(V,W), \Hom_k(T(V),T(W))\big)\]
\end{defn}

Earlier I promised that we would have a more representation-theoretic interpretation of the homogeneous degree 
of a strict polynomial functor. I am nothing if I am not true to my word:
\begin{lem}[Lem. 2.2 in \cite{friedlander-suslin}]
	Let $T$ be a strict polynomial functor and let $n\ge 0$ be an integer. Then the following conditions are equivalent:
	\begin{enumerate}
		\item For any $V\in\Vectk$, any field extension $k'/k$ and any $0\ne\lambda\in k'$, the $k'$-linear 
		map $T_{k'}(\lambda\cdot 1_{V_{k'}})\in\End_{k'}(T(V)_{k'})$ coincides with $\lambda^n1_{T(V)_{k'}}$.
		\item For any $V\in\Vectk$, $n$ is the only weight of the representation of the algebraic group $\Gm$ in $T(V)$
		obtained by applying $T$ to the evident representation of $\Gm$ in $V$.
		\item For any $V,W\in\Vectk$, the polynomial map 
		\[T_{V,W}:\Hom_k(V,W)\to \Hom_k(T(V),T(W))\] 
		is homogeneous of degree $n$ (in the sense of \ref{def:homog-poly-map}).
	\end{enumerate}
\end{lem}
%\begin{prf}
%	\color{red} sketch out ideas here. Maybe just important ones.
%\end{prf}

\begin{defn}
	The category $\calP_d$ is the full subcategory 
	\[\calP_d\subset\Func(\Vectk,\Vectk)\]
	whose objects are the \textbf{strict polynomial functors of degree $d$.}
\end{defn}

We refer the reader to \cite[Thm. 3.2]{friedlander-suslin} for a proof of the following fact:
\begin{thm}\label{thm:FS-equiv}
	Let $n\ge d$. Then the map
	\[\Psi:\calP_d\to \lmod {S(n,d)}\]
	given by evaluation at $k^n$:
	\[T\mapsto T(k^n)\]
	is an equivalence of categories with quasi-inverse 
	\[M\mapsto\Gamma^{d,n}\otimes_{S(n,d)}M\]
	where $\Gamma^{d,n}=\Gamma^d\circ\Hom_k(k^n,-)$ (c.f. \ref{defn:div-powers} below).
\end{thm}

The important idea in this proof is that, for any polynomial functor $T$ and any finite-dimensional $V,W\in\Vectk$,
we get 
\begin{align*}T_{VW}&\in\Hom(T(V),T(W))\otimes S^d(\Hom(V,W)^\vee)\\
	&\cong\Hom(S^d(\Hom(V,W)^\vee)^\vee,\Hom(T(V),T(W)))\\
	&\cong\Hom(S^d(\Hom(V,W)^\vee)^\vee\otimes T(V),T(W))
\end{align*}
and by using that $\Gamma^d(X)\cong S^d(X^\vee)^\vee\eqdef (S^d)^\sharp(X)$ and letting $V=W=k^n$, we can identify a canonical map 
\[T_{k^n\,k^n}:\Gamma^d(\End(k^n))\otimes T(k^n)\to T(k^n)\]
which gives us an action of $\Gamma^d(\End(k^n))$ on $T(k^n)$ and one can see without too much trouble that 
\[\Gamma^d(\End(k^n))\cong S(n,d).\]
The rest of the proof is showing that these maps do what we want them to do.

\subsection{Strict polynomial functors... again}
Just when you thought you had enough categories to consider, Krause developed a new category that more succinctly captures
the stucture of homogeneous degree $d$ polynomial maps: there the author changes the domain of these functors 
to encode the desired properties into the functors, rather than take a subcategory of objects satisfying a condition (which 
is inherently more difficult to work with).
\begin{defn}\label{defn:div-powers}
	When $k$ is any commutative ring, one can define the category $P_k\subseteq \lmod{k}$ as the full subcategory of finitely-generated projective $k$-modules.
	In this paper, we require that $k$ is an infinite field. In this case, $P_k=\Vect_k$, but we use the former notation so that 
	it aligns more closely with Krause's work.

	Define $\Gamma^d P_k$ to be the category of \textbf{divided powers}---the objects are the same as those of $P_k$, but such that 
	\[\Hom_{\Gamma^dP_k}(V,W)=\Gamma^d\Hom_{P_k}(V,W)\]
	where $\Gamma^d X=(X^{\otimes d})^{\frakS_d}$ denotes the \textbf{$d^{\text{th}}$} divided powers of the vector space $X$.

	Finally, as a matter of notation, let 
	\[\Rep\Gamma^d_k=\Rep\Gamma^dP_k=\Func(\Gamma^dP_k,\lmod k)\]
	which we (suggestively) call the \textbf{category of homogeneous degree $d$ strict polynomial functors.}
\end{defn}
\begin{rmk}\label{rmk:action}
	Of course since $P_k=\Vectk$, an element
	\[T\in\Rep\Gamma^d_k=\Func(\Gamma^d\Vectk,\Vectk),\]
	is a functor that, on objects, is a map $\Vectk\to \Vectk$ and on morphisms is of the form 
	\[T_{VW}:\Hom_{\Gamma^d\Vectk}(V,W)=\Gamma^d\Hom_k(V,W)\to \Hom_k(T(V),T(W))\]
	which, leveraging $\otimes$-$\Hom$ adjunction, gives us a map 
	\[\Gamma^d(V,W)\otimes T(V)\to T(W)\]
	just as we got in the discussion following thm.~\ref{thm:FS-equiv}.
\end{rmk}
Using the idea in the last remark, Krause proves that there is another equivalence of categories:
\begin{thm}[{\cite[Thm. 2.10]{krause-strict-poly-func}}]
	Let $d,n$ be positive integers. Then evaluation at $k^n$ induces a functor 
	\[\Rep\Gamma^d_k\to \lmod{S(n,d)}\]
	which is an equivalence of categories when $n\ge d$.
\end{thm}
The key idea of this proof is to restrict attention to small projective generators of $\Rep\Gamma^d_k$. A class of these 
are the weight spaces $\Gamma^\lambda$ of the object $\Gamma^{d,k^n}$. Then it just remains to see that 
\[\End_{\Gamma^d_k}(\Gamma^{d,k^n})\cong S_k(n,d)\]
and the result follows.	
\subsection{The monoidal structure on \texorpdfstring{$\Rep\Gamma^d_k$}{Rep Gdk}}
One of the upshots of Krause's reformulation of strict polynomial functors is that it admits a more obvious monoidal 
structure. His construction of the tensor product on $\Rep\Gamma_k^d$ takes the following tack: notice that the Yoneda embedding 
is a map 
\[y:(\Gamma^dP_k)\op\to \Rep\Gamma^d_k\]
sending each object $V\mapsto \Hom_{\Gamma^dP_k}(V,-)$. Furthermore, the embedding $y$ is dense!
\begin{lem}\label{lem:yoneda-dense}
	Given a small category $\calC$, let $y$ be the Yoneda embedding 
	\[y:(\calC)\to \Func(\calC\op,\Set)=\PreSh(\calC).\]
	Then every element in $\PreSh(\calC)$ is (in a canonical way) a colimit of elements in the image of $y$. That is, for some collection of $C_i\in \calC$,
	\[X=\colim_{\longrightarrow i}y(C_i)\]
\end{lem}
To prove this lemma, let us remind you of the construction called the \textbf{category of elements} of a functor $\calF:\calC\op\to \Set$. It elements are 
pairs $(C,x)$ where $C\in\calC$ and $x\in \calF(C)$ is a point. 

Morphisms between two objects
\[f:(C,x)\to (C',y)\]
are honest morphisms $f:C\to C'$ in $\calC$ such that the set morphism
\[\calF(f):\calF(C')\to \calF(C)\]
has the property that 
\[\calF(f)(y)=x.\]
This category gives us a way to work with elements of a category ``locally'' even if the category $\calC$ 
is not concrete.

\begin{prf}[of \ref{lem:yoneda-dense}]
	The setup (but not the details) for following proof comes from one in \textit{Sheaves in Geometry and Logic} \cite[41-43]{maclane-moerdijk}. 
	
	Define the functor
	\[R:\PreSh(\calC)\to \PreSh(\calC)\quad\text{via}\quad E\mapsto \Hom_{\hat\calC}(y(-),E).\]
	Define also the opposing functor 
	\[L:\PreSh(\calC)\to \PreSh(\calC)\quad\text{via}\quad F\mapsto \colim \calD_F\]
	where $\calD_F$ is the diagram  
	\[\int_\calC F\xrightarrow{\pi_\calC}\calC\xrightarrow{y}\hat\calC\]
	(this is makes sense since $\Set$ is cocomplete).

	Now we claim that $L\ladjointto R$ are a pair of adjoint functors. To prove this, it suffices to show that
	\[\Hom_{\PreSh(\calC)}(F,R(E))\cong \Hom_{\PreSh(\calC)}(L(F),E)\]
	for all $F,E\in\PreSh(\calC)$.

	But notice that maps from the colimit of a diagram to $E$ are in bijection with cones under the diagram (i.e. cocones)
	with nadir $E$ by the universal property of colimits! So 
	\[\Hom(L(F),E)=\Hom(\colim\calD_F,E)=\operatorname{cocone}(\calD_F,E)\]
	where an element of $\operatorname{cocone}(\calD_F,E)$ is a collection of maps $(\varphi_{(C,x)})_{(C,x)\in\int_\calC F}$
	such that for all morphisms $\alpha:(C',x')\to (C,x)$ in $\int_\calC F$, the following diagram commutes 
	\begin{center}
		\begin{tikzcd}
			\calD_F(C,x)\ar[rr,"\calD_F(\alpha)"]\ar[swap,dr,"\varphi_{(C,x)}"] & &\calD_F(C',y)\ar[dl,"\varphi_{(C',y)}"]\\
			& E &
		\end{tikzcd}
	\end{center}
	Using these maps, we can construct a natural transformation $\eta:F\to \Hom(y(-),E)$
	in the following way: for each $C\in\calC$, let 
	\[\eta_C:F(C)\to \Hom(y(C),E)\quad\text{such that}\quad \eta_C(x)=\varphi_{(C,x)}\in\Hom(y(C),E).\]
	This assembles to an honest natural transformation since for each $C,C'\in\calC$, and morphism $f:C\to C'$, we have a diagram
	\begin{center}
		\begin{tikzcd}
			F(C)\ar[r,"x\mapsto \varphi_{(C,x)}"] & \Hom(y(C),E)\\
			F(C')\ar[u,"F(f)"]\ar[r,"x'\mapsto \varphi_{(C',x')}",swap] & \Hom(y(C'),E)\ar[u,"{\Hom(y(f),E)}",swap]
		\end{tikzcd}
	\end{center}
	which commutes since (by the commutativity of the colimit diagram above),
	\[\varphi_{(C,x)}=\varphi_{(C',x')}\circ\Hom(-,\alpha|_C)\]
	where $\alpha|_C:C\to C'$ denotes the underlying map in $\calC$ (instead of in $\int_\calC F$). Then fixing
	$x'\in F(C')$---and therefore $x=F(f)(x')\in F(C)$---we can see that naturality of $\eta$ means that 
	\[\Hom(y(f),E)\circ\eta_{C'}(x')=\Hom(y(f),E)(\varphi_{(C',x')})=\varphi_{(C',x')}\circ y(f)\]
	and (continuing in the other direction)
	\[\eta_C\circ F(f)(x')=\varphi_{(C,x)}\]
	must be the same. But in this case the map $f:C\to C'$ lifts to a map $\hat f:(C',x')\to (C,x)$ in $\int_\calC F$,
	and $\hat f|_{C}=f$ we get that the equality of these two expressions is precisely the compatibility condition of the 
	structural morphisms of the cocone.

	Thus we have showed that there is a well-defined map
	\[\Psi_{E,F}:\operatorname{cocone}(\calD_F,E)\to\Hom(F,R(E))\]
	since a natural transformation is defined by its structural maps and a cocone by its legs, this map is injective.
	It is surjective because for every $\eta:F\to R(E)$, we can define legs for a cocone:
	\[\varphi_{(C,x)}=\eta_C(x)\in\Hom(y(C),E)=\Hom(\calD_F(C,x),E).\]
	
	Next, we aim to show is natural in $F$ and $E$. If $\epsilon:E\to E'$ is a morphism, we have the diagram
	\begin{center}
		\begin{tikzcd}
			\operatorname{cocone}(\calD_F,E)\ar[d,swap,"(\varphi_a)\mapsto(\epsilon\circ\varphi_a)"]\ar[r,"\Psi_{E.F}"] &\Hom(F,R(E))\ar[d,"{\Hom(F,R(\epsilon))}"] \\
			\operatorname{cocone}(\calD_F,E')\ar[r,swap,"\Psi_{E',F}"] & \Hom(F,R(E'))
		\end{tikzcd}
	\end{center}
	But tracing along the bottom left, a cocone $(\varphi_a)_a$ under $\calD_F$ with nadir $E$ maps to the cocone $(\epsilon\circ\varphi_a)_a$
	with nadir $E'$. Under the map $\Psi_{E',F}$ just defined, this has as its image the natural transformation 
	\[\eta:F\to \Hom(-,E')\quad\text{via}\quad \eta_{C}(x)=\varphi_{(C,x)}\circ\epsilon\in\Hom(C,E')\]

	Proceeding along the top right, the same cocone with nadir $E$ is mapped to the natural transformation 
	\[\hat\eta:F\to \Hom(-,E)\quad\text{via}\quad \hat\eta_{C}(x)=\varphi_{(C,x)}\in\Hom(C,E)\]
	which is then mapped to $\varphi_{(C,x)}\circ\epsilon\in\Hom(C,E')$. This gives us naturality in $E$.

	To show naturality in $F$, let $\beta:F\to F'$ be a natural map between presheaves. Then this induces a map 
	\[\int_\calC\beta:\int_\calC F\to\int_\calC F'\quad\text{via}\quad (C,x)\mapsto (C,\beta_C(x))\]
	which, in turn, induces a map between diagrams 
	\[\calD_\beta:\calD_{F'}\to \calD_{F}\]
	where, on points, this is the map (if $x\in F(C)$)
	\[\left(\calD_\beta(\calD_{F'})\right)_C(C,x)=(\calD_{F'})_C(C,\beta_C(x))\]
	which, in turn, induces the map 
	\[\hat\beta:\operatorname{cocone}(\calD_{F'},E)\to \operatorname{cocone}(\calD_{F},E)\]
	such that, if $(\varphi_{(C,x)})_{\int_\calC F'}$ is a cone under $\calD_{F'}$ with nadir $E$, 
	the image $\rho=\hat\beta(\varphi_{(C,x)})$ is such that 
	\[\rho_{(C,x)}=\varphi_{(C,\beta_C(x))}.\]

	So naturality in $F$ is equivalent to the commutivity of 
	\begin{center}
		\begin{tikzcd}
			\operatorname{cocone}(\calD_F,E)\ar[r,"\Psi_{E,F}"] & \Hom(F, R(E))\\
			\operatorname{cocone}(\calD_{F'},E)\ar[u,"\hat\beta"]\ar[swap,r,"\Psi_{E,F'}"] & \Hom(F',R(E))\ar[u,"{\Hom(\beta,R(E))}",swap]
		\end{tikzcd}
	\end{center}
	which follows from the statement that the two natural transformations below take the same values:
	\[\left(\Psi_{E,F}\circ\hat\beta(\varphi_{\alpha})\right)_C(x)=\varphi_{(C,\beta_C(x))}\]
	and
	\[\left(\Hom(\beta,R(E))\circ\Psi_{E,F'}(\varphi_{\alpha})\right)_C(x)=(\Psi_{E,F'}(\varphi_{\alpha})\circ\beta)_C(x)=(\Psi_{E,F'}(\varphi_{(C,x)}))_C(\beta_C(x))=\varphi_{(C,\beta_C(x))}\]
	This completes the proof that $L\ladjointto R$.

	But by the Yoneda lemma, we know that 
	\[R(E)(C)=\Hom(y(C),E)\cong E(C)\]
	which implies that $R$ is naturally isomorphic to $\id_{\PreSh(\calC)}$. But adjoints, when they exist, 
	are unique! Therefore $L\simeq\id_{\PreSh(\calC)}$, or in other words for all $F\in\PreSh(\calC)$,
	\[F=\id_{\PreSh(\calC)}(F)\simeq L(F)=\colim D_F=\colim_i \Hom(-,C_i)\]
	proving the result.
\end{prf}

Notice that since 
\[\Hom_\calC(-,X)=\Hom_{\calC\op}(X,-)\]
we can replace $\calC$ with $\calC\op$ in the above argument and get that any functor in $\Func(\calC,\Set)$ is a colimit 
of (covariant) representable functors of the form $\Hom(C_i,-)$.
Thus \textbf{all} the elements in our category $\Rep \Gamma^d_k$ can be written as a colimit of 
elements of the form 
\[\Gamma^{d,V}\eqdef\Hom_{\Gamma^dP_k}(V,-)\]

Let $0\to X\to Y\to Z\to 0$ be an exact sequence of elements in $\Rep\Gamma^d_k$. Then by applying $\Gamma^{d,V}$ and 
applying the Yoneda isomorphism $\Hom(\Gamma^{d,V},X)\simeq X(V)$, we get that 
\[0\to X(V)\to Y(V)\to Z(V)\to 0\]
is exact whence 
\[0\to \Hom(\Gamma^{d,V},X)\to \Hom(\Gamma^{d,V},Y)\to\Hom(\Gamma^{d,V},Z)\to 0\]
is. This proves the fact that 
\begin{lem}
	For all $V\in\Gamma^dP_k$, $\Gamma^{d,V}$ is a projective object.
\end{lem}

From these reasonably simple objects, one defines (letting $\Gamma^d_k=\Rep\Gamma^d_k$ in what follows)
\[\Gamma^{d,V}\otimes_{\Gamma^d_k}\Gamma^{d,W}\eqdef\Gamma^{d,V\otimes W}\]
and leveraging the facts above, for each $Y\in\Gamma^dP_k$,
\[\Gamma^{d,V}\otimes_{\Gamma^d_k} Y\eqdef\colim_{\Gamma^{d,W}\to Y}\Gamma^{d,V\otimes W}\]
and finally for each $X\in \Gamma^dP_k$,
\[X\otimes_{\Gamma^d_k} Y\eqdef\colim_{\Gamma^{d,V}\to X}\Gamma^{d,V}\otimes Y.\]

One can similarly define internal hom:
\[\iHom_{\Gamma^d_k}(X,Y)\eqdef \lim_{\Gamma^{d,V}\to X}\colim_{\Gamma^{d,W}\to Y}\Gamma^{d,\Hom(V,W)}\]
which in \cite[prop 2.4]{krause-strict-poly-func} is shown to satisfy the usual adjunction:
\begin{prop}[Krause]
	If $X,Y,Z\in\Gamma^dP_k$, 
	\[\Hom_{\Gamma^d_k}(X\otimes_{\Gamma^d_k} Y,Z)\cong\Hom_{\Gamma^d_k}(X,\iHom_{\Gamma^d_k}(Y,Z))\]
\end{prop}
\subsection{Monoidicity of the Schur-Weyl functor \texorpdfstring{$\calF$}{F}}
In \cite{aquilino-reischuk}, the authors show that this is the ``correct'' monoidal structure. This is summed up 
in the primary result of their paper:
\begin{thm}[{\cite[thm. 4.4]{aquilino-reischuk}}]
	The functor 
	\[\calF=\Hom(\Gamma^\omega,-):\Rep\Gamma^d_k\to \lmod{k\frakS_d}\]
	preserves the monoidal structure defined on strict polynomial functors, i.e.
	\[\calF(X\otimes_{\Gamma_k^d}Y)\cong\calF(X)\otimes_k\calF(Y)\]
	for all $X$ and $Y$ and if $\1$ is the tensor unit, 
	\[\calF(\1_{\Rep\Gamma^d_k})=\1_{k\frakS_d}.\]
\end{thm}

The key observation in their proof of this result is that strict polynomial functors can be computed 
as limits of representable presheaves where the representing objects are \textit{free}. This is believable enough 
if we (as they do) allow $k$ to be any commutative ring. Since we are only interested in the case when $k$ is a field, however,
we have to make no such reduction.

Then a combinatorial argument connecting weights in $\Lambda(mn,d)$ to the collection of all matrices $A^\lambda_\mu$ with $\lambda\in\Lambda(n,d)$ and $\mu\in\Lambda(m,d)$ such that the $i^{th}$ column sums to $\lambda_i$ and the 
$j^{th}$ row sums to $\mu_j$. Then we observe that
\[\Hom(\Gamma^\omega,\Gamma^{d,n}\otimes\Gamma^{d_m})\cong\bigoplus_{\lambda\in\Lambda(n,d),\,\mu\in\Lambda(m,d)}\bigoplus_{A\in A^\lambda_\mu}\Hom(\Gamma^\omega,\Gamma^A)=\bigoplus_{\lambda,\mu}\bigoplus_A{^\lambda M}\]
and by a decomposition result (their lemma 3.1), then have that 
\[\bigoplus_A {^\lambda M}\cong{^\lambda M}\otimes_k{^\mu M}\]
which is the crucial step in separating into a tensor product of $\frakS_d$ modules.

%I like this idea. I will try to do it, if time allows.
%\subsection{A dictionary}
%{\color{red} Spell out how one can translate between the three different categories: irreducibles and tensor structure.}

%%%%%%%%%%%%%%%%%%%%%%%%%%%%%%%%%%%%%%%%%%%%%%%%%
%%%%%%%%%%%%%%%%%%%%%%%%%%%%%%%%%%%%%%%%%%%%%%%%%
%%%%%%%%%%%%%%%%%%%%%%%%%%%%%%%%%%%%%%%%%%%%%%%%%
%%%%%%%%%%%%%%%%%%%%%%%%%%%%%%%%%%%%%%%%%%%%%%%%%
%%%%%%%%%%%%%%%%%%%%%%%%%%%%%%%%%%%%%%%%%%%%%%%%%
\newpage
\section{Tensor products in the derived category \texorpdfstring{$\Db(S(n,r))$}{DbS(n,r)}}
(Co)homology is a powerful tool in analyzing the composition of objects and their actions. This is evidenced 
by the sheer number of cohomology theories that are in use across many different fields. Homological computations are, 
in their nature, lossy---one is reducing the object to its signature and then we play the game of gleaning what we can 
from the structure that remains. 

It is a well-known fact of homological algebra that the cohomology of an $R$-module is independent of 
resolution by projective objects. Because of this fact, if we are interested in the homological properties of modules over a ring $R$,
it isn't useful to look at the (abelian) category $\lmod R$, but rather its ``homologically-distilled'' analog,
$\D(R)$. Throughout this section we will be relying on Weibel \cite{weibel} and his discussion on chain, homotopy, and derived categories.

\subsection{Derived categories}
In what follows, let $\calA$ denote any abelian category. If it helps, the reader can relatively safely assume that $\calA$ is $\lmod R$, the category 
of (left) $R$-modules.\footnote{That one can do this is the subject of the \textit{Freyd-Mitchell embedding theorem}, which tells us that any small Abelian category
can be embedded faithfully in $\lmod R$ for some ring $R$. Even if $\calA$ isn't small (a set), one can study it via this embedding by 
restricting attention to small abelian subcategories.} Denote by $\Ch(\calA)$ (or $\Ch(R)$ when $\calA=\lmod R$) the category of chain complexes $(C_\bullet,\partial)$
such that each $C_i\in\calA$ and $\partial\circ\partial=0$. Let $\Chb(\calA)$ denote the full subcategory of $\Ch(\calA)$ consisting of the 
complexes that are bounded---that is, $C_i=0$ for all $i>N$ and $i<M$ for some $N,M$.

Recall that a chain complex morphism\footnote{A morphism that commutes with the differential.} $f_\bullet:C_\bullet\to D_\bullet$ is a \textbf{chain nullhomotopic} in $\Ch(\calA)$ if
there exist maps $\sigma_i:C_i\to D_{i+1}$ such that we have following (non-commuting) diagram:
\begin{figure}[h]
	\centering
	\begin{tikzcd}
		\cdots\ar[r,"\partial"] &C_{n+1}\ar[d,"f_{n+1}",swap]\ar[r,"\partial"] & C_n\ar[dl,"\sigma_n"]\ar[r,"\partial"]\ar[d,"f_n"] & C_{n-1}\ar[dl,"\sigma_{n-1}"]\ar[d,"f_{n-1}"]\ar[r,"\partial"] & \cdots\\
		\cdots\ar[r,"\partial",swap] &D_{n+1}\ar[r,"\partial",swap] & D_n\ar[r,"\partial",swap] & D_{n-1}\ar[r,"\partial",swap] & \cdots
	\end{tikzcd}
\end{figure}

\noindent with the condition that (for all $n$)
\[f_n=\partial\circ\sigma_n+\sigma_{n-1}\circ \partial.\]
\begin{defn}
	Two chain maps $f,g:C_\bullet\to D_\bullet$ in $\Ch(\calA)$ are said to be \textbf{chain homotopic} if their difference is chain nullhomotopic. That is, if 
	there exists maps $\sigma_i:C_i\to D_{i+1}$ such that 
	\[f_n-g_n=\partial\circ\sigma_n+\sigma_{n-1}\circ\partial.\]
\end{defn}

A well-known lemma is the following:
\begin{lem}
	If $f$ and $g$ are chain homotopic maps, then they induce the same maps on (co)homology.
\end{lem}

Chain homotopies play the role of \textbf{homotopy equivalences} (keeping in mind the example of topological spaces with simplicial homology for intuition) and 
the fact we have nontrivial homotopy equivalences is the first indication that we aren't in the right category to study homology. A natural thing to do, then, is to 
attempt to pass to a category where we identify equivalent morphisms.

\begin{defn}
	Given the category $\Ch(\calA)$, we define the \textbf{homotopy category} $\K(\calA)$ to be the category whose objects are the 
	same as those in $\Ch(\calA)$ and whose morphisms between any two chains $C_\bullet$ and $D_\bullet$ are 
	\[\Hom_{\K(\calA)}(C_\bullet,D_\bullet)\eqdef \Hom_{\Ch(\calA)}(C_\bullet,D_\bullet)/H\]
	where $H$ consists of all chain nullhomotopic maps from $C_\bullet$ to $D_\bullet$.
\end{defn}

\begin{rmk}
	We can analogously define the category $\K^\text{b}(\calA)$ that is formed through the 
	same process after first restricting to the subcategory $\Chb(\calA)$ of bounded chain complexes.
\end{rmk}

The upshot here is that we are now closer to our (until now only implicit) goal: to find a category that captures the information 
in $\Ch(\calA)$ \textit{up to quasi-isomorphism.} One can show that, however, that in general there are quasi-isomoprhisms that are not 
homotopic to the identity map! So our job is only partially complete. 

A result of great importance to reaching our goal is that $\K(\calA)$ is \textit{triangulated} with distinguished triangles given by the mapping cones 
\[A\xrightarrow{u} B\to \cone(u)\to A[1]\]
and all triangles equivalent to them\footnote{We say a triangle $X\to Y\to Z\to X[1]$ is equivalent to a mapping cone if $X,Y,Z\in\K(\calA)$ and there exists isomorphisms (equivalently, homotopy equivalences when considered as maps in $\Ch(\calA)$)
$f,g,h$ such that the diagram in fig.~\ref{fig:tri-equiv} commutes (for some $A,B$ and $u$):
}
\begin{figure}
	\centering
	\begin{tikzcd}
		X\ar[r]\ar[d,"f"] & Y\ar[r]\ar[d,"g"] & Z\ar[r]\ar[d,"h"] & X[1]\ar[d,"{f[1]}"]\\
		A\ar[r,"u"] & B\ar[r] & \cone(u)\ar[r] & A[1]
	\end{tikzcd}
	\caption{Equivalence of triangles in $\K(\calA)$}
	\label{fig:tri-equiv}
\end{figure}

The importance of triangluated categories cannot be understated (it is critical, e.g. in the construction of the Balmer spectrum in sec.~\ref{sec:ttc}). Many people, including 
Verdier (\cite{verdier-thesis}), and Neeman (\cite{neeman-duality}, \cite{neeman-book}) have put considerable time and effort into developing a 
framework within the context of triangulated categories to enable examination and manipulation. One of the tools 
that we will now use is \textit{Verdier localization}. It closesly mirrors the idea of localization of a ring at a multiplicative 
subset (a parallel that will be extended further in the following section).

\begin{defn}
	Given a triangulated category $\calT$, a \textbf{multiplicative system} $S$ in $\calT$ is a collection of morphisms 
	in $\calT$ satisfying the following properties:
	\begin{itemize}
		\item If $s,s'\in S$, so are $s\circ s'$ and $s'\circ s$ (whenever either of these make sense).
		\item $\id_X\in S$ for all $X\in\calT$
		\item (\textbf{Ore condition}) If $t\in S$ with $t:Z\to Y$ then for every $g:X\to Y$ there are maps $f$ and $s$ (with $s\in S$) such that the diagram in figure \ref{fig:fractions} commutes. The symmetric statement also holds.
		\item (\textbf{Cancellation}) If $f,g:X\to Y$ are two morphisms, then there is an $s\in S$ with $sf=sg$ if and only if there is a $t\in S$ with $ft=gt$.
	\end{itemize}
\end{defn}
\begin{figure}
	\centering
	\begin{tikzcd}
		W\ar[d,"s"]\ar[r,"f"] & Z\ar[d,"t"]\\
		X\ar[r,"g"] & Y
	\end{tikzcd}
	\caption{Ore condition in a multiplicative system}
	\label{fig:fractions}
\end{figure}
\begin{rmk}
	Under the foresight we will eventually be inverting the elements in $S$, the Ore condition translates into the following idea: for all $g:X\to Y$ and $t:Z\to Y$ in $S$,
	\[t^{-1}g=fs^{-1}\]
	for some maps $s\in S$ and $f$. This fixes the inherent noncommutativity of function composition.
\end{rmk}
\subsubsection{The calculus of fractions}
We can finally construct the Verdier localization of $\K(\calA)$ using a generalization of the calculus of 
fractions in localization of a ring. We will call a diagram of the form 
\[fs^{-1}:X\xleftarrow{s} X_1\xrightarrow{f} Y\]
where $s\in S$ a \textbf{fraction} and say that two fractions $fs^{-1}$ and $gt^{-1}$ are equivalent if there exists an element $X_3$ fitting 
into the commutative diagram below:
\begin{center}
	\begin{tikzcd}
		& X_1\ar[dl,swap,"s"]\ar[dr,"f"] &\\
		X & X_3\ar[u]\ar[l]\ar[r]\ar[d] & Y\\
		& X_2\ar[ul,"t"]\ar[ur,"g",swap] &
	\end{tikzcd}
\end{center}

Then from this we can define
\begin{defn}
	Let $\calT$ be a triangulated category and $S$ be a multiplicative system for $\calT$. Then the \textbf{Verdier localization of $\calT$ at $S$}, 
	$\calT[S^{-1}]$ is a category whose objects are the same as those of $\calT$ and whose morphisms are equivalence classes of 
	fractions of maps, as defined above.
\end{defn}

From this more general framework, we can very simply define the \textbf{derived category of an abelian category $\calA$}
to be 
\[\D(\calA)=\K(\calA)[W^{-1}]\]
where $W$ is the collection of weak homotopy equivalences (quasi-isomorphisms). For our purposes, it will suffice 
to restrict to the full triangulated subcategory $\K^\text{b}(\calA)$, giving us the \textbf{bounded derived category}
\[\Db(\calA)=\K^\text{b}(\calA)[W^{-1}].\]

\subsubsection{Tensor products in \texorpdfstring{$\Db(R)$}{Db(R)}}
In the context of $R$ (where $R$ is a $k$ algebra) modules, there is a tensor bifunctor 
\[-\otimes_R-:\rmod R\times\lmod R\to \Vectk\]
and since it is right exact, but not exact, we can take the left derived functor
\[-\otimes_R^\mathbf{L}-\eqdef \L(-\otimes_R-):\D(\rmod R)\times\D(\lmod R)\to\D(\Vectk)\]
which we call \textbf{the derived tensor product}. This can defined via a Kan extension:
\begin{defn}
	Let $\calF:\calA\to \calB$ be an additive functor between abelian categories. Then since $\calF$ preserves 
	chain homotopies, it descends to a functor $\K\calF:\K(\calA)\to \K(\calB)$. 
	
	We define the \textbf{right derived functor} (if it exists) to be a functor $\R\calF:\D(\calA)\to\D(\calB)$ along with a natural transformation $\xi:q\circ\K\calF\to \R\calF\circ q$ such that
	for any $\calG:\D(\calA)\to \D(\calB)$ and $\zeta:q\circ\K\calF\Rightarrow \calG\circ q$
	fitting into the diagram 
	\begin{center}
		\begin{tikzcd}[row sep=large]
			\K(\calA)\ar[r,"\K\calF"]\ar[dr,"q",swap] & \K(\calB)\ar[r,"q"]\ar[d,"\zeta",Rightarrow] & \D(\calB)\\
			& \D(\calA)\ar[ur,bend right=45,"\R\calF",swap]\ar[phantom,bend right=45,ur,""{name=RF}]\ar[ur,"\calG"]\ar[phantom,ur,""{name=G,below}] & \arrow[from=RF,to=G,Rightarrow,"\eta"]
		\end{tikzcd}
	\end{center}
	there exists 
	a unique $\eta:\R\calF\Rightarrow \calG$ such that $\eta q\circ \xi=\zeta$.
	In other words, 
	
	\begin{center}
		$\R\calF$ is the \textit{right Kan extension of $q\circ\K\calF$ along the localization map $q$.}
	\end{center}
	Similarly, the left derived functor $\L\calF$ can be defined as the left Kan extension of $q\circ\K\calF$ along $q$,
	satisfying the same universal property with the natural transformations reversed.
\end{defn}

This gives us a property characterizing the functor, but in practice one usually computes this via resolutions. In the simplest case, let $M,N\in\calA$ for some abelian monoidal category $\calA$. Then 
\[M[0]\otimes^\mathbf{L}_R N[0]=F_\bullet\otimes G_\bullet\]
where $F_\bullet$ and $G_\bullet$ are chain complexes quasi-isomorphic to $M[0]$ and $N[0]$, respectively (e.g. flat resolutions).

%\subsection{Compatibility of monoidal structures}

%%%%%%%%%%%%%%%%%%%%%%%%%%%%%%%%%%%%%%%%%%%%%%%%%
%%%%%%%%%%%%%%%%%%%%%%%%%%%%%%%%%%%%%%%%%%%%%%%%%
%%%%%%%%%%%%%%%%%%%%%%%%%%%%%%%%%%%%%%%%%%%%%%%%%
%%%%%%%%%%%%%%%%%%%%%%%%%%%%%%%%%%%%%%%%%%%%%%%%%
%%%%%%%%%%%%%%%%%%%%%%%%%%%%%%%%%%%%%%%%%%%%%%%%%
\newpage
\section{The (Balmer) spectrum of a tensor triangulated category}\label{sec:ttc}
In Paul Balmer's 2005 paper \cite{balmer-spc}, he developed a general framework for understanding the structure of certain 
kinds of categories that arose from the original constructions in algebraic geometry. Serving as a source of inspiration for Balmer, in \cite{friedlander-pevtsova-pi} Friedlander and Pevtsova proved
that the projective geometry of the cohomology ring of a finite group scheme can be recovered by looking at ``ideals'' in the category $\stmod G$ 
of stable $G$ modules.

Using this as a springboard, Balmer ported the definitions of ideals and prime ideals to tensor-triangulated categories (see below)
and proved a broader result that gives some tools for better analyzing familes of representaitons of finite groups (among other things).
\subsection{Some motivation and a definition}
Let $\calC$ be a symmetric monoidal (i.e. tensor) category with tensor product $\otimes$ and unit object $\1$. After giving some 
thought to the matter, one realizes that a ring is given by putting a ``compatible'' monoidal structure on top of an abelian group,
and to that end, one may consider the case when $\calC$ is also additive. 

This perspective gives us an interesting analogy between (unital, commutative) rings in algebra and category theory. Since every 
triangulated category is also additive, we can further specify that $\calC$ be triangulated:
\begin{defn}
	A \textbf{tensor-triangulated} category $\calC$ is both a symmetric moniodal category and a triangulated category such that 
	the monoidal structure preserves the triangluated structure. 

	As a reminder, such a category is equipped with a tensor product $-\otimes -:\calC\times\calC\to \calC$ and unit object $\1$, along with
	a collecton distinguished triangles $\calT$ comprised of objects in $\calC$ and shift functor (an auto-equivalence) $(-)[1]:\calC\to \calC$ such that:
	$-\otimes-$ is a triangulated (or exact) functor in each entry (it takes $\calT$ to itself).
\end{defn}

\subsubsection{Aside: Why triangulation?}
In the construction of the spectrum, we will see that the triangulated structure isn't explicitly necessary. It appears that 
one only needs a symmetric (or not!) monoidal category with all sums (at least if we are just relying on analogy to rings). A question, 
which may not have an answer yet (fully or in part) is whether changing these requirements significantly changes things. For instance, 
what happens when one tries to compute the spectrum of the abelian (symmetric monoidal) category $\Rep G$?

\subsection{Construction of the spectrum}
Once the appropriate context is identified (which is the real ingenuity of Balmer's paper), the construction 
very closely mirrors the construction seen in elementary algebraic geometry:
\begin{defn}
	Let $\calC$ be a tensor-triangulated category (TTC). Then a \textbf{(thick tensor) ideal} $I\subseteq \calC$ is a full triangulated subcategory 
	with the following conditons:
	\begin{itemize}
		\item \textit{(2-of-3 rule/Triangulation)} If $A,B,$ and $C\in\calC$ are objects that fit into a distinguished triangle
		\[A\to B\to C\to A[1]\]
		in $\calC$, and if any two of the three are objects in $I$, then so is the third.
		\item \textit{(Thickness)} If $A\in I$ is an object that splits as $A\cong B\oplus C$ in $\calC$, then both $B$ and $C$ belong to $I$.
		\item \textit{(Tensor Ideal)} If $A\in I$ and $B\in \calC$ then $A\otimes B=B\otimes A\in I$.
	\end{itemize}
\end{defn}

\begin{rmk}
	The first condition just ensures that our ideals respect the triangulated structure (and thus stability) in the parent category $\calC$. 
	The final condition is the most direct analog of an ideal and is central in the analogy between this theory and classical AG.
\end{rmk}
From here the rest of the picture is relatively straightforward:
\begin{defn}
	Let $\calC$ be a TTC as before. Then an ideal $I\subseteq\calC$ is called a \textbf{prime ideal}
	if, whenever $A\otimes B\in I$ for some $A,B\in \calC$, either $A$ or $B$ is in $I$.

	We call the collection of all primes the \textbf{spectrum} of $\calC$ and write 
	$\operatorname{Spc}(\calC)$.
\end{defn}

Here the construction varies slightly from the traditional construction of $\Spec$: we define 
\[Z(S)\eqdef\{\calP\in\Spc(\calC)|S\cap\calP=\varnothing\}\]
and define sets (for any $S\subseteq\calC$ and $A\in \calC$):
\[U(S)\eqdef \Spc(\calC)\setminus Z(S)=\{\calP\in\Spc(\calC)|S\cap \calP\ne\varnothing\}\]
and
\[\supp(A)\eqdef Z(\{A\})=\{\calP\in\Spc\calC|A\notin \calP\}\]

A routine check of the axioms shows us
\begin{lem}[2.6 of \cite{balmer-spc}]
	The sets $U(S)$ for all $S\subseteq\calC$ form a basis for a topology on $\Spc\calC$.
\end{lem}
which we call the \textbf{Zariski topology}, giving $\Spc\calC$ the structure of a topological space.

\subsection{As a locally-ringed space}
The above discussion mentions how we can construct a topological space from the set of prime thick tensor ideals 
in a TTC, but there is even more we can get: the structure of a locally-ringed space. 

To get this, we need to define the structure sheaf:
\begin{defn}
	Let $\calC$ be a tensor-triangulated category and let $\Spc\calC$ be the construction discussed above. Then the structure sheaf on $\Spc\calC$ is given by the 
	sheafification $\O_\calC$ of the presheaf 
	\[\tilde\O_\calC:\operatorname{Open}(\Spc\calC)\op\to \Ring\]
	given by 
	\[\tilde\O_\calC(U)\eqdef \End_{\calC/\calC_Z}(\1_U)\]
	where $U\subseteq\Spc\calC$ is an open set and $\calC_Z$ is the thick tensor ideal in $\calC$ supported 
	on $Z=\Spc\calC\setminus U$. The ringed space $(\Spc \calC,\O_\calC)$ is denoted $\Spec_\text{Bal} \calC$.
\end{defn}

\begin{rmk}
	That $\calC_Z$ is a thick tensor ideal requires some work, but it follows from work that Balmer does to 
	define a support data $(X,\sigma)$ on a tensor-triangulated category and showing that for any subset $Y\subset X$ of 
	its associated topological space, the following set 
	\[\{A\in\calC|\sigma(A)\subseteq Y\}\]
	is a thick tensor ideal of $\calC$ (c.f. lem.~3.4).
\end{rmk}

Balmer emphasizes that this is the ``correct'' ringed space structure to put on $\Spc\calC$. To do so, one defines an abstract support datum:
\begin{defn}
	A \textbf{support datum} for a TTC $\calC$ is a pair 
	\[(X,\sigma)\]
	where $X$ is a topological space and $\sigma:\calC\to \operatorname{closed}(X)$ is a map sending $a\mapsto\sigma_a$ such that 
	\begin{enumerate}
		\item $\sigma(0)=\varnothing$ and $\sigma(1)=X$,
		\item $\sigma(a\oplus b)=\sigma(a)\cup\sigma(b)$,
		\item $\sigma (a[1])=\sigma(a)$,
		\item $\sigma(a)\subseteq \sigma(b)\cup\sigma(c)$ for any triangle $a\to b\to c\to a[1]$,
		\item $\sigma(a\otimes b)=\sigma(a)\cap\sigma(b).$
	\end{enumerate}
\end{defn}
Using this definition, Balmer shows 
\begin{thm}[{\cite[thm. 3.2]{balmer-spc}}]
	$(\Spc\calC,\supp)$ is a support datum for $\calC$ and furthermore this support datum is terminal in the category of 
	support data for $\calC$. That is, for any other $(X,\sigma)$, there exists a unique continuous map $f:X\to \Spc\calC$ such that 
	\[\sigma(a)=f^{-1}(\supp(a)).\]
\end{thm}

To finish up the discussion of tensor-triangulated geometry, we state a couple of results originally proven in different contexts but used 
by Balmer to motivate the utility of this construction. In \cite{thomason}, the author classifies the triangulated tensor subcategories 
of $\Dperf(X)$, thereby defining the set $\Spc\Dperf(X)$. Applying Balmer's language and structure, he proved that 
\begin{thm}[{\cite[thm. 6.3(a)]{balmer-spc}}]
	If $X$ is a topologically Noetherian scheme, then (as ringed spaces)
	\[\Spec_\text{Bal}\Dperf(X)\simeq X.\]
\end{thm}

Furthermore another result from Friedlander and Pevtsova \cite{friedlander-pevtsova-pi} showed (again using 
the language of $\Spec_\text{Bal}$):
\begin{thm}[{\cite[thm. 3.6]{friedlander-pevtsova-pi},\cite[thm. 6.3(b)]{balmer-spc}}]
	Let $G$ be a finite group scheme over a field $k$. Then 
	\[\Spec_\text{Bal}(\stmod(kG))\simeq\Proj(H^\bullet(G,k))\]
	where, $\stmod(kG)$ is the full subcategory of the stable module category consisting of the finitely generated modules and $H^\bullet(G,k)=\Ext_G^\bullet(k,k)$ is the cohomology ring of $G$.
\end{thm}

%%%%%%%%%%%%%%%%%%%%%%%%%%%%%%%%%%%%%%%%%%%%%%%%%
%%%%%%%%%%%%%%%%%%%%%%%%%%%%%%%%%%%%%%%%%%%%%%%%%
%%%%%%%%%%%%%%%%%%%%%%%%%%%%%%%%%%%%%%%%%%%%%%%%%
%%%%%%%%%%%%%%%%%%%%%%%%%%%%%%%%%%%%%%%%%%%%%%%%%
%%%%%%%%%%%%%%%%%%%%%%%%%%%%%%%%%%%%%%%%%%%%%%%%%
\newpage
\section{Questions and extensions}
The following are some rough outlines of research programs that we can look into moving forward. They vary in depth and difficulty and the questions 
asked herein may not end up being the ones that are most interesting in these different areas. These do, however, provide a good starting place as we transition 
into tackling new problems.

\subsection{Computing the spectrum of \texorpdfstring{$\Db(S(n,r))$}{DbS(n,r)}}
When one is interested in understanding the representation theory of an object, one often runs into  the problem 
of algebras of ``wild'' representation type. These are the algebras whose isomorphism types of indecomposables are in 
bijection with those of $k\langle x,y\rangle$. It has been shown (accurding to \cite{bensonI}) that the representation 
theory of algebras $\Lambda$ of wild type is \textit{undecideable} in that there exists no algorithm for a Turing machine that can 
decide the truth or falsehood of a statement about $\Lambda$ modules. 

While that may seem like a dismal prospect, the (Balmer) spectrum of the derived category of Schur algebras 
gives us a little more hope. Recall that the spectrum is comprised of prime thick tensor ideals in $\Db(S(n,r))$,
which are triangulated subcategories that, among other properties, are closed under summands and extensions. This gives us a coarser 
grouping of chains of $\Lambda$-modules to work with, and (hopefully!) gives us a better chance at being able to understand 
things better.

A first step would be to compute $\Db(S(n,r))$ in the ``nice'' case where our field is characteristic zero. From there, there are interesting computations to be done 
for Schur algebras over fields of characteristic $p>0$ which could yield interesting results.

\subsection{The representation theory of \texorpdfstring{$S(n,r)$}{S(n,r)} in positive characteristic}
When we are working over an infinite field of characteristic zero, the theory of Schur algebra representations affords a relatively 
nice, clean description. However, as is pointed out in \cite{erdmann}, the representation theory of Schur algebras in positive characteristic 
can be fraught with troubles. For instance, $S(3,10)$ over a field of characteristic 5 has wild representation type.

That such troublesome algebras exist and are so readily accessible (the above example is spanned by 66 elements) indicates that there are
an endless supply of computational examples that one could try to understand and that could eventually lead to questions and conjectures concerning 
the nature of the representations of algebras of wild type.

\subsection{Representation theory of the \texorpdfstring{$q$}{q}-Schur algebra}
Recall a motiviating example (c.f. \cite{majid}) of a quantum group: $\operatorname{SL}_q(2)$, so named because it is a ``$q$-analog'' of the algebra
$\operatorname{SL}_2$. Fix some $q\in k^\times$. Then it is defined (as an algebra) as a quotient 
\[k\langle a,b,c,d\rangle/R\]
where $R$ is the ideal generated by the following relations:
\[\begin{array}{ccc}
	ca=qac & ba=qab & db=qdb\\
	dc=qcd & bc=cb & da-ad=(q-q^{-1})bc
\end{array}\]
along with the ``$q$-determinant relation''
\[ad-q^{-1}bc=1.\]
Notice that setting $q=1$ makes $a,b,c,$ and $d$ commute, so we are left with the usual special linear group.

Quantum groups and, more generally, quantum deformations of objects in commutative algebra, give mathematicians 
a way to carefully perturb objects to open up areas of research in noncommutative algebra to the the same (or similar) techniques 
used by commutative algebraists and algebraic geometers.\footnote{See, for instance Taft and Towber's \textit{Quantum deformation of flag schemes and Grassmann schemes. I. A q-deformation of the shape-algebra for GL(n)}
or the second half of my notes on the Grassmannian at \href{https://github.com/NicoCourts/Grassmannian-Notes/}{https://github.com/NicoCourts/Grassmannian-Notes/} where I summarize this paper.} 

The $q$-Schur algebras were developed by Dipper and James and eventually summarized very nicely in \cite{donkin-q-schur} in a manner that
reflects the character of \cite{green} and re-derives the classical results as a degenerate case of a more complex and 
interesting interplay between quantum $\GL_n$ and Iwahori-Hecke algebras.

These algebras (and even further generalizations) are still an area of active research. The question of identifying representation 
types of $q$-Schur algebras has been completed already by Erdmann and Nakano in \cite{erdmann-nakano}, 
but the other questions persist. In particular, one can ask questions like:
\begin{itemize}
	\item What are explicit indecomposable representations and (in the finite and tame cases) how can we classify the families of indecomposable representations of these algebras?
	\item How can we generalize the idea of Schur duality to even broader families of noncommutative quasihereditary algebras?
\end{itemize}



%%%%%%%%%%%%%%%%%%%%%%%%%%%%%%%%%%%%%%%%%%%%%%%%%
%%%%%%%%%%%%%%%%%%%%%%%%%%%%%%%%%%%%%%%%%%%%%%%%%
%%%%%%%%%%%%%%%%%%%%%%%%%%%%%%%%%%%%%%%%%%%%%%%%%
%%%%%%%%%%%%%%%%%%%%%%%%%%%%%%%%%%%%%%%%%%%%%%%%%
%%%%%%%%%%%%%%%%%%%%%%%%%%%%%%%%%%%%%%%%%%%%%%%%%
\newpage
\section*{Acknowledgements}
\label{sec:ack}
\addcontentsline{toc}{section}{\nameref{sec:ack}}
I extend my most heartfelt thanks to my advisor, Julia Pevtsova, who not only helped me immensely in setting a target 
for this project, but also introduced me to many of the classical ideas found in this paper (some times more than once). 
Her knowledge and understanding while I learned this subject has been absolutely invaluable to me.

My thanks also to my loving partner Allison, who stands beside me in good times and in bad and always patiently humors me when 
I need someone to listen to my inane ramblings.

Finally, thank you to my friends and colleagues in the University of Washington math department for many fruitful conversations 
and inspiration for ideas to investigate along the way. In particular I am indebted to (in no particular order) Thomas Carr, Sean Griffin, Sam Roven, and Cody Tipton
for all their help and support.

%%%%%%%%%%%%%%%%%%%%%%%%%%%%%%%%%%%%%%
%%%%%%%%%%  Bibliography %%%%%%%%%%%%%
%%%%%%%%%%%%%%%%%%%%%%%%%%%%%%%%%%%%%%
\newpage

\printbibliography
\addcontentsline{toc}{section}{References}

\end{document}