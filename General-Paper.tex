\documentclass[12pt]{article}

\usepackage{setspace}

\usepackage{amsmath, graphicx, color, fancyhdr, tikz-cd, mdframed, enumitem, framed, adjustbox, bbm, upgreek, xcolor, hyperref, manfnt}
\usepackage[framed,thmmarks]{ntheorem}
\usepackage[style=alphabetic, bibencoding=utf8]{biblatex}
%Set the bibliography file
\bibliography{sources}

\usepackage[T1]{fontenc}
\usepackage[urw-garamond]{mathdesign}
\usepackage{garamondx}

%Replacement for the old geometry package
\usepackage{fullpage}

%Input my definitions
\input{./mydefs.tex}

%Shade definitions
\theoremindent0cm
\theoremheaderfont{\normalfont\bfseries} 
\def\theoremframecommand{\colorbox[rgb]{0.9,1,.8}}
\newshadedtheorem{defn}[thm]{Definition}

%%%%%%%%%%%%%%%%%%%%%%%%%%%%%%%%%%%%%%%%%%%%%%%%%%%%%%%%%%%%%%%%%%%%%%
%%%%%%%%%%%%%%%%%%%%%%% Customize Below %%%%%%%%%%%%%%%%%%%%%%%%%%%%%%
%%%%%%%%%%%%%%%%%%%%%%%%%%%%%%%%%%%%%%%%%%%%%%%%%%%%%%%%%%%%%%%%%%%%%%

%header stuff
\setlength{\headsep}{24pt}  % space between header and text
\pagestyle{fancy}     % set pagestyle for document
\lhead{General Exam Paper} % put text in header (left side)
\rhead{Nico Courts} % put text in header (right side)
\cfoot{\itshape p. \thepage}
\setlength{\headheight}{15pt}
%\allowdisplaybreaks

% Document-Specific Macros


\begin{document}
%make the title page
\title{General Exam Paper \vspace{-1ex}}
\author{Nico Courts}
\date{Winter 2019}
\maketitle

\begin{abstract}
	We begin by going through a considerable amount of domain knowledge concerning representations of $\GL_n$,
	representations of $\frakS_n$, and strict polynomial functors all in service of understanding the Schur-Weyl 
	functor that relates several of these categories. From there, we investigate recent work on the part of Krause 
	and his students Aquilino and Reischuk on this functor and the fact that it is monoidal under reasonably natural monoidal structures on 
	the categories in question. Finally we ask some questions about whether the monoidal structure on strict polynomial functors 
	extends meaningfully to pathologies that arise in positive characteristic.
\end{abstract}

\tableofcontents

\newpage
\section{Introduction}
The story of this project (more-or-less) begins with Schur's doctoral thesis \cite{schur-thesis} in which he defined
polynomial representations of $\GL_n$---a theory which he developed more completely in his later paper \textit{\"Uber die 
rationalen Darstellungen der allgemeinen linearen Gruppe}\footnote{\textit{On the rational representations of the general linear group}}
\cite{schur-rational}. In these papers, Schur develops the idea of a \textbf{polynomial representation of $\GL_n$},
meaning a (finite dimensional) representation where the coefficient functions of the representing map 
\[\rho:\GL_n\to \GL(V)\]
is polynomial in each coordinate. That is, if $V=\oplus_{i=1}^n kv_i$, then for every $1\le i,j\le n$, we have the map
$r_{v_iv_j}:\GL_n\to k$ such that 
\[\rho(g)\cdot v_i=\sum_{i=1}^n r_{v_iv_j}v_j\]

\section{Representations of \texorpdfstring{$\GL_n$}{GLn} and of \texorpdfstring{$\frakS_n$}{Sn}}

\section{Strict Polynomial Functors}

\section{The Schur-Weyl Functor}

\section{Questions and Extensions}


%%%%%%%%%%%%%%%%%%%%%%%%%%%%%%%%%%%%%%
%%%%%%%%%%  Bibliography %%%%%%%%%%%%%
%%%%%%%%%%%%%%%%%%%%%%%%%%%%%%%%%%%%%%
\medskip

\printbibliography

\end{document}