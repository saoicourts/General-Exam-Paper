\documentclass[12pt]{article}

\usepackage{setspace}

\usepackage{graphicx, color, fancyhdr, tikz-cd, mdframed, enumitem, framed, adjustbox, bbm, upgreek, xcolor, manfnt}
\usepackage[framed,thmmarks]{ntheorem}
\usepackage{hyperref}
\hypersetup{
	colorlinks = true,
	linkcolor = [rgb]{0,0,0.5},
	citecolor = [rgb]{0.6,0,0},
	urlcolor = [rgb]{0,0,0.5}
}
\usepackage[style=alphabetic, bibencoding=utf8]{biblatex}
%Set the bibliography file
\bibliography{sources}

\usepackage[T1]{fontenc}
\usepackage[urw-garamond]{mathdesign}
\usepackage{garamondx}

%Replacement for the old geometry package
\usepackage{fullpage}
\usepackage{amsmath}

%Input my definitions
\input{./mydefs.tex}

%Shade definitions
\theoremindent0cm
\theoremheaderfont{\normalfont\bfseries} 
\def\theoremframecommand{\colorbox[rgb]{0.9,1,.8}}
\newshadedtheorem{defn}[thm]{Definition}

%%%%%%%%%%%%%%%%%%%%%%%%%%%%%%%%%%%%%%%%%%%%%%%%%%%%%%%%%%%%%%%%%%%%%%
%%%%%%%%%%%%%%%%%%%%%%% Customize Below %%%%%%%%%%%%%%%%%%%%%%%%%%%%%%
%%%%%%%%%%%%%%%%%%%%%%%%%%%%%%%%%%%%%%%%%%%%%%%%%%%%%%%%%%%%%%%%%%%%%%

%header stuff
\setlength{\headsep}{24pt}  % space between header and text
\pagestyle{fancy}     % set pagestyle for document
\lhead{General Exam Paper} % put text in header (left side)
\rhead{Nico Courts} % put text in header (right side)
\cfoot{\itshape p. \thepage}
\setlength{\headheight}{15pt}
%\allowdisplaybreaks

% Document-Specific Macros


\begin{document}
%make the title page
\title{General Exam Paper \vspace{-1ex}}
\author{Nico Courts}
\date{Winter 2019}
\maketitle

\begin{abstract}
	We begin by going through a considerable amount of domain knowledge concerning representations of $\GL_n$,
	representations of $\frakS_n$, and strict polynomial functors all in service of understanding the Schur-Weyl 
	functor that relates several of these categories. From there, we investigate recent work on the part of Krause 
	and his students Aquilino and Reischuk on this functor and the fact that it is monoidal under reasonably natural monoidal structures on 
	the categories in question. Finally we ask some questions about whether the monoidal structure on strict polynomial functors 
	extends meaningfully to pathologies that arise in positive characteristic.
\end{abstract}

\renewcommand{\baselinestretch}{0.75}\normalsize
\tableofcontents
\renewcommand{\baselinestretch}{1.0}\normalsize

\newpage
\section{Introduction}
\subsection{Schur and Polynomial Representations}
The story of this project (more-or-less) begins with Schur's doctoral thesis \cite{schur-thesis} in which he defined
polynomial representations of $\GL_n$---a theory which he developed more completely in his later paper \textit{\"Uber die 
rationalen Darstellungen der allgemeinen linearen Gruppe}\footnote{English: \textit{On the rational representations of the general linear group}}
\cite{schur-rational}. In these papers, Schur develops the idea of a \textbf{polynomial representation of $\GL_n$},
meaning a (finite dimensional) representation where the coefficient functions of the representing map 
\[\rho:\GL_n\to \GL(V)\]
is polynomial in each coordinate. That is, if $V=\oplus_{i=1}^n kv_i$, then for every $1\le i,j\le n$, we have the map
$r_{v_iv_j}:\GL_n\to k$ such that 
\[\rho(g)\cdot v_i=\sum_{i=1}^n r_{v_iv_j}v_j.\]

A result in \cite{schur-thesis} gives us a nice simplifying observation: if $V$ is a polynomial representation,
then $V$ decomposes as a direct sum of representations 
\[V=\bigoplus_\delta V_\delta\]
where each $V_\delta$ is a polynomial representation where the coefficient functions are \textit{homogeneous degree $\delta$}. 
This allows us to focus our attention to the structure of these $V_\delta$ as the fundamental building blocks of the theory.

One of the more surprising connections made in this theory comes from the observation that the vector space (recall $V\cong k^n$)
\[E=V^{\otimes r}\]
is made into a $(\GL_n(k),\frakS_r)$-bimodule in a very natural way, and that this bimodule gives us a way to relate 
$\rmod {\frakS_r}$ with (a subcategory of) $\lmod {\GL_n(k)}$ via the so-called \textbf{Schur-Weyl functor.}

\subsection{The Schur-Weyl Functor}
Clearly a connection between representations of two groups that are so ubiquitous in group theory and math in general 
is a stunning observation, and much effort has been expended since the late 20th century to study this functor and its 
properties---especially in how it relates the representation theory of these two groups. 

For instance, Friedlander and Suslin \cite{friedlander-suslin}
originally discussed the idea of \textbf{strict polynomial functors} and showed that the category of repesentations 
of the Schur algebra $S(n,d)$ was equivalent to the category $\calP_d$ of homogeneous degree $d$ strict polynomial functors.

In later work, Krause \cite{krause-strict-poly-func} used an alternative construction of $\calP_d$ as the category of
of reprsentations of the $d$-divided powers of the category of finitely generated projective $k$-modules. The upshot being that 
the latter object $\Gamma^d P_k$ has an obvious monoidal structure which $\calP_d$ inherits in a natural way. This new concrete 
monoidal structure opens up the field to discussing several notions of duality defined in different contexts 
and solidifying connections between them.

Krause's students Aquilino and Reischuk, in their paper \cite{aquilino-reischuk}, prove, among other facts, that 
under these natural monoidal structures the Schur-Weyl functor is in fact monoidal. This puts the theory of representations 
of these groups and algebras firmly in the realm of monoidal categories, opening up the area to new questions using 
tools from category theory.

\section{Representations of \texorpdfstring{$\GL_n$}{GLn} and of \texorpdfstring{$\frakS_n$}{Sn}}
Through the following discussion, let $\Gamma=\Gamma_k=\GL_n(k)$ for some field $k$. Let $\rho:\Gamma\to \GL(V)$ be a representation of $\Gamma.$

\section{The Schur-Weyl Functor}
From the discussion in the last section it is evident that the combinatorics behind the representation theory of $S(n,r)$ and $\frakS(r)$ have some intersections
in their use of Young tableaux and this connection is more than superficial. In fact, there is a functor relating the representations
of these two objects in the following way:
\subsection{Construction of the functor}

Let $V\in M_k(n,r)$ be a $S(n,r)$-representation and select any weight $\alpha\in\Lambda(n,r)$. Then the weight space 
\[V^\alpha=\xi_\alpha V\]
becomes a $S(\alpha)\eqdef\xi_\alpha S(n,r)\xi_\alpha$-module using the action from $S(n,r)$. Now if we allow $r\le n$ and let
\[\omega=(1,\dots,1,0,\dots,0)\in\Lambda(n,r)\]
notice that $S(\omega)$ is spanned by the elements
\[\xi_\omega\xi_{i,j}\xi_\omega,\quad i,j\in I(n,r)\]
but by the multiplication rules established in the definition of $S(n,r)$, these are nonzero precisely when 
$i$ and $j$ are both of shape $\omega$. So then since $\xi_{i,j}=\xi_{i\sigma,j\sigma}$ for all $\sigma\in\frakS_r$, we can take as
a basis of $S(\omega)$ the set 
\[\{\xi_{u\pi,u}|\pi\in\frakS_r\}\]
where $u=(1,2,\cdots,r)\in I(n,r)$.

To prove the next statement we require a computational result.
\begin{lem}\label{lem:somega-mult}
	If $u=(1,2,\dots,r)\in I(n,r)$, then for all $\pi,\sigma\in\frakS_r$,
	\[\xi_{u\pi,u}\cdot \xi_{u\sigma,u}=\xi_{u\pi\sigma,u}.\]
\end{lem}
\begin{prf}
	Using the formulas for multiplication in $S(n,r)$, recall that 
	\begin{equation}
		\xi_{u\pi,u}\cdot\xi_{u\sigma,u}=\sum Z_{i,j} \xi_{i,j}\label{eq:1}
	\end{equation}
	where 
	\[Z_{i,j}=\#\{s\in I(n,r)|(u\pi,u)\sim(i,s)\text{ and }(u\sigma,u)\sim (s,j)\}.\]
	Then for each $i,j$, since $u=(1,2,\dots,r)$ has no stabilizer in $\frakS_r$, there is a unique 
	$g$ such that $u\pi g=i$, meaning that $s=ug$. 

	But then this fixes (again a unique) $h\in\frakS_r$ such that $u\sigma h=s=u g$ whence $\sigma h= g$. 
	One computes that 
	\[u\pi\sigma h = u\pi g=i\quad\text{and}\quad uh = j\]
	therefore since in the above computation $s$ was completely determined by $i$, we have
	\[Z_{i,j}=\left\{\begin{array}{lr}
		1, &  (i,j)\sim(u\pi\sigma,u)\\
		0, & \text{otherwise}
	\end{array}\right.\]
	and the result follows.
\end{prf}
Using this result, we prove a more obviously useful statement:

\begin{lem}
	$S(\omega)\cong k\frakS(r)$.
\end{lem}
\begin{prf}
	Define the map $\varphi:S(\omega)\to k\frakS_r$ on the basis above to be 
	\[\varphi (\xi_{u\pi,u})=\pi\]
	and extending $k$-linearly.

	This is a homomorphism since 
	\[\varphi(\xi_{u\pi,u}\xi_{u\sigma,u})=\varphi(\xi_{u\pi\sigma,u})=\pi\sigma\]
	and it is bijective since it is bijective on the respective bases.
\end{prf}
The upshot of these lemmas is that one can define the functor 
\[\calF:M_k(n,r)\to \Rep(\frakS_r)\]
via the map that sends any representation $V$ to its $\omega$ weight space.


\subsection{}

\section{Strict Polynomial Functors}

\section{Questions and Extensions}

\section*{Acknowledgements}
\label{sec:ack}
\addcontentsline{toc}{section}{\nameref{sec:ack}}
I extend my most heartfelt thanks to my advisor, Julia Pevtsova, who not only helped me immensely in setting a target 
for this project, but also introduced me to many of the classical ideas found in this paper (some times more than once). 
Her knowledge and understanding while I learned this subject has been absolutely invaluable to me.

My thanks also to my loving partner, who stands besides me in good times and in bad and always patiently humors me when 
I need someone to spew math at (or a shoulder to cry on).

Finally, thank you to my friends and colleagues in the University of Washington math department for many fruitful conversations 
and for helping me to foster my natural mathematical curiosity into a raging fire.

%%%%%%%%%%%%%%%%%%%%%%%%%%%%%%%%%%%%%%
%%%%%%%%%%  Bibliography %%%%%%%%%%%%%
%%%%%%%%%%%%%%%%%%%%%%%%%%%%%%%%%%%%%%
\medskip

\printbibliography
\addcontentsline{toc}{section}{References}

\end{document}