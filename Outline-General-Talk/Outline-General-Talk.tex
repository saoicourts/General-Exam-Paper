\documentclass[12pt]{article}

\usepackage{setspace}

\usepackage{graphicx, color, fancyhdr, tikz-cd, enumitem, framed, adjustbox, bbm, upgreek, xcolor, manfnt}
\usepackage[framed,thmmarks]{ntheorem}
\usepackage[framemethod=tikz]{mdframed}
\usepackage{hyperref}

\hypersetup{
	colorlinks = true,
	linkcolor = [rgb]{0,0,0.5},
	citecolor = [rgb]{0.6,0,0},
	urlcolor = [rgb]{0,0,0.5}
}
\usepackage[style=alphabetic, bibencoding=utf8]{biblatex}
%Set the bibliography file
\bibliography{sources}

%lots of font stuff
\usepackage[T1]{fontenc}
\usepackage[urw-garamond]{mathdesign}
\usepackage{garamondx}
\let\mathcal\undefined
\newcommand{\mathcal}[1]{\text{\usefont{OMS}{cmsy}{m}{n}#1}}

%Document-Specific includes
\usepackage{ytableau}
\usepackage{mathtools}
\usepackage{scalerel}

%Replacement for the old geometry package
\usepackage{fullpage}
\usepackage{amsmath}

%Input my definitions
\input{./mydefs.tex}

%Shade definitions
\theoremindent0cm
\theoremheaderfont{\normalfont\bfseries} 
\def\theoremframecommand{\colorbox[rgb]{0.9,1,.8}}
\newshadedtheorem{defn}[thm]{Definition}

%Set apart my theorems and lemmas and such
\surroundwithmdframed[outerlinewidth=0.4pt,
  innerlinewidth=0pt,
  middlelinewidth=1pt,
  middlelinecolor=white,
  topline=false,bottomline=false,rightline=false,leftmargin=2em]{thm}
\surroundwithmdframed[outerlinewidth=0.4pt,
  innerlinewidth=0pt,
  middlelinewidth=1pt,
  middlelinecolor=white,
  topline=false,bottomline=false,rightline=false,leftmargin=2em]{lem}
\surroundwithmdframed[outerlinewidth=0.4pt,
  innerlinewidth=0pt,
  middlelinewidth=1pt,
  middlelinecolor=white,
  topline=false,bottomline=false,rightline=false,leftmargin=2em]{cor}
  \surroundwithmdframed[outerlinewidth=0.4pt,
  innerlinewidth=0pt,
  middlelinewidth=1pt,
  middlelinecolor=white,
  topline=false,bottomline=false,rightline=false,leftmargin=2em]{prop}
  \surroundwithmdframed[outerlinewidth=0.4pt,
  innerlinewidth=0,
  middlelinewidth=1pt,
  middlelinecolor=white,
  topline=false,bottomline=false,rightline=false,leftmargin=2em]{rmk}

%%%%%%%%%%%%%%%%%%%%%%%%%%%%%%%%%%%%%%%%%%%%%%%%%%%%%%%%%%%%%%%%%%%%%%
%%%%%%%%%%%%%%%%%%%%%%% Customize Below %%%%%%%%%%%%%%%%%%%%%%%%%%%%%%
%%%%%%%%%%%%%%%%%%%%%%%%%%%%%%%%%%%%%%%%%%%%%%%%%%%%%%%%%%%%%%%%%%%%%%

%header stuff
\setlength{\headsep}{24pt}  % space between header and text
\pagestyle{fancy}     % set pagestyle for document
\lhead{General Exam Outline} % put text in header (left side)
\rhead{Nico Courts} % put text in header (right side)
\cfoot{\itshape p. \thepage}
\setlength{\headheight}{15pt}
%\allowdisplaybreaks

% Document-Specific Macros
\newcommand*{\ttc}{{\large $\triangle$}\kern-0.86em\raisebox{0.3ex}{$\scaleobj{0.78}\otimes$}\hspace{1ex}}
\DeclareMathOperator{\Spc}{Spc}
\DeclareMathOperator{\Pol}{Pol}

\begin{document}
%make the title page
\title{Schur Duality and Strict Polynomial Functors}
\author{Nico Courts}
\date{Exam Presentation: March 10th, 2020, 10:15am, DEN 213}
\maketitle

%\newpage
\setcounter{tocdepth}{2}
\tableofcontents

\newpage

%%%%%%%%%%%%%%%%%%%%%%%%%%%%%%%%%%%%%%
%%%%%%%%%%%%% Part I %%%%%%%%%%%%%%%%%
%%%%%%%%%%%%%%%%%%%%%%%%%%%%%%%%%%%%%%
\section{Getting a taste}
Let's begin with some computations elucidating the phenomenon. Let $k=\bbC$ and $A=k[x_{ij}|1\le i,j\le 3]_{\det}$ be the polynomial ring localized 
at the determinant of $(x_{ij})_{i,j}$. This is a Hopf algera with coalgebra structure 
\[\Delta(x_{ij})=\sum_k x_{ik}\otimes x_{kj}\quad\text{and}\quad \varepsilon(x_{ij})=\delta_{ij}.\]

Let $A(3,3)$ be the degree $3$ graded piece (from the standard grading on polynomial rings). Then 
\[S(3,3)=A(3,3)^\vee=\Hom_k(A(3,3),k)\]
is the \textbf{(3,3)-Schur algebra.} This is a 165-dimensional algebra over $k$ and is one of the objects we are 
planning on talking about today.

\textbf{Draw out the table for reps of $\frakS_3$ and $S(3,3)$ over $\bbC$ and signal that the representation theory is the same.}
\newpage


%%%%%%%%%%%%%%%%%%%%%%%%%%%%%%%%%%%%%%
%%%%%%%%%%%%% Part II %%%%%%%%%%%%%%%%
%%%%%%%%%%%%%%%%%%%%%%%%%%%%%%%%%%%%%%
\section{Group schemes and categories}
	Next we do things a bit more elegantly. In this section we relax our definitions somewhat:
	\begin{itemize}
		\item Now $k$ can be any infinite field of characteristic zero.
		\item Now $\Gamma=\GL_n$ is the affine group scheme (over $k$)
		\item We retain the assumption that representations are finite dimensional, unless otherwise noted.
	\end{itemize}
	\subsection{The group scheme \texorpdfstring{$\GL_n$}{GLn}}
	We begin by defining the object in question. Let $A$ be the ring 
	\[A\eqdef k[x_{ij}|1\le i,j\le n]_{\det}\]
	where $\det$ is the polynomial given by computing the determinant of the matrix $(x_{ij})$.

	Then 
	\[\Gamma=\GL_n=\Hom_{\Algk}(A,-):\Algk\to\Grp\]
	is an affine group scheme. 
	\begin{defn}
		A \textbf{representation of $\Gamma$} is defined as a scheme morphism 
		\[\rho:\Gamma\to \GL(V)\eqdef \Aut(V\otimes -)\]
		where $V$ is a finite dimensional vector space over $k$.
	\end{defn}
	\begin{rmk}
		If $\dim_k V=m$, $\GL(V)$ is represented by the algebra 
		\[k[\GL(V)]\cong k[y_{ij}|1\le i,j\le m]_{\det}.\]
		Yoneda lemma tells us that a representation of $\Gamma$, equivalently, is a map of algebras 
		\[\rho^\ast:k[y_{ij}]_{\det}\to k[x_{ij}]_{\det}\]
		The upshot here is that representations are automatically ``algebraic''.
	\end{rmk}
	This enables us to make a simpler definition of polynomial and homogeneous representations:
	\begin{defn}
		Let $\rho:\Gamma\to \GL(V)$ be a representation. We say that it is a \textbf{polynomial representation} if 
		\[\Im \rho^\ast\subseteq k[x_{ij}]\subset k[\Gamma]\]
		and that $\rho$ is a \textbf{homogeneous degree $r$} if 
		\[\Im\rho^\ast\subseteq (k[x_{ij}])_r\subset k[x_{ij}]\subset k[\Gamma]\]
		where $(k[x_{ij}])_r$ is the degree $r$ graded piece of the polynomial ring.
	\end{defn}
	{\color{red} Give examples of analytic, polynomial, and homogeneous reps.}
	\begin{thm}[Schur]
		Every polynomial representation of $\Gamma$ decomposes as a direct sum of homogeneous representations.
	\end{thm}
	In the light of this new definition, it makes sense to go back and confirm that, on $k$ points it still aligns with our 
	definition. But if we evaulate at $k$, our matrices in $\GL_n(k)$ are, equivalently, maps $g:k[x_{ij}]_{\det}\to k$ and
	$\rho(g)=g\circ\rho^\ast:k[y_{ij}]_{\det}\to k$. But then if $V=\langle v_i\rangle$,
	\[g\cdot v_i=\rho(g)v_i=\sum_j \rho(g)_{ij}v_j=\sum_j g\circ\rho^\ast(y_{ij})v_j\]
	and so the structure maps of $\rho$ are 
	\[f_{ij}(g)=g\circ\rho^\ast(y_{ij})\]
	so since the image of $\rho^\ast$ lies in the homogeneous degree $r$ polynomials, we see this is the same as the original condition.

	Therefore we can also extend our old notation:
	\begin{defn}
		From now on, let $\Pol(n)$ and $\Pol(n,r)$ denote the polynomial and homogeneous degree $r$ group scheme 
		representations of $\GL_n$.
	\end{defn}

	\subsection{Comodules and the Schur algebra}
		\subsubsection{The Hopf algebra}
		The coordinate algebra $k[\Gamma]$ has not only the structure of an algebra, but of a Hopf algebra!
		The coalgebra structure is imbued by the group structure on $\Gamma$: the maps 
		\[m:\Gamma\times\Gamma\to \Gamma\quad\text{and}\quad u:\ast\to \Gamma\]
		giving multiplication and unit maps induces maps 
		\[\Delta:k[\Gamma]\to k[\Gamma]\otimes k[\Gamma]\quad\text{and}\quad \varepsilon:k[\Gamma]\to k\]
		that are given on the $x_{ij}$ as 
		\[\Delta(x_{ij})=\sum_k x_{ik}\otimes x_{kj}\quad\text{and}\quad \varepsilon(x_{ij})=\delta_{ij}\]
		which extends to an algebra morphism multiplicatively.

		\begin{thm}
			The subspace $A(n,r)\subseteq k[\Gamma]\cong k[x_{ij}]$ is a subcoalgebra.
		\end{thm}
		The key observation here is the comultiplication preserves degree.

		\subsubsection{Colagebras}
		Recall that a (right) $A=A(n,r)$ comodule is a vector space $V$ along with a map 
		\[\phi:V\to V\otimes A\]
		satisfying counit and coassociativity axioms. These coalesce into a category of left $A$ comodules with morphisms 
		$f:V\to W$ satisfying 
		\[\phi_W\circ f(v)=(f\otimes \id)\circ \phi_V(v)\]
		We denote this category $\rcomod{A}$.

		Let $(\rho,V)\in \Pol(n,k)$ using the characterization in terms of the associated algebra morphism. Then 
		we can define a right $A$-module structure on $V=\langle v_1,\dots,v_m\rangle$ by the map 
		\[\phi(v_i)=\sum_jv_j\otimes \rho^\ast(x_{ij})\] 
		
		This construction defines a map
		\[\Psi:\Pol(n,r)\to \rcomod{A(n,r)}\] 
		and a key result here is the observation that 
		\begin{thm}
			The map $\Psi$ given above is a functor and an equivalence of categories.
		\end{thm}
		\begin{rmk}
			To show that $\Gamma$-module morphisms go to comodule morphisms takes a little work, mostly in decoding 
			the different notation we've set up here. From that, the fullness and faithfulness fall out and essential surjectivity 
			follows by defining the action 
			\[g\cdot v =(\id\otimes e_g)\circ\phi(v)\]
			and one can compute this homogeneous degree $r$ (giving us our pseudo inverse).
		\end{rmk}

		\subsubsection{The Schur Algebra}
		\begin{defn}
			The \textbf{Schur algebra $S(n,r)$} is the linear dual 
			\[S(n,r)=(A(n,r))^\vee=\Hom_k(A(n,r),k).\]
		\end{defn}
		Since $A(n,r)$ is a coalgebra, this gives $S(n,r)$ an algebra structure. A standard result is the following:
		\begin{lem}
			The category of finite dimensional right comodules of a coalgebra $\Lambda$ is the same as the category 
			of finite dimensional left modules of $\Lambda^\vee$.
		\end{lem}
		\begin{cor}
			The categories $\lmod{S(n,r)}$ and $\rcomod{A(n,r)}$ are equivalent (isomorphic!).
		\end{cor}
		\begin{rmk}
			The unit $\1\in S(n,r)$ is given by $\1(x_{\underline i,\underline j})=x_{\underline i,\underline j}(I_n)$. The multiplication is, as 
			per usual, given by 
			\[(\alpha\cdot\beta)(x_{\underline i,\underline j})=\sum_{\underline k\in I(n,r)}\alpha(x_{\underline i,\underline k})\beta(x_{\underline k,\underline j}).\]
		\end{rmk}
	\subsection{Weights and irreducible representations}
	Recall the definitions of multi-indices and weights from before. Now that we are working with $S(n,r)$-modules, 
	we can give another formulation of weight spaces: Let $\alpha\in\Lambda(n,r)$ and let 
	\[\xi_\alpha=\xi_{\underline i,\underline i}\quad\text{for some}\quad \underline i\in\alpha.\]
	Then if $V\in \lmod{S(n,r)}$, the weight space corresponding to $\alpha$ is 
	\[V^\alpha=\xi_{\alpha}V.\]
	\begin{prop}
		This agrees with the original defintion of $V^\alpha$ in terms of the action by diagonal matrices.
	\end{prop}
	Then the fact (which can be checked by evaluating at the $x_{\underline i,\underline j}$) that 
	\[\1=\sum_{\alpha\in\Lambda(n,r)}\xi_\alpha\]
	gives us a proof of the fact that 
	\[V=\oplus_{\alpha\in \Lambda(n,r)}V^\alpha.\]

	\subsection{The SW functor}
	Let's begin by reformulating the Schur-Weyl functor: first we see a small lemma:
	\begin{lem}
		For $\alpha\in\Lambda(n,r)$ and $i,j\in I(n,r)$,
		\[\xi_\alpha\xi_{i,j}=\begin{cases}
			\xi_{i,j}, & i\in\alpha\\
			0, &\text{otherwise}
		\end{cases}\quad\text{and}\quad\xi_{i,j}\xi_\alpha=\begin{cases}
			\xi_{i,j}, & j\in\alpha\\
			0, &\text{otherwise}
		\end{cases}\]
	\end{lem}
	This can be checked by evaluating at $x_{\underline a,\underline b}$. Then in particular we see that $\xi_\omega^2=\xi_\omega\eqdef e$,
	an idempotent in $\Lambda\eqdef S(n,r)$. From there, there is a general theory which says the functors 
	\[\calF:\lmod{\Lambda}\to \lmod{e\Lambda e}\quad\text{via}\quad \calF(V)=eV\]
	and 
	\[\calG:\lmod{e\Lambda e}\to \lmod{\Lambda}\quad\text{via}\quad \calF(W)=\Lambda e\otimes_{e\Lambda e}W/(\Lambda e\otimes_{e\Lambda e}W)_e\]

	In this case, we can compute that for any $\xi_{\underline i,\underline j}\in S(n,r)$,
	\[\xi_\alpha\xi_{\underline i,\underline j}\xi_\alpha=\begin{cases}
		\xi_{\underline i,\underline j}& \underline i,\underline j\in \omega,\underline i\sim \underline j\\
		0 & \text{otherwise}
	\end{cases}\]
	so there exist unique $\sigma,\tau\in\frakS_d$ such that $\sigma\cdot\underline i=(1,1,\dots,1,0,\dots,0)=\underline l$ and $\tau\cdot\underline j=\underline l$,
	and so $\xi_{\underline i,\underline j}=\xi_{\underline l,\sigma\tau^{-1}\underline l}$.
	This shows us that we can extract from each element in $eAe$ an element $\sigma\tau^{-1}\in\frakS_d$ and in this way 
	we can prove 
	\begin{lem}
		If $e=\xi_\omega$ and $\Lambda=S(n,r)$,
		\[e\Lambda e\cong k\frakS_r.\]
	\end{lem}
	\begin{cor}
		The Schur Weyl functor given by $V\mapsto \xi_\omega V$ gives a functor 
		\[\calF:\lmod{S(n,r)}\to \lmod{k\frakS_r}.\]
	\end{cor}
	\begin{rmk}
		As long as his functor preserves irreducibles, as does $\calG$.
	\end{rmk}

	Schur's classical work dealt only with the case when $k$ is a field of characteristic zero. In Aquilino and Reischuk's paper \cite{aquilino-reischuk} 
	on the monoidal structure of $\lmod{S(n,d)}$, the authors mention that (in general),
	\[\lmod{S(n,d)}\not\cong\lmod{\frakS_d}.\]

	To fix this problem, the authors restrict attention to the ``nicely behaved ones''. Let $M^\lambda$ denote 
	the $\lambda\in\Lambda(n,d)$ weight space of $E^{\otimes d}=(k^n)^{\times d}$. Then one can define 
	\begin{defn}
		Let $M=\{M^\lambda|\lambda\in\Lambda(n,d)\}$ and let the category $\mathbf{add}(M)$ be the full subcategory of $\lmod{\frakS_d}$ consisting 
		of modules that are summands of finite direct sums of weight modules $M^\lambda\in M$.
	\end{defn}
	One can define an analogous subcategory $\mathbf{add}(S(n,d))$, and the usual Schur-Weyl functor 
	\[\calF(M)=\xi_\omega M=M^\omega\]
	restricts to an equivalence between the categories $\mathbf{add}(M)$ and $\mathbf{add}(S(n,d)).$

	\subsection{Strict polynomial functors}
	\begin{rmk}
		In our opinion, the best definition of these maps is as follows:
		Let $V,W$ be finite dimensional vector spaces over $k$. Then the set of \textbf{polynomial maps from $V$ to $W$} is defined to be 
		\[\Hom_\text{Pol}(V,W)\eqdef \Hom_{\Sch/k}(V,W).\]
	\end{rmk}
	
	\begin{defn}
		If $V,W\in\Vectk$ are finite dimensional, a \textbf{polynomial map $f:V\to W$} can be alternatively defined as an element 
		\[f\in W\otimes S^\ast(V^\vee)\cong\Hom_{\Sch/k}(V,W)\]
		through the identifications above. An element is said to be \textbf{homogeneous degree $d$} 
		if $f\in W\otimes S^d(V^\vee)$.
	\end{defn}
	\begin{defn}
		A \textbf{strict polynomial functor} is a functor $T:\Vect_k\to \Vect_k$ such that for any $V,W\in\Vect_k$,
		the map on $\Hom$s
		\[T_{V,W}:\Hom_k(V,W)\to \Hom_k(T(V),T(W))\]
		is a polynomial map. That is,
		\[T_{V,W}\in\Hom_\text{Pol}\big(\Hom_k(V,W), \Hom_k(T(V),T(W))\big)\]
	\end{defn}
	\begin{lem}[Lem. 2.2 in \cite{friedlander-suslin}]
		Let $T$ be a strict polynomial functor and let $n\ge 0$ be an integer. Then the following conditions are equivalent:
		\begin{enumerate}
			\item For any $V\in\Vectk$, any field extension $k'/k$ and any $0\ne\lambda\in k'$, the $k'$-linear 
			map $T_{k'}(\lambda\cdot 1_{V_{k'}})\in\End_{k'}(T(V)_{k'})$ coincides with $\lambda^n1_{T(V)_{k'}}$.
			\item For any $V\in\Vectk$, $n$ is the only weight of the representation of the algebraic group $\Gm$ in $T(V)$
			obtained by applying $T$ to the evident representation of $\Gm$ in $V$.
			\item For any $V,W\in\Vectk$, the polynomial map 
			\[T_{V,W}:\Hom_k(V,W)\to \Hom_k(T(V),T(W))\] 
			is homogeneous of degree $n$.
		\end{enumerate}
	\end{lem}
	\begin{defn}
		The category $\calP_d$ is the full subcategory 
		\[\calP_d\subset\Func(\Vectk,\Vectk)\]
		whose objects are the \textbf{strict polynomial functors of degree $d$.}
	\end{defn}
	
	We refer the reader to \cite[Thm. 3.2]{friedlander-suslin} for a proof of the following fact:
	\begin{thm}\label{thm:FS-equiv}
		Let $n\ge d$. Then the map
		\[\Psi:\calP_d\to \lmod {S(n,d)}\]
		given by evaluation at $k^n$:
		\[T\mapsto T(k^n)\]
		is an equivalence of categories with quasi-inverse 
		\[M\mapsto\Gamma^{d,n}\otimes_{S(n,d)}M\]
		where $\Gamma^{d,n}=\Gamma^d\circ\Hom_k(k^n,-)$ and $\Gamma^dV=(V^{\otimes d})^{\frakS_d}$.
	\end{thm}
	
	The important idea in this proof is that, for any polynomial functor $T$ and any finite-dimensional $V,W\in\Vectk$,
	we get 
	\begin{align*}T_{VW}&\in\Hom(T(V),T(W))\otimes S^d(\Hom(V,W)^\vee)\\
		&\cong\Hom(S^d(\Hom(V,W)^\vee)^\vee,\Hom(T(V),T(W)))\\
		&\cong\Hom(S^d(\Hom(V,W)^\vee)^\vee\otimes T(V),T(W))
	\end{align*}
	and by using that $\Gamma^d(X)\cong S^d(X^\vee)^\vee\eqdef (S^d)^\sharp(X)$ and letting $V=W=k^n$, we can identify a canonical map 
	\[T_{k^n\,k^n}:\Gamma^d(\End(k^n))\otimes T(k^n)\to T(k^n)\]
	which gives us an action of $\Gamma^d(\End(k^n))$ on $T(k^n)$ and one can see without too much trouble that 
	\[\Gamma^d(\End(k^n))\cong S(n,d).\]

	\begin{defn}\label{defn:div-powers}
		When $k$ is any commutative ring, one can define the category $P_k\subseteq \lmod{k}$ as the full subcategory of finitely-generated projective $k$-modules.
		In this paper, we require that $k$ is an infinite field. In this case, $P_k=\Vect_k$, but we use the former notation so that 
		it aligns more closely with Krause's work.
	
		Define $\Gamma^d P_k$ to be the category of \textbf{divided powers}---the objects are the same as those of $P_k$, but such that 
		\[\Hom_{\Gamma^dP_k}(V,W)=\Gamma^d\Hom_{P_k}(V,W)\]
		where $\Gamma^d X=(X^{\otimes d})^{\frakS_d}$ denotes the \textbf{$d^{\text{th}}$} divided powers of the vector space $X$.
	
		Finally, as a matter of notation, let 
		\[\Rep\Gamma^d_k=\Rep\Gamma^dP_k=\Func(\Gamma^dP_k,\lmod k)\]
		which we (suggestively) call the \textbf{category of homogeneous degree $d$ strict polynomial functors.}
	\end{defn}
	\begin{rmk}\label{rmk:action}
		Of course since $P_k=\Vectk$, an element
		\[T\in\Rep\Gamma^d_k=\Func(\Gamma^d\Vectk,\Vectk),\]
		is a functor that, on objects, is a map $\Vectk\to \Vectk$ and on morphisms is of the form 
		\[T_{VW}:\Hom_{\Gamma^d\Vectk}(V,W)=\Gamma^d\Hom_k(V,W)\to \Hom_k(T(V),T(W))\]
		which, leveraging $\otimes$-$\Hom$ adjunction, gives us a map 
		\[\Gamma^d(V,W)\otimes T(V)\to T(W)\]
		just as we got in the discussion following thm.~\ref{thm:FS-equiv}.
	\end{rmk}
	Using the idea in the last remark, Krause proves that there is another equivalence of categories:
	\begin{thm}[{\cite[Thm. 2.10]{krause-strict-poly-func}}]
		Let $d,n$ be positive integers. Then evaluation at $k^n$ induces a functor 
		\[\Rep\Gamma^d_k\to \lmod{S(n,d)}\]
		which is an equivalence of categories when $n\ge d$.
	\end{thm}
	The key idea of this proof is to restrict attention to small projective generators of $\Rep\Gamma^d_k$. A class of these 
	are the weight spaces $\Gamma^\lambda$ of the object $\Gamma^{d,k^n}$. Then it just remains to see that 
	\[\End_{\Gamma^d_k}(\Gamma^{d,k^n})\cong S_k(n,d)\]
	and the result follows.	

		\subsubsection{Monoidal structure}
		An important aspect of the Schur-Weyl functor is highlighted by Aquilino and Reischuk:
		\begin{thm}[{\cite[thm. 4.4]{aquilino-reischuk}}]
			The functor 
			\[\calF=\Hom(\Gamma^\omega,-):\Rep\Gamma^d_k\to \lmod{k\frakS_d}\]
			preserves the monoidal structure defined on strict polynomial functors, i.e.
			\[\calF(X\otimes_{\Gamma_k^d}Y)\cong\calF(X)\otimes_k\calF(Y)\]
			for all $X$ and $Y$ and if $\1$ is the tensor unit, 
			\[\calF(\1_{\Rep\Gamma^d_k})=\1_{k\frakS_d}.\]
		\end{thm}

		What is the tensor product on $\Gamma^d_k?$ It is defined in Krause's paper. We use the following lemma (fairly standard in category theory c.f. \cite{maclane-moerdijk})
		\begin{lem}\label{lem:yoneda-dense}
			Given a small category $\calC$, let $y$ be the Yoneda embedding 
			\[y:(\calC)\to \Func(\calC\op,\Set)=\PreSh(\calC).\]
			Then every element in $\PreSh(\calC)$ is (in a canonical way) a colimit of elements in the image of $y$. That is, for some collection of $C_i\in \calC$,
			\[X=\colim_{\longrightarrow i}y(C_i)\]
		\end{lem}
		The final observation is that $\Hom(y(-),E):\PreSh(\calC)\to \PreSh(\calC)$ is naturally isomorphic to the identity, meaning its adjoint
		\[F=\id_{\PreSh(\calC)}(F)\simeq L(F)=\colim D_F=\colim_i \Hom(-,C_i)=\colim_i y(C_i)\]
		Let $0\to X\to Y\to Z\to 0$ be an exact sequence of elements in $\Rep\Gamma^d_k$. Then by applying $\Gamma^{d,V}$ and 
		applying the Yoneda isomorphism $\Hom(\Gamma^{d,V},X)\simeq X(V)$, we get that 
		\[0\to X(V)\to Y(V)\to Z(V)\to 0\]
		is exact whence 
		\[0\to \Hom(\Gamma^{d,V},X)\to \Hom(\Gamma^{d,V},Y)\to\Hom(\Gamma^{d,V},Z)\to 0\]
		is. This proves the fact that 
		\begin{lem}
			For all $V\in\Gamma^dP_k$, $\Gamma^{d,V}$ is a projective object.
		\end{lem}

		One can similarly define internal hom:
		\[\iHom_{\Gamma^d_k}(X,Y)\eqdef \lim_{\Gamma^{d,V}\to X}\colim_{\Gamma^{d,W}\to Y}\Gamma^{d,\Hom(V,W)}\]
		which in \cite[prop 2.4]{krause-strict-poly-func} is shown to satisfy the usual adjunction:
		\begin{prop}[Krause]
			If $X,Y,Z\in\Gamma^dP_k$, 
			\[\Hom_{\Gamma^d_k}(X\otimes_{\Gamma^d_k} Y,Z)\cong\Hom_{\Gamma^d_k}(X,\iHom_{\Gamma^d_k}(Y,Z))\]
		\end{prop}

%%%%%%%%%%%%%%%%%%%%%%%%%%%%%%%%%%%%%%
%%%%%%%%%%%%% Part III %%%%%%%%%%%%%%%
%%%%%%%%%%%%%%%%%%%%%%%%%%%%%%%%%%%%%%
\section{Modern tools in action}
	\subsection{Triangulated and derived categories and the Balmer spectrum}
	This perspective gives us an interesting analogy between (unital, commutative) rings in algebra and category theory. Since every 
	triangulated category is also additive, we can further specify that $\calC$ be triangulated:
	\begin{defn}
		A \textbf{tensor-triangulated} category $\calC$ is both a symmetric moniodal category and a triangulated category such that 
		the monoidal structure preserves the triangluated structure. 

		As a reminder, such a category is equipped with a tensor product $-\otimes -:\calC\times\calC\to \calC$ and unit object $\1$, along with
		a collecton distinguished triangles $\calT$ comprised of objects in $\calC$ and shift functor (an auto-equivalence) $(-)[1]:\calC\to \calC$ such that:
		$-\otimes-$ is a triangulated (or exact) functor in each entry (it takes $\calT$ to itself).
	\end{defn}
	\begin{defn}
		Let $\calC$ be a tensor-triangulated category (TTC). Then a \textbf{(thick tensor) ideal} $I\subseteq \calC$ is a full triangulated subcategory 
		with the following conditons:
		\begin{itemize}
			\item \textit{(2-of-3 rule/Triangulation)} If $A,B,$ and $C\in\calC$ are objects that fit into a distinguished triangle
			\[A\to B\to C\to A[1]\]
			in $\calC$, and if any two of the three are objects in $I$, then so is the third.
			\item \textit{(Thickness)} If $A\in I$ is an object that splits as $A\cong B\oplus C$ in $\calC$, then both $B$ and $C$ belong to $I$.
			\item \textit{(Tensor Ideal)} If $A\in I$ and $B\in \calC$ then $A\otimes B=B\otimes A\in I$.
		\end{itemize}
	\end{defn}
	\begin{defn}
		Let $\calC$ be a TTC as before. Then an ideal $I\subseteq\calC$ is called a \textbf{prime ideal}
		if, whenever $A\otimes B\in I$ for some $A,B\in \calC$, either $A$ or $B$ is in $I$.
	
		We call the collection of all primes the \textbf{spectrum} of $\calC$ and write 
		$\operatorname{Spc}(\calC)$.
	\end{defn}
	Here we define
	\[Z(S)\eqdef\{\calP\in\Spc(\calC)|S\cap\calP=\varnothing\}\]
	and define sets (for any $S\subseteq\calC$ and $A\in \calC$):
	\[U(S)\eqdef \Spc(\calC)\setminus Z(S)=\{\calP\in\Spc(\calC)|S\cap \calP\ne\varnothing\}\]
	and
	\[\supp(A)\eqdef Z(\{A\})=\{\calP\in\Spc\calC|A\notin \calP\}\]
	\begin{lem}[2.6 of \cite{balmer-spc}]
		The sets $U(S)$ for all $S\subseteq\calC$ form a basis for a topology on $\Spc\calC$.
	\end{lem}
	which we call the \textbf{Zariski topology}.

	To define the structure sheaf:
	\begin{defn}
		Let $\calC$ be a tensor-triangulated category and let $\Spc\calC$ be the construction discussed above. Then the structure sheaf on $\Spc\calC$ is given by the 
		sheafification $\O_\calC$ of the presheaf 
		\[\tilde\O_\calC:\operatorname{Open}(\Spc\calC)\op\to \Ring\]
		given by 
		\[\tilde\O_\calC(U)\eqdef \End_{\calC/\calC_Z}(\1_U)\]
		where $U\subseteq\Spc\calC$ is an open set and $\calC_Z$ is the thick tensor ideal in $\calC$ supported 
		on $Z=\Spc\calC\setminus U$. The ringed space $(\Spc \calC,\O_\calC)$ is denoted $\Spec_\text{Bal} \calC$.
	\end{defn}
	Thus we have a \textbf{locally-ringed space} $\Spec\calC$ for any TTC $\calC$.

	Balmer emphasizes that this is the ``correct'' ringed space structure to put on $\Spc\calC$. To do so, one defines an abstract support datum:
	\begin{defn}
		A \textbf{support datum} for a TTC $\calC$ is a pair 
		\[(X,\sigma)\]
		where $X$ is a topological space and $\sigma:\calC\to \operatorname{closed}(X)$ is a map sending $a\mapsto\sigma_a$ such that 
		\begin{enumerate}
			\item $\sigma(0)=\varnothing$ and $\sigma(1)=X$,
			\item $\sigma(a\oplus b)=\sigma(a)\cup\sigma(b)$,
			\item $\sigma (a[1])=\sigma(a)$,
			\item $\sigma(a)\subseteq \sigma(b)\cup\sigma(c)$ for any triangle $a\to b\to c\to a[1]$,
			\item $\sigma(a\otimes b)=\sigma(a)\cap\sigma(b).$
		\end{enumerate}
	\end{defn}
	In Thomason \cite{thomason}, the author classifies the triangulated tensor subcategories 
	of $\Dperf(X)$, thereby defining the set $\Spc\Dperf(X)$. Applying Balmer's language and structure, he proved that 
	\begin{thm}[{\cite[thm. 6.3(a)]{balmer-spc}}]
		If $X$ is a topologically Noetherian scheme, then (as ringed spaces)
		\[\Spec_\text{Bal}\Dperf(X)\simeq X.\]
	\end{thm}

	Furthermore another result from Friedlander and Pevtsova \cite{friedlander-pevtsova-pi} showed (again using 
	the language of $\Spec_\text{Bal}$):
	\begin{thm}[{\cite[thm. 3.6]{friedlander-pevtsova-pi},\cite[thm. 6.3(b)]{balmer-spc}}]
		Let $G$ be a finite group scheme over a field $k$. Then 
		\[\Spec_\text{Bal}(\stmod(kG))\simeq\Proj(H^\bullet(G,k))\]
		where, $\stmod(kG)$ is the full subcategory of the stable module category consisting of the finitely generated modules and $H^\bullet(G,k)=\Ext_G^\bullet(k,k)$ is the cohomology ring of $G$.
	\end{thm}

		\subsubsection{Derived categories}
		Recall the category of chain complexes in an abelian category and chain homotopies: $g\sim h$ if $f=g-h$ is nullhomotopic:
		\begin{defn}
			Two chain maps $f,g:C_\bullet\to D_\bullet$ in $\Ch(\calA)$ are said to be \textbf{chain homotopic} if their difference is chain nullhomotopic. That is, if 
			there exists maps $\sigma_i:C_i\to D_{i+1}$ such that 
			\[f_n-g_n=\partial\circ\sigma_n+\sigma_{n-1}\circ\partial.\]
		\end{defn}

		\begin{defn}
			Given the category $\Ch(\calA)$, we define the \textbf{homotopy category} $\K(\calA)$ to be the category whose objects are the 
			same as those in $\Ch(\calA)$ and whose morphisms between any two chains $C_\bullet$ and $D_\bullet$ are 
			\[\Hom_{\K(\calA)}(C_\bullet,D_\bullet)\eqdef \Hom_{\Ch(\calA)}(C_\bullet,D_\bullet)/H\]
			where $H$ consists of all chain nullhomotopic maps from $C_\bullet$ to $D_\bullet$.
		\end{defn}
		We will call a diagram of the form 
		\[fs^{-1}:X\xleftarrow{s} X_1\xrightarrow{f} Y\]
		where $s\in S$ a \textbf{fraction} and say that two fractions $fs^{-1}$ and $gt^{-1}$ are equivalent if there exists an element $X_3$ fitting 
		into the commutative diagram below:
		\begin{center}
			\begin{tikzcd}
				& X_1\ar[dl,swap,"s"]\ar[dr,"f"] &\\
				X & X_3\ar[u]\ar[l]\ar[r]\ar[d] & Y\\
				& X_2\ar[ul,"t"]\ar[ur,"g",swap] &
			\end{tikzcd}
		\end{center}

		Then from this we can define
		\begin{defn}
			Let $\calT$ be a triangulated category and $S$ be a multiplicative system for $\calT$. Then the \textbf{Verdier localization of $\calT$ at $S$}, 
			$\calT[S^{-1}]$ (or $\calT/S$) is a category whose objects are the same as those of $\calT$ and whose morphisms are equivalence classes of 
			fractions of maps, as defined above.
		\end{defn}
		\begin{defn}
			Given a triangulated category $\calT$, a \textbf{multiplicative system} $S$ in $\calT$ is a collection of morphisms 
			in $\calT$ satisfying the following properties:
			\begin{itemize}
				\item If $s,s'\in S$, so are $s\circ s'$ and $s'\circ s$ (whenever either of these make sense).
				\item $\id_X\in S$ for all $X\in\calT$
				\item (\textbf{Ore condition}) If $t\in S$ with $t:Z\to Y$ then for every $g:X\to Y$ there are maps $f$ and $s$ (with $s\in S$) such that the diagram in figure \ref{fig:fractions} commutes. The symmetric statement also holds.
				\item (\textbf{Cancellation}) If $f,g:X\to Y$ are two morphisms, then there is an $s\in S$ with $sf=sg$ if and only if there is a $t\in S$ with $ft=gt$.
			\end{itemize}
		\end{defn}
		\begin{figure}
			\centering
			\begin{tikzcd}
				W\ar[d,"s"]\ar[r,"f"] & Z\ar[d,"t"]\\
				X\ar[r,"g"] & Y
			\end{tikzcd}
			\caption{Ore condition in a multiplicative system}
			\label{fig:fractions}
		\end{figure}
		From this more general framework, we can very simply define the \textbf{derived category of an abelian category $\calA$}
		to be 
		\[\D(\calA)=\K(\calA)[W^{-1}]\]
		where $W$ is the collection of weak homotopy equivalences (quasi-isomorphisms). For our purposes, it will suffice 
		to restrict to the full triangulated subcategory $\K^\text{b}(\calA)$, giving us the \textbf{bounded derived category}
		\[\Db(\calA)=\K^\text{b}(\calA)[W^{-1}].\]

		In the context of $R$ (where $R$ is a $k$ algebra) modules, there is a tensor bifunctor 
		\[-\otimes_R-:\rmod R\times\lmod R\to \Vectk\]
		and since it is right exact, but not exact, we can take the left derived functor
		\[-\otimes_R^\mathbf{L}-\eqdef \L(-\otimes_R-):\D(\rmod R)\times\D(\lmod R)\to\D(\Vectk)\]
		which we call \textbf{the derived tensor product}. This can defined via a Kan extension:
		\begin{defn}
			Let $\calF:\calA\to \calB$ be an additive functor between abelian categories. Then since $\calF$ preserves 
			chain homotopies, it descends to a functor $\K\calF:\K(\calA)\to \K(\calB)$. 
			
			We define the \textbf{right derived functor} (if it exists) to be a functor $\R\calF:\D(\calA)\to\D(\calB)$ along with a natural transformation $\xi:q\circ\K\calF\to \R\calF\circ q$ such that
			for any $\calG:\D(\calA)\to \D(\calB)$ and $\zeta:q\circ\K\calF\Rightarrow \calG\circ q$
			fitting into the diagram 
			\begin{center}
				\begin{tikzcd}[row sep=large]
					\K(\calA)\ar[r,"\K\calF"]\ar[dr,"q",swap] & \K(\calB)\ar[r,"q"]\ar[d,"\zeta",Rightarrow] & \D(\calB)\\
					& \D(\calA)\ar[ur,bend right=45,"\R\calF",swap]\ar[phantom,bend right=45,ur,""{name=RF}]\ar[ur,"\calG"]\ar[phantom,ur,""{name=G,below}] & \arrow[from=RF,to=G,Rightarrow,"\eta"]
				\end{tikzcd}
			\end{center}
			there exists 
			a unique $\eta:\R\calF\Rightarrow \calG$ such that $\eta q\circ \xi=\zeta$.
			In other words, 
			
			\begin{center}
				$\R\calF$ is the \textit{right Kan extension of $q\circ\K\calF$ along the localization map $q$.}
			\end{center}
			Similarly, the left derived functor $\L\calF$ can be defined as the left Kan extension of $q\circ\K\calF$ along $q$,
			satisfying the same universal property with the natural transformations reversed.
		\end{defn}

	\subsection{Extensions and problems to work on}
	The following are some rough outlines of research programs that we can look into moving forward. They vary in depth and difficulty and the questions 
	asked herein may not end up being the ones that are most interesting in these different areas. These do, however, provide a good starting place as we transition 
	into tackling new problems.

		\subsubsection{Computing the spectrum of \texorpdfstring{$\Db(S(n,r))$}{DbS(n,r)}}
		\begin{defn}
			An algebra $\Lambda$ of \textbf{wild representation type} is one whose irreducible 
			modules are in bijection with those of the free algebra $k\langle x,y\rangle$.
		\end{defn}
		\begin{thm} 
			It has been shown (according to \cite{bensonI}) that the representation theory of 
			algebras $\Lambda$ of wild type is \textit{undecideable} in that there exist statements about $\Lambda$-modules for which no algorithm for a Turing machine exists
			that candecide the truth or falsehood of a statement in finite time.
		\end{thm} 

		Instead we can use spectra! These gives us a coarser tool to work with and understanding the thick TT subcategories of 
		the representation category may give us a way at grasping what this looks like.

		A first step would be to compute $\Db(S(n,r))$ in the ``nice'' case where our field is characteristic zero. From there, there are interesting computations to be done 
		for Schur algebras over fields of characteristic $p>0$ which could yield interesting results.

		\subsubsection{The representation theory of \texorpdfstring{$S(n,r)$}{S(n,r)} in positive characteristic}
		\begin{thm}[\cite{erdmann}]
			$S(3,10)$ over a field of characteristic 5 has wild representation type.
		\end{thm}

		This has dimension 66! This gives us an ample supply of computationally-tangible objects whose representation theory 
		could shine light on the representation theory of wild type algebras in general.

		\subsubsection{Representation theory of the \texorpdfstring{$q$}{q}-Schur algebra}
		Recall a motiviating example (c.f. \cite{majid}) of a quantum group: $\operatorname{SL}_q(2)$, so named because it is a ``$q$-analog'' of the algebra of functions on
		$\operatorname{SL}_2$. Fix some $q\in k^\times$. Then it is defined (as an algebra) as a quotient 
		\[k\langle a,b,c,d\rangle/R\]
		where $R$ is the ideal generated by the following relations:
		\[\begin{array}{ccc}
			ca=qac & ba=qab & db=qdb\\
			dc=qcd & bc=cb & da-ad=(q-q^{-1})bc
		\end{array}\]
		along with the ``$q$-determinant relation''
		\[ad-q^{-1}bc=1.\]
		Notice that setting $q=1$ makes $a,b,c,$ and $d$ commute, so we are left with the usual special linear group.

		Quantum groups and, more generally, quantum deformations of objects in commutative algebra, give mathematicians 
		a way to carefully perturb objects to open up areas of research in noncommutative algebra to the the same (or similar) techniques 
		used by commutative algebraists and algebraic geometers.\footnote{See, for instance Taft and Towber's \textit{Quantum deformation of flag schemes and Grassmann schemes. I. A q-deformation of the shape-algebra for GL(n)}
		or the second half of my notes on the Grassmannian at \href{https://github.com/NicoCourts/Grassmannian-Notes/}{https://github.com/NicoCourts/Grassmannian-Notes/} where I summarize this paper.} 

		The $q$-Schur algebras were developed by Dipper and James and eventually summarized very nicely in \cite{donkin-q-schur} in a manner that
		reflects the character of \cite{green} and re-derives the classical results as a degenerate case of a more complex and 
		interesting interplay between quantum $\GL_n$ and Iwahori-Hecke algebras.

		These algebras (and even further generalizations) are still an area of active research. The question of identifying representation 
		types of $q$-Schur algebras has been completed already by Erdmann and Nakano in \cite{erdmann-nakano}, 
		but the other questions persist. In particular, one can ask questions like:
		\begin{itemize}
			\item What are explicit indecomposable representations and (in the finite and tame cases) how can we classify the families of indecomposable representations of these algebras?
			\item How can we generalize the idea of Schur duality to even broader families of noncommutative quasihereditary algebras?
		\end{itemize}


%%%%%%%%%%%%%%%%%%%%%%%%%%%%%%%%%%%%%%
%%%%%%%%%%  Bibliography %%%%%%%%%%%%%
%%%%%%%%%%%%%%%%%%%%%%%%%%%%%%%%%%%%%%
\newpage

\printbibliography
\addcontentsline{toc}{section}{References}

\end{document}